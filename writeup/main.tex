\documentclass{article}
\usepackage[T1]{fontenc}
\usepackage[utf8]{inputenc}
\usepackage[american]{babel}
\usepackage[nonatbib]{neurips_2019}
\usepackage{amsmath}
\usepackage{amsthm}
\usepackage{url}
\usepackage{booktabs}
\usepackage{amsfonts}
\usepackage{nicefrac}
\usepackage{microtype}
\usepackage{enumitem}
\usepackage{etoolbox}
\usepackage{amstext}
\usepackage{amssymb}
\usepackage{csquotes}
\usepackage{stmaryrd}
\usepackage{xparse}

%%\usepackage{color}
\usepackage{xcolor}
%\usepackage[]{todonotes}
%\setlength{\marginparwidth}{2cm}

\usepackage{appendix}
\usepackage{graphicx}
\usepackage{epstopdf}
\usepackage{relsize}
\usepackage{subcaption}




\usepackage[textwidth=2cm, textsize=footnotesize]{todonotes}  
%\setlength{\marginparwidth}{1.5cm}               %  this goes with todonotes

%\usepackage{amsthm}
%\usepackage[noabbrev,capitalize]{cleveref}

\usepackage[colorlinks,linkcolor={red!80!black},citecolor={green!80!black},urlcolor={blue!80!black},hypertexnames=false]{hyperref}

\usepackage[noabbrev,capitalize]{cleveref}
%% make \cref{whatever} render like \eqref{whatever}
\crefformat{equation}{(#2#1#3)}
\crefrangeformat{equation}{(#3#1#4) to~(#5#2#6)}
\crefmultiformat{equation}{(#2#1#3)}{ and~(#2#1#3)}{, (#2#1#3)}{ and~(#2#1#3)}
%% https://tex.stackexchange.com/a/121055/9019
\crefname{appsec}{Appendix}{Appendices}

% \usepackage{autonum}

%\renewcommand\equationautorefname{\@gobble}
%\usepackage[backend=biber,style=numeric,maxcitenames=2,maxbibnames=8,sortcites,isbn=false,doi=false,giveninits=true]{biblatex}
%\renewbibmacro{in:}{}
%\usepackage{utils/patch_biblatex}
\usepackage[backend=bibtex,style=numeric,maxcitenames=2,maxbibnames=8,sortcites,isbn=false,doi=false,giveninits=true]{biblatex}
%\usepackage[backend=bibtex]{biblatex}


\addbibresource{biblio.bib}


% Example definitions.
% --------------------
\def\x{{\mathbf x}}
\def\L{{\cal L}}

\DeclareMathOperator{\var}{\mathbb Var}
\DeclareMathOperator{\cov}{cov}
\DeclareMathOperator{\rank}{rank}
\DeclareMathOperator{\dom}{dom}
\DeclareMathOperator{\zer}{zer}
\DeclareMathOperator{\aver}{av}
\DeclareMathOperator{\inter}{int}
\DeclareMathOperator{\relint}{ri}
\DeclareMathOperator{\epi}{epi}
\DeclareMathOperator{\graph}{gr}
\DeclareMathOperator{\prox}{prox}
\DeclareMathOperator{\tr}{tr}
\DeclareMathOperator{\support}{supp}
\DeclareMathOperator{\dist}{dist}
\DeclareMathOperator{\lev}{lev}
\DeclareMathOperator{\rec}{rec}
\DeclareMathOperator{\cl}{cl}
\DeclareMathOperator{\co}{co}
\DeclareMathOperator{\clo}{\overline co}
\DeclareMathOperator{\distC}{\mathsf d}
\DeclareMathOperator*{\diag}{diag}
\newcommand{\KL}{\mathop{\mathrm{KL}}\nolimits}


\newcommand{\leftnorm}{\left|\!\left|\!\left|}
\newcommand{\rightnorm}{\right|\!\right|\!\right|}

%\newcommand{\eqdef}{{\stackrel{\text{def}}{=}}} 
\newcommand{\eqdef}{:=} 

\newcommand{\1}{\mathbbm 1}
\newcommand{\bs}{\boldsymbol}

\newcommand{\itpx}{{\mathsf x}}
\newcommand{\sx}{{\mathsf x}}
\newcommand{\sy}{{\mathsf y}}
\newcommand{\sz}{{\mathsf z}}
\newcommand{\sw}{{\mathsf w}}
\newcommand{\sF}{{\mathsf F}}
\newcommand{\sH}{{\mathsf H}}

\newcommand{\ZZ}{\mathbb Z}
\newcommand{\CC}{\mathbb{C}}
\newcommand{\bP}{{{\mathbb P}}} 
\newcommand{\bE}{{{\mathbb E}}} 
\newcommand{\bV}{{{\mathbb V}}} 
\newcommand{\bN}{{{\mathbb N}}} 

% Operators, domains, etc.  
\newcommand{\mA}{{\mathcal A}} 
\newcommand{\mB}{{\mathcal B}} 
\newcommand{\mC}{{\mathcal C}} 
\newcommand{\mD}{{\mathcal D}} 
\newcommand{\mO}{{\mathcal O}} 
\newcommand{\mU}{{\mathcal U}}
\newcommand{\mX}{{\mathcal X}}
\newcommand{\mY}{{\mathcal Y}}
\newcommand{\mZ}{{\mathcal Z}} 
\newcommand{\bmD}{\cl({\mathcal D})} 

\newcommand{\sA}{{\mathsf A}}
\newcommand{\sB}{{\mathsf B}}
\newcommand{\sJ}{{\mathsf J}}
\newcommand{\sX}{{\mathsf X}}
\newcommand{\sG}{{\mathsf G}}
\newcommand{\sY}{{\mathsf Y}}

\newcommand{\maxmon}{{\mathscr M}} 
\newcommand{\Selec}{{\mathfrak S}} 

% Sigma fields
\newcommand{\mcA}{{\mathscr A}} 
\newcommand{\mcB}{{\mathscr B}} 
\newcommand{\mcN}{{\mathscr N}} 
\newcommand{\mcT}{{\mathscr T}} 
\newcommand{\mcI}{{\mathscr I}} 
\newcommand{\mcF}{{\mathscr F}} 
\newcommand{\mcG}{{\mathscr G}} 
\newcommand{\mcX}{{\mathscr X}} 
\newcommand{\cP}{{{\mathcal P}}} 
\newcommand{\cS}{{{\mathcal S}}} 
\newcommand{\cZ}{{{\mathcal Z}}} 
\newcommand{\cF}{{{\mathcal F}}} 
\newcommand{\cG}{{{\mathcal G}}} 
\newcommand{\cM}{{{\mathcal M}}} 
\newcommand{\cD}{{{\mathcal D}}} 
\newcommand{\cE}{{{\mathcal E}}} 
\newcommand{\cL}{{{\mathcal L}}}
\newcommand{\cT}{{{\mathcal T}}} 
\newcommand{\cN}{{{\mathcal N}}} 
\newcommand{\cK}{{{\mathcal K}}} 
\newcommand{\cI}{{{\mathcal I}}} 

% Spaces 
\newcommand{\R}{{{\mathbb R}}} 
\newcommand{\E}{{{\mathbb E}}} 
\newcommand{\kH}{{{\mathcal H}}} 
\newcommand{\X}{{{\mathcal X}}} 
\newcommand{\F}{{{\mathcal F}}} 
\newcommand{\Hil}{E}                % Hilbert   
\newcommand{\Ban}{E}                % Banach   
%\newcommand{\RN}{{{\mathbb R}^N}} 
\newcommand{\bR}{{{\mathbb R}}} 

\newcommand{\m}{\mathfrak{m}}
\newcommand{\toL}{\xrightarrow[]{{\mathcal L}}}
\newcommand{\toweak}{\xrightharpoonup[]{{\mathcal L}}}

\newcommand{\ps}[1]{\langle #1 \rangle}
\newcommand{\psh}[1]{\langle #1 \rangle_{\kH}}
% 
% Almost sure convergence
\newcommand{\toasshort}{\stackrel{\text{as}}{\to}}
\newcommand{\toaslong}{\xrightarrow[n\to\infty]{\text{a.s.}}}

% Convergence in probability 
\newcommand{\toprobashort}{\,\stackrel{\mathcal{P}}{\to}\,}
\newcommand{\toprobalong}{\xrightarrow[n\to\infty]{\mathcal P}}
%
% Convergence in law 
\newcommand{\todistshort}{{\stackrel{\mathcal{D}}{\to}}}
\newcommand{\todistlong}{\xrightarrow[n\to\infty]{\mathcal D}}


\newcommand{\aknote}[1]{\todo[color=cyan!20]{#1}}
\newcommand{\asnote}[1]{\todo[color=green!20]{#1}}
\newcommand{\manote}[1]{\todo[color=magenta]{#1}}

\newcommand*\diff{\mathop{}\!\mathrm{d}}
\newcommand*\Diff[1]{\mathop{}\!\mathrm{d^#1}}

%Moreau
\newcommand{\my}{{{\nabla ^\gamma g}}}
\newcommand{\myn}{{{\nabla ^{\gamma_{n+1}} g}}}
\def\macom#1{{\textcolor{red}{[MA: #1]}}}
%% ==============================================================


%\theoremstyle{definition}
\makeatother
\newtheorem{theorem}{Theorem}
\newtheorem{lemma}[theorem]{Lemma}
\newtheorem{corollary}[theorem]{Corollary}
\newtheorem{proposition}[theorem]{Proposition}
\newtheorem{definition}{Definition}
\newtheorem{remark}{Remark}
\newtheorem{condition}{Condition}
\newtheorem{assumption}{Assumption}
\newtheorem{example}{Example}

%% make \cref{whatever} render like \eqref{whatever}
\crefformat{equation}{(#2#1#3)}
\crefrangeformat{equation}{(#3#1#4) to~(#5#2#6)}
\crefmultiformat{equation}{(#2#1#3)}{ and~(#2#1#3)}{, (#2#1#3)}{ and~(#2#1#3)}
%% https://tex.stackexchange.com/a/121055/9019


\newcommand\numberthis{\addtocounter{equation}{1}\tag{\theequation}}

\newlist{assumplist}{enumerate}{1}
\setlist[assumplist]{label=(\textbf{\Alph*})}
\Crefname{assumplisti}{Assumption}{Assumptions}

\newlist{assumplist2}{enumerate}{1}
\setlist[assumplist2]{label=(\textbf{\Roman*})}
\Crefname{assumplist2i}{Assumption}{Assumptions}

\newlist{proplist}{enumerate}{1}
\setlist[proplist]{label=({\roman*})}
\Crefname{proplisti}{Property}{Properties}



\title{Maximum Mean Discrepancy Gradient Flow}

\begin{document}
\maketitle


\begin{abstract}
We study the Wassertein gradient flow of the Maximum Mean Discrepancy (MMD). MMD can be seen as an energy functional %We derive algorithms that converge to the target distribution and verify numerically their performance.
\end{abstract}




\section{Introduction}

%OLD INTRO ON LANGEVIN MONTE CARLO
%This paper deals with the problem of sampling from a probability measure $\mu$ on $(\R^d,\mathcal{B}(\R^d))$ which admits a density, still denoted by $\mu$, with respect to the Lebesgue measure.
%This problem appears in machine learning, Bayesian inference, computational physics... Classical methods to tackle this issue are Markov Chain Monte Carlo methods, for instance Metropolis-Hastings algorithm, Gibbs sampling. The main drawback of these methods is that one needs to choose an appropriate proposal distribution, which is not trivial. Consequently, other algorithms based on continuous dynamics have been proposed, such as the over-damped Langevin diffusion:
%\begin{equation}\label{eq:langevin_diffusion}
%dX_t= -\nabla \log \mu (X_t)dt+\sqrt{2}dB_t
%\end{equation}
%where $(B_t)_{t\ge0}$ is a $d$-dimensional Brownian motion. The Langevin Monte-Carlo (LMC) algorithm, or Unadjusted Langevin algorithm (ULA) considers the Markov chain $(X_k)_{k\ge1 }$ given by the Euler-Maruyama discretization of the diffusion \eqref{eq:langevin_diffusion}:
%\begin{equation}\label{eq:langevin_algorithm}
%X_{k+1} = X_k - \gamma_{k+1}\nabla \log \mu(X_k) + \sqrt{2\gamma_{k+1}G_{k+1}}
%\end{equation}
%where $(\gamma_k)_{k\ge1}$ is a sequence of step sizes (constant or convergent to zero), and
%$(G_k)_{k \ge 1}$ is a sequence of i.i.d. standard $d$-dimensional Gaussian random variables. This algorithm has attracted a lot of attention... But....\aknote{say something about the requirement of the knowledge of gradient of log target and how it is difficult to estimate?}

%Neural networks with a large number of parameters Theoretical explanation of 
%gradient descent is a solution of a PDE that converges to a globally optimal solution for networks with a single hidden layer under appropriate assumptions. \cite{rotskoff2019global} propose a Birth-Death dynamics that leads to a modified PDE with the same
%minimizer.


%Recently, using mathematical tools from optimal transport theory and interacting particle systems, it was shown that gradient descent [RVE18, MMN18, SS18, CB18b] and stochastic gradient descent converge asymptotically to the target function in the large data limit


Optimal transport theory provides a
powerful conceptual and mathematical framework for gradient flows on the space of distributions, and has thus found numerous applications in statistics and machine learning. A seminal work is surely the one of \cite{jordan1998variational}, who revealed that the Fokker–Planck equation is a gradient
flow equation for a the relative entropy functional (also known as the KL-divergence) with respect to the Wassertein metric. This consideration has spawned a range of sampling
algorithms, based on time discretizations of the gradient flow equation. Indeed, under appropriate conditions on the coefficient of the equation, it can be shown that its stationary distribution is unique and is the target distribution $\pi$ (see for instance \cite{pavliotis2011stochastic}, Chapter 4). In particular, the Unadjusted
Langevin Algorithm (ULA) and its Metropolis adjusted counterpart MALA have received much
attention \cite{durmus2018analysis,bernton2018langevin}). Recently, gradient flows and interacting particle systems were also used to explain the convergence of gradient descent to a globally optimal solution for networks with a single hidden layer under appropriate assumptions (see \cite{chizat2018global}). Indeed, gradient descent optimization dynamics is described by a partial differential equation (PDE), corresponding to a Wasserstein gradient flow of a convex energy functional. In this setting, the parameters of the neural networks can be seen as particles propagating in the direction of this gradient flow. \cite{rotskoff2019global} propose a Birth-Death dynamics that leads to a modified PDE which is shown to accelerate this convergence.


In this paper, we study the gradient flow equation in the Wasserstein space of order 2 associated with another free energy functional, the Maximum Mean Discrepancy (MMD), introduced in \cite{gretton2012kernel}. \aknote{to complete. state our contribution.}
%To the best of our knowledge, this work is the first one which investigates the flow of a discrepancy between distributions different from the Kullback-Leibler divergence. 
\asnote{We will probably reviewed by Mroueh so we need to compare their results to ours}

This paper is organized as follows. In \cref{sec:preliminaries}, definitions and mathematical background needed in the paper are introduced, and \cref{sec:mmd_flow} is devoted to deriving the MMD flow and reviewing its theoretical properties. 
\cref{sec:discretized_flow} investigates at length the discretized versions (in time and space) of this flow. 


\section{Gradient flow of the MMD in $W_2$}\label{sec:gradient_flow}
\subsection{Construction of the gradient flow}
In this section we introduce the gradient flow of the Maximum Mean Discrepancy (MMD) and highlight some of its properties. We start by briefly reviewing the MMD introduced in \cite{gretton2012kernel}, more details are provided in \cref{sec:rkhs}. In all the paper, $\X\subset\R^d$ is the closure of a convex open set.
\paragraph{Maximum Mean Discrepancy.}\label{subsec:MMD}
Given a characteristic kernel $k$ defined on $\X$, we denote by $\kH$ its corresponding RKHS (see \cite{smola1998learning}). The space $\kH$ is a Hilbert space with inner product $\langle .,. \rangle_{\kH}$ and corresponding norm $\Vert . \Vert_{\kH}$.
When the kernel has at most a quadratic growth, it is possible to define a distance on $\mathcal{P}_2(\X)$ called the Maximum Mean Discrepancy. It is given as the RKHS norm of the \textit{witness function} $f_{\mu,\nu}$ between $\mu$ and $\nu$ which is by definition the difference between the mean embeddings of $\nu$ and $\mu$: 
\begin{align}\label{eq:witness_function}
MMD(\mu,\nu) = \Vert f_{\mu,\nu} \Vert_{\kH}, \qquad f_{\nu,\mu}(z) = \int k(x,z)\diff \nu(x) - \int k(x,z)\diff \mu(x)  \qquad z\in \X
\end{align}
Throughout all the paper, $\mu$ will be fixed and $\nu$ could vary, hence, we will only consider the dependence in $\nu$ and denote by $\F(\nu)= \frac{1}{2}MMD^2(\mu,\nu)$. 
A direct computation shows that for any finite signed measure $\chi$ the following holds:
\begin{align}\label{prop:differential_mmd}
		\lim_{\epsilon\rightarrow 0} \epsilon^{-1}(\F(\nu +\epsilon\chi) - \F(\nu)) = \int f_{\mu,\nu}(x)d\chi(x).
	\end{align}
which implies that the differential of $\F(\nu)$ is given by $f_{\mu,\nu}$. This property will be key to construct a gradient flow of the MMD in $W_2$ as we will see in \cref{paragraph:flow_MMD}.
 Interestingly, $\F(\nu)$ can be written by means of a confinement potential $V$, an interaction potential $W$ and a constant off-set $C$:
\begin{align}\label{eq:potentials}
V(x)=-\int  k(x,x')\mu(x'), \qquad
W(x,x')=k(x,x'), \qquad
C = \frac{1}{2} \int k(x,x')\diff\mu(x) \diff\mu(x') 
\end{align}
This leads to a \textit{free-energy} expression for $\mathcal{F}(\nu)$:
\begin{align}\label{eq:mmd_as_free_energy}
	\F(\nu) = \int V(x) \diff \nu(x) +\frac{1}{2} \int W(x,y)\diff \nu(x)\diff \nu(y) + C.
\end{align}
If $X$ and $Y$ are two samples or particles following $\nu$, then $V$ reflects the potential generated by $\mu$ and acting on each particle $X$ and $Y$, while $W$ would reflect the potential arising from the interactions between the two particles $X$ and $Y$. This formulation will be useful to define a gradient flow of $\F(\nu)$. % in \cref{paragraph:flow_MMD}. 
%When samples from both $\mu$ and $\nu$ are available \cref{eq:closed_form_MMD} can be estimated using those samples.
%by finding a function $g$ in the unit ball $\mathcal{B} = \{ g\in \kH : \quad \Vert g\Vert_{\kH}\leq 1 \}$ that maximizes the mean difference between two given distributions $\mu$ and $\nu$. Such a distance is called the Maximum Mean Discrepancy  (MMD) \cite{Gretton:2012}:
%\begin{align}\label{eq:MMD}
%MMD(\mu,\nu) = \sup_{g\in \mathcal{B}} \int g\diff\mu - \int g \diff\nu
%\end{align}
%The optimal function  $g^*$ in $\mathcal{B}$ is then proportional to the  witness function between $\nu$
% and $\mu$ (see  \cite{gretton2012kernel}):
%\begin{align}\label{eq:witness_function}
%f_{\nu,\mu}(z) = \int k(.,z)\diff \mu - \int k(.,z)\diff \nu  \qquad z\in \X
%\end{align}
%This allows to provide a closed form expression for the $MMD$ in terms of expectations of the kernel under $\mu$ and $\nu$, or as the RKHS norm of \cref{eq:witness_function}:
%\begin{align}\label{eq:loss_functional}
%\F(\nu) = \frac{1}{2} MMD^2(\mu,\nu) \qquad \forall \nu \in \mathcal{P}_2(\X).
%\end{align}
\paragraph{Gradient flow of the MMD.}\label{paragraph:flow_MMD}
We consider now the problem of transporting mass from an initial distribution $\nu_0$ to a target distribution $\mu$ by finding a continuous path $\nu_t$ starting from $\nu_0$ that would ideally converge to $\mu$ while decreasing $\F(\nu_t)$. Such path should be physically plausible in that  teleportation phenomena are not allowed. For instance, the path $\nu_t = (1-e^{-t})\mu + e^{-t}\nu_0$ would constantly teleport mass between $\mu$ and $\nu_0$ although it decreases  $\F$ since $\F(\nu_t)=e^{-2t}\F(\nu_0)$ (see Corollary 1 in \cite{mroueh2018regularized}). The physicality of the path is better understood in terms of classical statistical physics: if $\nu_0$ is an initial configuration of $N$ particles, these can be moved towards a different configuration $\mu$ through successive small transformations without jumping from a location to another. There are practical situations in machine learning where this constraint will arise naturally as we will see in \cref{subsec:training_neural_networks}. 

The optimal transport theory provides a way to construct such a continuous path by means of the \textit{continuity equation}. Given a vector field $V_t$ on $\X$ and an initial condition $\nu_0$, the continuity equation is a partial differential equation which defines a path $\nu_t$ evolving under the action of the vector field $V_t$, and reads $\partial_t \nu_t = -div(\nu_t V_t)$ for all $t\geq0$.
This equation expresses two facts, the first one is that $-div(\nu_t V_t)$ reflects the infinitesimal changes in $\nu_t$ as dictated by the vector field (also referred to as velocity field) $V_t$, the second fact is that the total mass of $\nu_t$ does not vary in time as a consequence of the divergence theorem. The reader can find more detailed discussions in \cite{Santambrogio:2015}. Following  \cite{ambrosio2008gradient}, a natural choice is to choose $V_t$ as the negative gradient of the differential of $\F(\nu_t)$ at $\nu_t$, since it corresponds to a gradient flow of $\F$ associated with the 2-Wasserstein metric (see the definition \cref{subsec:gradient_flows_functionals}). %The choice of $V_t$ determines the path $\nu_t$, therefore there must be a direct link between $V_t$ and $\F(\nu_t)$ if one wants to construct a gradient flow for of $\F$. 
%Following  \cite{ambrosio2008gradient}, such link is obtained by choosing $V_t$ to the negative gradient of the differential of $\F(\nu_t)$ at $\nu_t$.
By \cref{prop:differential_mmd}, we know that the differential of $\F(\nu_t)$  at $\nu_t$ is given by $f_{\mu,\nu_t}$, hence $V_t(x) = -\nabla f_{\mu,\nu_t}(x)$. The
gradient flow of $\F$ is then defined by the solution $(\nu_t)_{t\geq 0}$ of:
\begin{align}\label{eq:continuity_mmd}
	\partial_t \nu_t = div(\nu_t \nabla f_{\mu,\nu_t}).
\end{align}
\cref{eq:continuity_mmd} is peculiar in that the vector field depends itself on $\nu_t$. \aknote{state existence of a solution and talk about uniqueness}  This type of equation is associated in the probability theory literature to the so-called McKean-Vlasov process \cite{kac1956foundations,mckean1966class}:
\begin{align}\label{eq:mcKean_Vlasov_process}
	d X_t = -\nabla f_{\mu,\nu_t}(X_t)dt \qquad X_0\sim \nu_0.
\end{align}

In fact,  \cref{eq:mcKean_Vlasov_process} defines a process $(X_t)_{t\geq 0}$ whose distribution $(\nu_t)_{t\geq 0}$ satisfies \cref{eq:continuity_mmd} as shown in \cref{prop:existence_uniqueness}. 
$(X_t)_{t\geq 0}$ can be interpreted as  the trajectory of a single particle starting from an initial random position $X_0$ drawn from $\nu_0$. Its trajectory is then driven by a velocity field $-\nabla f_{\mu,\nu_t}$. However, such particle interacts with other particles driven by the same velocity field, which affects its trajectory. This interaction is captured by the velocity field through the dependence on the current configuration of all particles $\nu_t$.
%Existence and uniqueness of a solution to \cref{eq:continuity_mmd,eq:mcKean_Vlasov_process} is guaranteed under mild conditions on the kernel $k$ and are provided in \cref{thm:existence_uniqueness}:
%\begin{theorem}\label{thm:existence_uniqueness}(Existence and uniqueness)
%	Under \manote{some assumptions}, and given $\nu_0\in \mathcal{P}_2(\X)$ there exists a unique process $(\X_t)_{t\geq 0}$ with $X_0\sim \nu_0$ and satisfying the McKean-Vlasov equation in \cref{eq:mcKean_Vlasov_process}. Moreover, the distribution $\nu_t$ of $X_t$ is the unique solution of \cref{eq:continuity_mmd} in a weak sense and hence defines a gradient flow of $\F$. 
%\end{theorem}
%A proof of \cref{thm:existence_uniqueness} is provided in \manote{proof} and relies on standard existence and uniqueness results of McKean-Vlasov processes under regularity of the map $(x,\nu)\mapsto \nabla f_{\mu,\nu}(x)$ \cite{Jourdain:2007}. Such regularity is ensured if the kernel $k$ has Lipschitz gradients for instance. Besides, the existence and uniqueness of the process itself, one would like the functional $\F$ to decrease along the path $\nu_t$ and ideally to converge towards $0$. While the latter is hard to obtain and will be discussed in \cref{sec:Lojasiewicz_inequality}, the first property is rather easy to get and is the object of \cref{prop:decay_mmd}:
Existence and uniqueness of a solution to \cref{eq:continuity_mmd,eq:mcKean_Vlasov_process} is guaranteed under mild conditions on the kernel $k$ using \cite[Proposition 2.5]{chizat2018global} (see \cref{proof:prop:existence_uniqueness} for a full proof):
\begin{proposition}\label{prop:existence_uniqueness}(Existence and uniqueness)
	Under Lipschitzness of $\nabla k$, and given $\nu_0\in \mathcal{P}_2(\X)$ there exists a unique process $(\X_t)_{t\geq 0}$ with $X_0\sim \nu_0$ and satisfying the McKean-Vlasov equation in \cref{eq:mcKean_Vlasov_process}. Moreover, the distribution $\nu_t$ of $X_t$ is the unique solution of \cref{eq:continuity_mmd} and defines a gradient flow of $\F$. 
\end{proposition}
Besides existence and uniqueness of the gradient flow of $\F$, one expects $\F$ to decrease along the path $\nu_t$ and ideally to converge towards $0$. The first property is rather easy to get and is the object of \cref{prop:decay_mmd} with a proof differed to \cref{proof:prop:decay_mmd}: 
\begin{proposition}\label{prop:decay_mmd}
	Under the assumptions of \cref{prop:existence_uniqueness}, $\F(\nu_t)$ is decreasing in time and satisfies:
	\begin{align}\label{eq:time_evolution_mmd}
		\frac{d\F(\nu_t)}{dt}= - \int \Vert \nabla f_{\mu,\nu_t}(x) \Vert^2 \diff \nu_t(x)  
	\end{align}
\end{proposition}
Convergence of the flow is harder to obtain and will be discussed in \cref{sec:Lojasiewicz_inequality}.
From \cref{eq:time_evolution_mmd}, $\F$ can be seen as a Lyapounov functional for the dynamics defined by \cref{eq:continuity_mmd}. This property will be crucial in the analysis of \cref{sec:Lojasiewicz_inequality} and ensures that $\F(\nu_t)$  converges to some non-negative limit $l$ since it is a lower-bounded and non-increasing function of time. The reader may refer to \cref{subsec:kl_flow} for a connection to other well-known gradient flows in Wassertein space. We know motivate the study of this flow by recent considerations and applications in Machine Learning.

%We conclude this section with a remark referring to other well-known gradient flows in Wasserstein space.
%\begin{remark}\label{remark:gradient_flow}
%A different choice for $\F$  leads to a different continuity equation. A celebrated example is the case where $\F$ is the KL divergence between the target distribution $\mu$ and a current distribution $\nu_t$: $KL(\nu_t\Vert \mu ) =  \int \log (\nu_t(x))\diff \nu_t(x) -\int \log(\mu(x))\diff \nu_t(x)$. In \cite{jordan1998variational}, it was shown, under mild conditions on $\mu$ and $\nu_0$, that the gradient flow associated to the $KL$ leads to the so-called Fokker-Planck equation: $ \partial_t \nu_t = - div(\nu_t \nabla \log(\mu(x))) + \Delta \nu_t $ with Langevin diffusion as its corresponding process: $d X_t = \nabla \log(\mu(x))\diff t + \sqrt{2} \diff W_t $ with $X_0\sim \nu_0$ and $W_t$ is a Brownian motion.
 
 %While the entropy term in the $KL$ functional prevents the particle from "crashing" onto the mode of $\mu$, this role could be played by the interaction energy $W$ for $MMD^2$ defined in \cref{eq:potentials}. Indeed, consider for instance the gaussian kernel $k(x,x')=e^{-\|x-x'\|^2}$. It is convex thus attractive at long distances ($\|x-x'\|>1$) but not at small distances (so repulsive).\aknote{not very precise yet but i think the interpretation could be interesting}

%The solution to the Fokker-Planck equation describing the gradient flow of the $KL$ can be shown to converge towards $\mu$ under mild assumptions. This follows from the displacement convexity of the $KL$ along the Wasserstein geodesics \manote{ref}. Unfortunately the $MMD^2$ is not displacement convex in general as we will see in \manote{ref}. This makes the task of proving the convergence of the gradient flow of the $MMD^2$ to the global optimum $\mu$ much harder. In fact, when the external potential $V$ in \cref{eq:potentials} is not convex, which is the case for instance when $k$ is a gaussian kernel, it was shown that the diffusion may admit several local minima (see \cite{herrmann2010non,tugaut2014phase}) \manote{make sure that these conditions are relevant for us}. Moreover, we show in \cref{sec:Lojasiewicz_inequality} that local minima which are not global exist and that it is rather easy to reach them.\aknote{hm wrong section?}
%\end{remark}
 






\subsection{Training neural networks as flow of the MMD}\label{subsec:training_neural_networks}
In this sub-section we establish a formal connection between the MMD gradient flow defined in \cref{eq:continuity_mmd} and neural networks optimization in the limit of infinitely many neurons.
The MMD was successfully used to train generative networks \cite{Arbel:2018,Binkowski:2018}\manote{cite others}, where it is used as a loss functional between the data distribution and a parametric model distribution.
Here we consider a different problem which arises naturally from optimizing neural networks using gradient descent. We show that this problem can be lifted to a non-parametric problem with the MMD as a cost function in a similar way as in \cite{Rotskoff:2019}. To remain consistent with the rest of the paper, the parameters of a network will be denoted by $x\in \X$ while the input and outputs will be denoted as $z$ and $y$.
 Given a neural network or any parametric function $(z,x)\mapsto \psi(z,x)$ with parameter $x \in \X $ and input data $z$ we consider the supervised learning problem:
\begin{align}\label{eq:regression_network}
	\min_{(x_1,...,x_m )\in \X} \frac{1}{2}\mathbb{E}_{(y,z)\sim \mathbb{P}  } \left[ \Big\Vert y - \frac{1}{m}\sum_{i=1}^m\psi(z,x_i) \Big\Vert^2 \right ]
\end{align}
where $(y,z)$ are samples from the data distribution and the regression function is an average network of $m$ different networks. The formulation in \cref{eq:regression_network} includes any type of networks. Indeed, the averaged function can itself be seen as one network with augmented parameters $(x_1,...,x_m)$ and any network can be written as an average of sub-networks with potentially shared weights. In the limit $m\rightarrow \infty$, the average can be seen as an expectation over the parameters under some probability distribution $\nu$. This leads to an expected network $\Psi(z,\nu) =  \int \psi(z,x) \diff \nu(x) $ and the optimization problem in \cref{eq:regression_network} can be lifted to an optimization problem in $\mathcal{P}_2(\X)$ the space of probability distributions:
\begin{align}\label{eq:lifted_regression}
	\min_{\nu \in \mathcal{P}_2(\X)}  \mathcal{L}(\nu) :=  \mathbb{E}_{(y,u)\sim \mathbb{P} } \left [ \big\Vert y - \int \psi(z,x) \diff \nu(x) \big\Vert^2 \right ]
\end{align} 
For convenience, we consider $\bar{\mathcal{L}}(\nu)$ the function obtained by subtracting the variance of $y$ from $\mathcal{L}(\nu)$, i.e.: $\bar{\mathcal{L}}(\nu) = \mathcal{L}(\nu) - var(y) $. When the model is well specified, there exists $\mu \in \mathcal{P}_2(\X) $ such that $\mathbb{E}_{y\sim \mathbb{P}(.|z)}[y] =  \int \psi(z,x) \diff \mu(x)$. In that case, the cost function $\bar{\mathcal{L}}$ matches  the functional $\F$ defined in \cref{eq:mmd_as_free_energy}  for a particular choice of the kernel $k$. More generally, as soon as a global minimizer for  \cref{eq:lifted_regression} exists,  \cref{prop:inequality_mmd_loss} relates the two losses $\bar{\mathcal{L}}$ and $\mathcal{F}$:
\begin{proposition}\label{prop:inequality_mmd_loss}
	Assuming a global minimizer of \cref{eq:lifted_regression} is achieved by some $\mu\in \mathcal{P}_2(\X)$, the following inequality holds for any $\nu \in \mathcal{P}_2(\X)$:
	\begin{align}\label{eq:inequality_mmd_nn}
		\left(\bar{\mathcal{L}}(\mu)^{\frac{1}{2}} + \F^{\frac{1}{2}}(\nu)\right)^2
		\geq 
		\bar{\mathcal{L}}(\nu)
		\geq
		\mathcal{F}(\nu) + \bar{\mathcal{L}}(\mu)
	\end{align}
	where $\F(\nu)$ is defined by \cref{eq:mmd_as_free_energy} with  a kernel $k$  constructed from the data as an expected product of networks:
\begin{align}
	k(x,x') = \mathbb{E}_{z\sim \mathbb{P}} [\psi(z,x)^T\psi(z,x')]
\end{align}
Moreover, $\bar{\mathcal{L}} = \F$ iif $\bar{\mathcal{L}}(\mu)=0$, which means that the model is well-specified. 
\end{proposition}
\cref{prop:inequality_mmd_loss} is shown in \cref{proof:prop:inequality_mmd_loss} and is a consequence of the optimality of $\mu$ after rearranging the terms obtained by expanding $\bar{\mathcal{L}}(\nu)$.
The framing \cref{eq:inequality_mmd_nn} implies that optimizing $\mathcal{F}$ can decrease  $\bar{\mathcal{L}}$ and vice-versa. However, the two functionals do not generally share the same local minima although they share the same global optima in general.\aknote{not clear to me..} One interesting class of problems where \cref{eq:lifted_regression} corresponds exactly to minimizing the $MMD$ is the student-teacher problem or the problem of distilling a pre-trained network into another network with the same architecture (see \cite{rotskoff2019global})\manote{some references}. In this case the gradient flow of the MMD defined in \cref{eq:continuity_mmd} corresponds to the population limit of the usual gradient flow of \cref{eq:regression_network} when the final layer becomes infinitely wide\aknote{same, globally we should rewrite this paragraph}. Indeed, solving \cref{eq:regression_network} is usually done using gradient descent. When the step-size approaches $0$, the parameters $(x_1,...,x_m)$ satisfy the continuous-time system of equations:
\begin{align}\label{eq:particle_dynamics}
	\dot{x}_i(t)= -\nabla \mathcal{L}(x_1(t),...,x_m(t)) \text{ for } i=1, \dots, m
\end{align}  
As pointed out in \cite{chizat2018global,Rotskoff:2019}, the dynamics in \cref{eq:particle_dynamics} can be analyzed in the "mean-field" limit when $m\rightarrow \infty$. For \cref{eq:particle_dynamics}, this leads to the continuity equation \cref{eq:continuity_mmd}. A formal statement that relates the finite population dynamics to the mean-field limit is provided in \manote{statement here}. 
%\begin{align}\label{eq:mmd_flow}
%	\partial_t \nu_t = div(\nu_t \nabla f_{\mu,\nu_t})
%\end{align}
%for an initial choice of distribution $\nu_0$. In \cref{eq:mmd_flow}, $f_{\mu,\nu_t}$ is the witness function between $\mu$ and $\nu_t$:% defined and is given by:
%\[
%f_{\mu,\nu_t}(\theta) =  \int k(\theta,\theta')(\diff \nu_t(\theta')-\diff\mu(\theta') ) 
%\]
%\cref{eq:mmd_flow} can be shown to have a unique solution and has a particle version which is described by:
%\begin{align}\label{eq:particle_equation}
%	\dot{X}_t = -\nabla f_{\mu,\nu_t}(X_t) \qquad X_0\sim \nu_0
%\end{align}
%In \manote{section} we describe the properties of this flow and explain why in general it might not reach the global optimum.


%
%\subsection{Background on optimal transport}

In this section we recall how to endow the space of probability measures $\mathcal{P}(\X)$ on $\X$ a compact, convex subset of $\R^d$ with a distance (e.g, optimal transport distances), and then deal with gradient flows of suitable functionals on such a metric space. The reader may refer to %\cite{santambrogio2017euclidean} for a clear review on the subject. For a given distributions $\nu\in\mathcal{P}(\X)$ and an integrable function $f$ under $\nu$, the expectation of $f$ under $\nu$ will be written either as $\nu(f)$ or $\int f \diff\nu$ depending on the context. 

\subsubsection{$2$-Wasserstein geometry}\label{subsec:wasserstein_flow}

Let $T: \X \rightarrow \X$ be a measurable map, and $\rho \in \mathcal{P}(\X)$. The push-forward measure $T_{\#}\rho$
is characterized by:
\begin{align}
%	&\quad T_{\#}\rho(A) = \rho(T^{-1}(A)) \text{ for every measurable set A,}\\
%\text{or}&
 \int_{y \in \X} \phi(y) d(T_{\#}\rho)(y) =\int_{x \in \X}\phi(T(x)) d\rho(x) \text{ for every measurable function $\phi$.}
\end{align}
Let $\mathcal{P}_2(\X)$ the set of probability distributions on $\X$ with finite second moment. For two given probability distributions $\nu$ and $\mu$ in $\mathcal{P}_2(\X)$ we denote by $\Pi(\nu,\mu)$ the set of possible couplings between $\nu$ and $\mu$. In other words $\Pi(\nu,\mu)$ contains all possible distributions $\pi$ on $\X\times \X$ such that if $(X,Y) \sim \pi $ then $X \sim \nu $ and $Y\sim \mu$. The $2$-Wasserstein distance on $\mathcal{P}_2(\X)$ is defined by means of optimal coupling between $\nu$ and $\mu$ in the following way:
\begin{align}\label{eq:wasserstein_2}
W_2^2(\nu,\mu) := \inf_{\pi\in\Pi(\nu,\mu)} \int \Vert x - y\Vert^2 d\pi(x,y) \qquad \forall \nu, \mu\in \mathcal{P}_2(\X)
\end{align}
It is a well established fact that such optimal coupling $\pi^*$ exists. Moreover, it can be used to define a path $(\rho_t)_{t\in [0,1]}$ between $\nu$ and $\mu$ in $\mathcal{P}_2(\X)$. \aknote{Maybe we can defer the def of pushforward measures and equation of $s_t$ to the Appendix?}For a given time $t$ in $[0,1]$ and given a sample $(x,y)$ from $\pi^{*}$, it possible to construct a sample $z_t$ from $\rho_t$ by taking the convex combination of $x$ and $y$: $z_t = s_t(x,y)$ where $s_t$ is given by \cref{eq:convex_combination}
\begin{equation}\label{eq:convex_combination}
s_t(x,y) = (1-t)x+ty \qquad \forall x,y\in \X, \; \forall t\in [0,1].
\end{equation}
The function $s_t$ is well defined since $\X$ is a convex set. More formally, $\rho_t$ can be written as the projection or push-forward of the optimal coupling $\pi^{*}$ by $s_t$:  
\begin{equation}\label{eq:displacement_geodesic}
\rho_t = (s_t)_{\#}\pi^{*}
\end{equation}
It is easy to see that \cref{eq:displacement_geodesic} satisfies the following boundary conditions:
\begin{align}\label{eq:boundary_conditions}
\rho_0 = \nu \qquad \rho_1 = \mu.
\end{align}
Paths of the form of \cref{eq:displacement_geodesic} are called \textit{displacement geodesics}. They can be seen as the shortest paths from $\nu$ to $\mu$ in terms of mass transport (\cite{Santambrogio:2015} Theorem 5.27). It can be shown that there exists a \textit{velocity vector field} $(t,x)\mapsto v_t(x)$ with values in $\R^d$ such that $\rho_t$ satisfies the continuity equation:
\begin{equation}\label{eq:continuity_equation}
\partial_t \rho_t + div(\rho_t v_t ) = 0 \qquad \forall t\in[0,1].
\end{equation}
Equation \cref{eq:continuity_equation} is well defined in distribution sense even when $\rho_t$ doesn't have a density. $v_t$ can be interpreted as a tangent vector to the curve $(\rho_t)_{t\in[0,1]}$ at time $t$ so that the length $l(\rho_t)$ of the curve $\rho_t$ would be given by:
\begin{equation}
l(\rho)^2 = \int_0^1 \Vert v_t \Vert^2_{L_2(\rho_t)} \diff t \quad \text{ where } \quad 
\Vert v_t \Vert^2_{L_2(\rho_t)} =  \int \Vert v_t(x) \Vert^2 \diff \rho_t(x)
\end{equation}
%\aknote{add constant speed geodesics}
This perspective allows to provide a dynamical interpretation of the $W_2$ as the length  of the shortest path from $\nu$ to $\mu$ and is summarized by the celebrated Benamou-Brenier formula (\cite{Santambrogio:2015}, Theorem\aknote{check} 5.28):
\begin{align}\label{eq:benamou-brenier-formula}
W_2(\nu,\mu) = \inf_{(\rho,v)} l(\rho)
\end{align}
where the infimum is taken  over all couples  $\rho$ and $v$ satisfying  \cref{eq:continuity_equation}  with boundary conditions given by \cref{eq:boundary_conditions}.

\begin{remark}
	Such paths should not be confused with another kind of paths called \textit{mixture geodesics}. The mixture geodesic $(m_t)_{t\in[0,1]}$ from $\nu$ to $\mu$ is obtained by first choosing either $\nu$ or $\mu$ according to a Bernoulli distribution of parameter $t$ and then sampling from the chosen distribution:
	\begin{align}\label{eq:mixture_geodesic}
	m_t = (1-t)\nu + t\mu \qquad \forall t \in [0,1].
	\end{align}
	Paths of the form \cref{eq:mixture_geodesic} can be thought as the shortest paths between two distributions when distances on $\mathcal{P}_2(\X)$ are measured using the $MMD$ (\cite{Bottou:2017} Theorem 5.3). We refer to \cite{Bottou:2017} for an overview of the notion of shortest paths in probability spaces and for the differences between mixture geodesics and displacement geodesics.
	Although, we will be interested in the $MMD$ as a loss function, we will not consider the geodesics that are naturally associated to it and we will rather consider the displacement geodesics defined in \cref{eq:displacement_geodesic} for reason that will become clear in \cref{subsec:wasserstein_flow}.
\end{remark}

\subsubsection{Gradient flows on the space of probability measures}\label{sec:gradient_flows_functionals}


%Let $\F : \mathcal{P}(\X) \rightarrow \R \cup \infty$, $\rho \mapsto \F(\rho)$ a functional. %We call $\frac{\partial{\F}}{\partial{\rho}}$ if it exists, the unique (up to additive constants) function such that $\frac{d}{d\epsilon}\F(\rho+\epsilon  f)_{\epsilon=0}=\int\frac{\partial{\F}}{\partial{\rho}}(\rho) df$ for every perturbation $f$ such that, at least for $\epsilon \in [0, \epsilon_0]$, the measure $\rho +\epsilon f$ belongs to $\mathcal{P}(\X)$. The function $\frac{\partial{\F}}{\partial{\rho}}$ is called first variation of the functional $\F$ at $\rho$. 
Consider a 
\textit{Lyapunov functional} 
(or "free energy") $\F$ over the space of probability measures $\mathcal{P}(\X)$
(see \cite{Villani:2004}), 
i.e. a functional of the form:
\begin{equation}\label{eq:lyapunov}
\F(\rho)=\int U(\rho(x)) \rho(x)dx + \int V(x)\rho(x)dx + \int W(x,y)\rho(x)\rho(y)dxdy
\end{equation}
where  $U$ is the internal energy, $V$ the potential (or confinement) energy and $W$ the
interaction energy. The formal gradient flow equation associated to this functional can be written:
\begin{equation}\label{eq:continuity_equation1}
\frac{\partial \rho}{\partial t}= div( \rho \nabla \frac{\partial \F}{\partial \rho})=div( \rho \nabla (U'(\rho) + V + W * \rho))
\end{equation}
where $\nabla \frac{\partial \F}{\partial \rho}$ is the strong subdifferential of $\F$ associated with the 2-Wasserstein
metric (see \cite{ambrosio2008gradient}, Lemma 10.4.1). Indeed, for some generalized notion of gradient $\nabla_{W_2}$, and for sufficiently regular $\rho$ and $\F$, the r.h.s. of \eqref{eq:continuity_equation1} corresponds to $-\nabla_{W_2}\F(\rho)$.
The dissipation of entropy is defined as\aknote{add ref villani again}: 
\begin{align}
       \frac{d \F(\rho)}{dt} =-D(\rho) \quad \text{ with } D(\rho)= \int |\nabla \frac{\partial \F}{\partial \rho}|^2 \rho(x)dx
%&\text{ and } \xi= \nabla \frac{\partial \F}{\partial \rho} = \nabla (U'(\rho) + V + W * \rho)
\end{align}
Standard considerations from fluid mechanics tell us that the continuity equation \eqref{eq:continuity_equation1} may be interpreted as the equation ruling the evolution of the density $\rho_t$ of a family of particles initially distributed according to some $\rho_0$, and each particle follows the velocity vector field $v_t=\nabla \frac{\partial{\F}}{\partial{\rho_t}}(\rho_t)$.

\begin{remark} \label{rem:KL_Lyapunov}\aknote{define in the appendix div, laplacian...}
	A famous example of a free energy \eqref{eq:lyapunov} is the Kullback-Leibler divergence, defined for $\rho, \mu \in \mathcal{P}(\X)$ by
	$KL(\rho,\mu)=\int log(\frac{\rho(x)}{\mu(x)})\rho(x)dx$. Indeed, $KL(\rho, \mu)=\int U(\rho(x))dx + \int V(x) \rho(x)dx$ with $U(s)=s\log(s)$ the entropy function and $V(x)=-log(\mu(x))$. In this case, $\nabla \frac{\partial \F}{\partial \rho}= \nabla \log(\rho) + \nabla V=  \nabla \log(\frac{\rho}{\mu})$ and equation \eqref{eq:continuity_equation1} leads to the classical Fokker-Planck equation:
	\begin{equation}\label{eq:Fokker-Planck}
	\frac{\partial{\rho}}{\partial t}= div(\rho \nabla V )+ \Delta \rho
	\end{equation}
It is well-known (see for instance \cite{jordan1998variational}) that the distribution of the Langevin diffusion:
	\begin{equation}\label{eq:langevin_diffusion}
	dX_t= -\nabla \log \mu (X_t)dt+\sqrt{2}dB_t
	\end{equation}
	where $(B_t)_{t\ge0}$ is a $d$-dimensional Brownian motion, satisfies \eqref{eq:Fokker-Planck}.
\end{remark}


The next section describes the dynamics of the gradient flow of \cref{eq:closed_form_MMD} under the $2$-Wasserstein metric as defined in \cref{subsec:wasserstein_flow}.
%The MMD was successfully used for training generative models (\cite{mmd-gan,Binkowski:2018,Arbel:2018}) where it is used in a loss functional to learn the parameters of the generator network. This motivate the  


\subsection{Convexity w.r.t. 2-Wassertein geodesics}\label{subsec:more_lambda_convexity}



\begin{definition}\label{def:displacement_convexity}[Displacement convexity]. Let $\mu$
	and $\nu$ in $\mathcal{P}(\X)$. There exists a $\mu-a.e.$
	unique gradient of a convex function, denoted by $\nabla\phi$, such that $\mu$
	is equal to $\nabla\phi_{\#}\nu$ and one can define the displacement geodesic $\nu_{t}=((1-t)Id+t\nabla\phi)_{\#}\nu$
	for $0\leq t\leq1$. We say that a functional $\nu\mapsto\mathcal{F}(\nu)$
	is displacement convex if 
	\begin{equation}
	t\mapsto\mathcal{F}(\nu_{t})
	\end{equation}
	is convex for any $\nu$ and $\mu$. %Moreover, we say that $\mathcal{F}$ is displacement convex in a neighborhood of $\mu$ if there exists a radius $r>0$ such that the above property holds for any $\nu$ with $W_{2}(\mu,\nu)\leq r$.
\end{definition}

\begin{definition}\label{def:lambda-convexity}[$\Lambda$ convexity]
	We say that a functional $\nu\mapsto\mathcal{F}(\nu)$ is $\Lambda$-convex
	if for any $\nu$ and $\mu$ and a minimizing geodesic $\text{\ensuremath{\nu_{t}}}$
	between $\nu$ and $\mu$ with velocity vector field $v_{t}$, i.e:
	$\partial_{t}\nu_{t}+div(\nu_{t}v_{t})=0;\nu_{0}=\nu;\nu_{1}=\mu;$
	the following holds:
	\begin{equation}\label{eq:lambda_displacement_convex}
	\frac{d^{2}\mathcal{F}(\nu_{t})}{dt^{2}}\geq\Lambda(\nu_{t},v_{t})\qquad\forall\; t\in[0,1].
	\end{equation}
	where $(\nu,v)\mapsto\Lambda(\nu,v)$
	is a function that defines for each $\nu \in \mathcal{P}(\X)$
	a quadratic form on the set of square integrable vectors valued functions
	$v$ , i.e: $v\in L_{2}(\mathbb{R}^{d},\mathbb{R}^{d},\nu)$. We
	further assume that $\inf_{\nu,v}\Lambda(\nu,v)/\Vert v\Vert_{L_{2}(\nu)}^{2}>-\infty$. 
	Also, the following holds:
	\begin{equation}\label{eq:integral_lambda_convexity}
	\F(\nu_{t})\leq(1-t)\F(\nu_{0})+t\F(\nu_{1})-\int_{0}^{1}\Lambda(\nu_{s},v_{s})G(s,t)ds
	\end{equation}
	where $G(s,t)=s(1-t) \mathbb{I}\{s\leq t\}
	+t(1-s) \mathbb{I}\{s\geq t\}$.
\end{definition}

Then, to show the $\Lambda$-convexity of the functional defined in \cref{sec:gradient_flow} we first make the following assumptions on the kernel:
\begin{assumplist} 
	\item \label{assump:bounded_trace} $ \vert \sum_{1\leq i\leq d} \partial_i\partial_ik(x,x) \vert\leq \frac{L}{3}  $ for all $x\in \mathbb{R}^d$.
	\item \label{assump:bounded_hessian} $\Vert H_xk(x,y) \Vert_{op} \leq \frac{L}{3}$ for all $x,y\in \mathbb{R}^d$, where $H_xk(x,y)$ is the hessian of $x\mapsto k(x,y)$ and $\Vert.\Vert_{op}$ is the operator norm.
	\item \label{assump:bounded_fourth_oder} $\Vert Dk(x,y) \Vert\leq \lambda  $ for all $x,y\in \mathbb{R}$, where $Dk(x,y)$ is an $\mathbb{R}^{d^2}\times \mathbb{R}^{d^2}$ matrix with entries given by $\partial_{x_{i}}\partial_{x_{j}}\partial_{x'_{i}}\partial_{x_{j}'}k(x,y)$.
\end{assumplist}
The next proposition states that the functional defined in \cref{sec:gradient_flow} is $\Lambda$-displacement convex and provide and explicit expression for the functional $\Lambda$. 
%Some additional mild assumptions on the derivative of the kernel are also needed but deferred to the appendix for presentation purpose.\aknote{to do!!}

\begin{proposition}
	\label{prop:lambda_convexity} Suppose $\sup_{x,y} \partial_{x_{i}}\partial_{x_{j}}\partial_{x'_{i}}\partial_{x_{j}'}k(x,y)\le \lambda$ is satisfied for some $\lambda \in \R^+$. The functional $\nu\mapsto \F(\nu)$ is $\text{\ensuremath{\Lambda}}$-convex
	with $\Lambda$ given by:
	\begin{equation}
	\Lambda(\nu,v)=\langle v,(C_{\nu}-\lambda \F(\nu)^{\frac{1}{2}}I)v\rangle_{L_{2}(\nu)}\label{eq:Lambda}
	\end{equation}
	where $C_{\nu}$ is the (positive) operator defined by $(C_{\nu}v)(x)=\int\nabla_{x}\nabla_{x'}k(x,x')v(x')d\nu(x')$ for any $x \in \X$.
	%\begin{align}\label{eq:positive_operator_C}
	%(C_{\nu}v)(x)=\int\nabla_{x}\nabla_{x'}k(x,x')v(x')d\nu(x')
	%\end{align}
\end{proposition}
%
%
Consider the geodesic $\nu_{t}=((1-t)Id+t\nabla\phi)_{\#}\nu$ of \cref{def:displacement_convexity}, so that $\nu_{1}=\mu$ and at time $t=1$ and thus $\F(\nu_{1})=0$. It is worth noting that by \cref{prop:lambda_convexity}, we get that the non-negative hessian \eqref{eq:lambda_displacement_convex} at the global minimum $\mu$, is $\langle v_{t},C_{\nu_{t}}v_{t}\rangle_{L_{2}(\nu_{t})}$ which is positive. Also, we can now write the following convexity inequalities along the gradient flow of $\F$.

%
%In this section we consider the setting of learning a neural network to solve a regression problem in a similar context as in \cite{Rotskoff:2019} and show how this problem is related minimizing an MMD between an unknown target distribution and the current one. The neural network can be written as 
%
%
%
%From now on we consider $\mathcal{P}_2(\X)$ the set of probability distributions with finite second moments defined over a domain $\X\subset \R^d$.
%The Maximum Mean Discrepancy (MMD) between two distributions $\mu$ and $\nu$ in $\mathcal{P}_2(\X)$ is defined by 
% \begin{align}\label{eq:closed_form_MMD}
%MMD^2(\mu,\nu) = \int k(x,x')(\diff\mu(x) -\diff\nu(x))(\diff\mu(x') -\diff\nu(x'))
%\end{align}
%where $k$ is a positive semi-definite kernel \cite{gretton2012kernel}.  We are interested in minimizing the MMD between an unknown target distribution $\mu$ and a current proposal distribution $\nu$:
%\begin{align}
%	\nu^{*} \in \arg\min_{\nu\in \mathcal{P}_2(\X)} MMD(\mu,\nu)
%\end{align}
%When $k$ is characteristic, the MMD defines a distance over $\mathcal{P}_2(\X)$ hence the only optimal solution would be $\nu^*=\mu$.
%
%
%Given a fixed target distribution $\mu$ known only indirectly through observations and an initial distribution  $\nu_0$, we consider the problem of constructing flow of distributions $(\nu_t)_{t \geq 0}$ that would decrease $MMD(\mu,\nu_t)$ in time and eventually converge towards $\mu$. There are many ways one could construct such sequences and a trivial way would be to consider mixtures between samples from $\mu$ and samples $\nu_0$ with proportions gradually in favor of $\mu$:
%\[
%\nu_t = (1-e^{-t})\mu +e^{-t}\nu_0
%\]
%However, such choice would be effectively infeasible in many cases, especially when samples from $\mu$ are  indirectly observed as in the case of regression with neural networks \cite{Rotskoff:2019}.  
%
%
%
%\begin{align}
%	\nu^* = 
%\end{align} 
%
%
%To introduce the problem we consider the consider a setting similar to \cite{Rotskoff:2019}
%
%
%
%
%For a given kernel $k$ defined on a domain $\X\subset \R^d$, we denote by $\kH$ its corresponding Reproducing Kernel Hilbert Space \manote{some reference here needed}. The Maximum Mean Discrepancy between two distributions $\mu$
%Under mild conditions \manote{write conditions} on the kernel $k$, it is possible to define a distance on $\mathcal{P}_2(\X)$ by finding a function $f$ in $\mathcal{B}$ that maximizes the mean difference between two given distributions $\mu$ and $\nu$.  
%
%
% $\kH$ is a Hilbert space with inner product $\langle .,. \rangle_{\kH}$ and corresponding norm $\Vert . \Vert_{\kH}$. The unit ball in $\kH$ which will be denoted as $\mathcal{B}$ is simply the set of functions $f$ in $\kH$ such that $\Vert f\Vert_{\kH}\leq 1 $:
%\begin{align}\label{eq:unit_ball_RKHS}
%\mathcal{B} = \{ f\in \kH : \quad \Vert f\Vert_{\kH}\leq 1 \}
%\end{align}
% Such distance is called the Maximum Mean Discrepancy  (MMD) \cite{Gretton:2012}:
%\begin{align}\label{eq:MMD}
%MMD(\mu,\nu) = \sup_{g\in \mathcal{B}} \int g\diff\mu - \int g \diff\nu
%\end{align}
%The maximization problem in \cref{eq:MMD} is achieved for an optimal $g^*$ in $\mathcal{B}$ that is proportional to the  witness function between $\nu$ and $\mu$:
%\begin{align}\label{eq:witness_function}
%f_{\nu,\mu}(z) = \int k(.,z)\diff \mu - \int k(.,z)\diff \nu  \qquad z\in \X
%\end{align}
%This allows to express the $MMD$ as the norm of \cref{eq:witness_function} $f_{\nu,\mu}$:
%\begin{align}\label{eq:mmd_norm_witness}
%MMD(\mu,\nu) = \Vert f_{\nu,\mu} \Vert_{\mathcal{H}} 
%\end{align}
%Furthermore, a closed form expression in terms of expectations of the kernel under $\mu$ and $\nu$ can be obtained \cite{gretton2012kernel}:
%\begin{align}\label{eq:closed_form_MMD}
%MMD^2(\mu,\nu) = \int k\diff\mu \diff\mu + \int k\diff\nu \diff \nu - 2\int k\diff\mu \diff \nu
%\end{align}
%When samples from both $\mu$ and $\nu$ are available \cref{eq:closed_form_MMD} can be estimated using those samples. For a fixed target distributions $\mu$ we will consider the loss functional defined as:
%\begin{align}\label{eq:loss_functional}
%\F(\nu) = \frac{1}{2} MMD^2(\mu,\nu) \qquad \forall \nu \in \mathcal{P}_2(\X).
%\end{align}
%The next section describes the dynamics of the gradient flow of \cref{eq:loss_functional} under the $2$-Wasserstein metric as defined in \cref{subsec:gradient_flows_functionals}.
%%The MMD was successfully used for training generative models (\cite{mmd-gan,Binkowski:2018,Arbel:2018}) where it is used in a loss functional to learn the parameters of the generator network. This motivate the  
%
%
%
%
%
%In this section we recall how to endow the space of probability measures $\mathcal{P}(\X)$ on $\X$ a convex  subset of $\R^d$ with a distance (e.g, optimal transport distances), and then deal with gradient flows of suitable functionals on such a metric space. The reader may refer to the recent review of \cite{santambrogio2017euclidean} for further details. For a given distributions $\nu\in\mathcal{P}(\X)$ and an integrable function $f$ under $\nu$, the expectation of $f$ under $\nu$ will be written either as $\nu(f)$ or $\int f \diff\nu$ depending on the context. 
%




%\section{Discretized Gradient flow of the MMD}
\subsection{Euler scheme}
The continuous-time gradient flow introduced in \cref{eq:continuity_mmd} allows to formally consider the notion of gradient descent on $\mathcal{P}_2(\X)$ with $\F$ as a cost function. This is done by considering a forward-Euler scheme of \cref{eq:continuity_mmd}. For any $T: \X \rightarrow \X$ a measurable map, and $\nu \in \mathcal{P}(\X)$, we denote the pushforward measure by $T_{\#}\nu$ (see the formal definition in \cref{subsec:wasserstein_flow}). Starting from an initial $\nu_0\in \mathcal{P}_2(\X)$ and using a step-size $\gamma>0$, it is possible to construct a sequence $\nu_n\in \mathcal{P}_2(\X)$ by iteratively applying:
\begin{align}\label{eq:euler_scheme}
	\nu_{n+1} = (I - \gamma \nabla f_{\mu,\nu_n})_{\#}\nu_n
\end{align} 
Equation \cref{eq:euler_scheme} is the distribution of the stochastic process defined by:
\begin{align}\label{eq:euler_scheme_particles}
	X_{n+1} = X_n - \gamma \nabla f_{\mu,\nu_n}(X_n) \qquad X_0\sim \nu_0.
\end{align}
The asymptotic behavior of \cref{eq:euler_scheme} as $n\rightarrow \infty$ can be analyzed much more easily than the continuous-time flow. \aknote{thought this was false? (since we have PL+the barrier in continuous time) but that the rates in discretized times are stronger} This will be the object of \cref{sec:convergence_mmd_flow}. For now, we provide a guarantee that the sequence $(\nu_n)_{n\geq 0}$ approaches $(\nu_t)_{t\geq 0}$ as $\gamma\rightarrow 0$:
\begin{proposition}\label{prop:convergence_euler_scheme}
	Consider the interpolation path $\rho_t^{\gamma}$ defined as:
	\begin{align}
		\rho_t^{\gamma} = (I-(t- n\gamma) \nabla f_{\mu,\nu_n})_{\#}\nu_n \qquad \forall t\in [n\gamma,(n+1)\gamma), \forall n\geq 0
	\end{align}
where $\nu_n$ is defined in \cref{eq:euler_scheme}. Then under \manote{asumptions: Lipschitz gradient + bounded hessian} and for all $T>0$:
	\begin{align}
		W_2(\rho_t^{\gamma},\nu_t)\leq \gamma C(T) \quad \forall t\in [0,T]
	\end{align}
\end{proposition} 
 A proof of \cref{prop:convergence_euler_scheme} is provided in \cref{proof:prop:convergence_euler_scheme} and relies on standard techniques to control the discretization error of a forward-Euler scheme.
\cref{prop:convergence_euler_scheme} means that $\nu_n$ can be interpolated giving rise to a path $\rho_t^{\gamma}$ which gets arbitrarily close to $\nu_t$ on finite intervals. Note that as $T \rightarrow \infty$ it is expected that the bound $C(T)$ would blow-up. However, this is enough to show that \cref{eq:euler_scheme} is indeed a discrete-time flow of $\F$. In fact, provided that $\gamma$ is small enough, $\F(\nu_k)$ is a decreasing sequence:
\begin{proposition}\label{prop:decreasing_functional}
	Under \cref{assump:bounded_trace,assump:bounded_hessian}, the following inequality holds:
	\begin{align*}
	\F(\nu_{n+1})-\F(\nu_n)\leq -\gamma (1-\frac{\gamma}{2}L )\int \Vert \phi_n(X)\Vert^2 d\nu_n
	\end{align*}
\end{proposition}
\cref{prop:decreasing_functional} is a discrete analog of \cref{prop:decay_mmd} and is proven in \cref{proof:prop:decreasing_functional}. In fact, \cref{eq:euler_scheme} is intractable in general as it requires to evaluate $f_{\mu,\nu_n}$ exactly at each iterations. Nevertheless, we present in \cref{sec:sample_based} a practical algorithm using a finite number of samples which is provably convergent towards \cref{eq:euler_scheme} as the sample-size increases. We thus begin by studying the convergence properties of the discretized in time MMD flow \eqref{eq:euler_scheme} in the next section. 

\section{MMD flow}\label{sec:mmd_flow}


%Let $\kH$ a Reproducing Kernel Hilbert Space (RKHS) and $k$ its reproducing kernel. This means that for all $f \in \kH$, $x \in \X$, we can write the \textit{reproducing property} $f(x)=\psh{f, k(x,.)}$. The kernel Maximum Mean Discrepancy between two distributions $\rho,\pi$ is defined as:
%\begin{equation}
%MMD(\rho,\pi)=\sup_{f \in \kH,  \|f\|_{\kH}\le 1} (\E_{X \sim \rho}[f(X)]-\E_{Y \sim \pi}[f(Y)])
%\end{equation}
%Under some appropriate assumptions on the kernel (see \cite{gretton2012kernel}):
%\begin{align}
%MMD^2(\rho,\pi)&=\|\E_{\rho}[k(X,.)] - \E_{\pi}[k(Y,.)]\|^2_{\kH}\\
%&=\E_{\rho \otimes \rho}[k(X,X')]+\E_{\pi \otimes %\pi}[k(Y,Y')] - 2\E_{\rho \otimes \pi}[k(X,Y)]
%\end{align}

We will consider a flow $(\rho_t)_{t>0}$ as described in \cref{sec:gradient_flows_functionals} and denote $f_t= \int k(.,z)\diff \mu - \int k(.,z)\diff \rho_t$. In this case:
\begin{equation}
\F(\rho_t)=\frac{1}{2}\|f_t\|^2_{\kH}
%&= \E_{\rho_t \otimes \rho_t}[k(X,X')]+\E_{\pi \otimes \pi}[k(Y,Y')] - 2\E_{\rho_t \otimes \pi}[k(X,Y)]
\end{equation} 

We define the potential energy (also called confinement energy) $V$ and interaction energy $W$ as follows:
\begin{equation}
V(X)=-\int 2 k(X,x')\pi(x')\text{,} \quad
W(X,Y)=k(X,Y)
\end{equation}
We have $MMD^2(\rho,\pi)=C+ \int V(x) \rho(x)dx + \int W(x,x')\rho(x)\rho(x')$, where $C=\E_{\pi\otimes \pi}[k(Y,Y')]$. $MMD^2$ can thus be written as a \textit{Lyapunov functional} (or "free energy" or "entropy") $\F$. \aknote{add that interestingly, both KL and MMD have the V potential term, but the diffusion of the particle derive from U for KL and from W for MMD?}


\begin{proposition}\label{prop:mmd_flow}
 The velocity in \eqref{eq:continuity_equation1} is given by $\nabla \frac{\partial{\F}}{\partial{\rho_t}}=2 \nabla f_t$ and the dissipation of MMD can be written:  
	\begin{equation}
	\frac{d MMD^2(\rho_t, \mu)}{dt}=-2 \E_{X \sim \rho_t}[\|\nabla f_t(X)\|^2]
	\end{equation}
	where $\nabla f_t(Y)= \int \nabla_{Y}k(.,Y) d\mu -  \int \nabla_{Y}k(.,Y) d\rho_t$.
\end{proposition}

\begin{remark}
	If the functional $\F$ was the KL divergence and $\rho_t$ a weak solution of the Fokker-Planck equation \eqref{eq:Fokker-Planck}, we would obtain the following dissipation (see \cite{wibisono2018sampling}):
	\begin{equation}
	\frac{d KL(\rho_t, \pi)}{dt}=-\E_{X \sim \rho_t}[\|\nabla log(\frac{\rho_t}{\pi}(X))\|^2]
	\end{equation}
\end{remark}




\subsection{Algorithm}

As explained in \cref{sec:gradient_flows_functionals} and according to \cref{prop:mmd_flow}, the gradient flow of the MMD can be written:
\begin{equation*}
\frac{\partial \rho}{\partial t}= 2 div(\rho  \nabla f_t)
\end{equation*}
which is the density of the stochastic process (see \cref{sec:ito_stochastic}):
\begin{equation}\label{eq:stochastic_process}
dX_t=-2\nabla f_t(X_t) 
\end{equation}
\eqref{eq:stochastic_process} represents the position $X_t$ of a particle at time $t > 0$.
We naturally consider the Euler discretization of \eqref{eq:stochastic_process}, which gives:
\begin{equation}\label{eq:discretization}
X_{k+1}=X_k - \gamma_{k+1} \nabla f_k(X_k)
\end{equation}
where $\nabla f_k(X_k)= \int \nabla_{X_k}k(X,X_k)\diff \mu - \int \nabla_{X_k}k(X,X_k) \diff \rho_k$ and $(\gamma_k)_{k\ge1}$ is a sequence of step sizes.


\begin{remark}[Stochastic setting]\aknote{to investigate much further} The ULA algorithm requires the knowledge of $\nabla \log \mu$, while our algorithm requires the one of $\nabla f_t$ and thus to integrate under $\mu$. However, in many situations, we only have access to samples of the target distribution $\mu$. Whereas $\nabla \log \mu$ may be difficult to estimate (see \cite{li2017gradient}), in contrast, the gradient of $f_t$ can be 'easily' estimated by:
\begin{equation}
\widehat{\nabla f_k}(X_k)=\frac{1}{n}\sum_{i=1,\dots,n}\nabla_{X_k}k(x_i,X_k) - \frac{1}{n}\sum_{i=1,\dots,n}\nabla_{X_k}k(y_i,X_k)
\end{equation}
where $(x_1, \dots, x_n)\sim \rho_k$ and $(y_1, \dots, y_n)\sim \pi$. 
\end{remark}


In this subsection we assume that $MMD^2$ is $\lambda$-geodesically-convex. Conditions under which this holds will be provided in the next section.

\subsection{Analysis of the theoretical algorithm}

Equation~\eqref{eq:discretization} provides a theoretical algorithm to minimize $MMD^2(\cdot,\pi)$. The algorithm is only theoretical because it requires to compute $\nabla f_k(X_k)$.

This algorithm is the discretization of the Gradient flow associated to $MMD^2$. Since $MMD^2$ is $\lambda$-convex, using Theorem 11.1.4 of~\cite{ambrosio2008gradient}, the gradient flow $(\rho_t)$ satisfies
\begin{equation}
    \label{eq:evi}
    \frac12 \frac{d}{dt} W^2(\rho_t,\nu) + \frac{\lambda}{2}W^2(\rho_t,\nu) \leq MMD^2(\nu,\mu) - MMD^2(\rho_t,\pi)
\end{equation}
Unfortunately, $\lambda \leq 0$ (otherwise it would mean that $MMD^2$ is strongly-geodesically-convex and hence geodesically convex).

For the theoretical algorithm~\eqref{eq:discretization} we can expect a discretized version of~\eqref{eq:evi} to hold : \asnote{This should hold, I haven't proved it yet but I will do it later}
\begin{equation}
    \label{eq:evi-discrete}
    W^2(\rho_{k+1},\pi) \leq  W^2(\rho_{k},\pi) -2\gamma_{k+1}\left( \frac{\lambda}{2}W^2(\rho_{k},\pi) + MMD^2(\rho_{k+1},\pi) - MMD^2(\pi,\pi)\right)
\end{equation}
Since $\lambda \leq 0$ we cannot have a rate from this inequality (I think).
However, if $\sum \gamma_k < \infty$, using Robbins Siegmund lemma, we know that $W^2(\rho_{k},\pi)$ converges to some $\ell \geq 0$. \asnote{From this it might be possible to prove that $W^2(\overline{\rho_{k}},\pi)$ converges to zero, but it would be a lot of work. It looks like Pakes Hasminskii criterion}

\subsection{Another Lyapunov function}

In this section we try to use $MMD^2(\cdot,\pi)$ as a Lyapunov function (instead of $W^2(\cdot,\pi)$), like in Theorem 3.3 of Liu 2017. Once this is done, we can use the Gradient Lojasiewicz inequality to get a rate (see Bolte, it's like log sobolev inequality)

Taylor : 


\subsection{Sample-based setting}

Two settings are usually encountered in the sampling literature:
\begin{itemize}
	\item Density-based: $\mu$ is known up to a constant
	\item Sample-based: we only have access to a set of samples $X \sim \mu$.
\end{itemize}

\aknote{to investigate much further} 
The Unadjusted Langevin Algorithm (ULA) seems much more adapted to the first setting, since it only requires the knowledge of $\nabla \log \mu$ while our algorithm requires the knowledge of $\mu$ (since $\nabla f_t$ involves an integration over $\mu$). However, in the sample-based setting, it may be difficult to adapt ULA by replacing $\nabla \log \mu$ in \eqref{eq:langevin_algorithm} by an estimator based on samples. Indeed, it has been the subject of a lot of work (see \cite{li2017gradient})\aknote{check reference}. In contrast, the gradient of $f_t$ can be 'easily' estimated by:
\begin{equation}
\widehat{\nabla f_k}(X_k)= \frac{1}{n}\sum_{i=1,\dots,n}\nabla_{X_k}k(y_i,X_k) -\frac{1}{n}\sum_{i=1,\dots,n}\nabla_{X_k}k(x_i,X_k) 
\end{equation}
where $(x_1, \dots, x_n)\sim \rho_k$ and $(y_1, \dots, y_n)\sim \mu$. 
It is thus natural to consider the stochastic process:
\begin{equation}\label{eq:sample_based_stochastic_process}
Y_{k+1}=Y_k-\gamma_{k+1}\widehat{\nabla f_k}(Y_k) 
\end{equation}


\begin{proposition} Let $\widehat{\rho}_k$ be the distribution of~\eqref{eq:sample_based_stochastic_process}.\aknote{not sure, proof not complete}
	\begin{equation}
	MMD^2(\rho_k,\widehat{\rho_k})\le \frac{C}{n}
	\end{equation}
\end{proposition}
\begin{proof}
	Let $\mu, \nu$ in $\mathcal{P}(\X)$. Suppose that $k$ is bounded and measurable on $\X$, and that there exists $L_k$ such that $\forall x,y \in \X$, $\| k(x,.)-k(y,.) \|_{\kH}^2\le L_k \|x-y\|^2$. By Proposition 20 in \cite{sriperumbudur2010hilbert} we firstly get the following inequality between $MMD$ and $W_2$:
	\begin{align}
	MMD^2(\mu, \nu)	 &\le \inf_{\pi \in \Pi(\mu, \nu)} \int \| k(x,.)-k(y,.) \|_{\kH}^2 d\pi(\mu, \nu)\\
	&\le  \inf_{\pi \in \Pi(\mu, \nu)} \int L_k \| x-y \|^2 d\pi(\mu, \nu)= L_k W_2^2(\mu, \nu)
	\end{align}
	We now introduce the process:
	\begin{equation}\label{eq:intermediary_process}
	Z_{k+1}=Z_k-\gamma_{k+1}\widetilde{\nabla f_k}(Z_k) 
	\end{equation}
	where $\widetilde{\nabla f_k}(Z_k)=\int \nabla_{Z_k}k(X,Z_k)\diff \mu- \frac{1}{n}\sum_{i=1,\dots,n}\nabla_{Z_k}k(x_i,Z_k)$. Let $\widetilde{\rho}_k$ be the distribution of \eqref{eq:intermediary_process}.
	\begin{align}\label{eq:decompose_process}
	MMD^2(\widehat{\rho_k}, \rho_k) \le MMD^2(\widehat{\rho_k}, \widetilde{\rho_k})+ MMD^2(\widetilde{\rho_k}, \rho_k)
	\end{align}
	 Then, concerning the first term on r.h.s. of \eqref{eq:decompose_process}:
	 \begin{align}
	 W_2^2(\widehat{\rho_k}, \widetilde{\rho_k}) \le \inf_{\pi \in \Pi(\widehat{\rho_k}, \widetilde{\rho_k})} \int \| y-z \|^2 d\pi(\widehat{\rho_k}, \widetilde{\rho_k}) \le \int \|y-z\|^2 d\widehat{\rho_k}(y)d\widetilde{\rho_k}(z)\le \frac{C_1}{n}
	 \end{align}
	 where the last inequality results from Theorem 3 in \cite{jourdain2007nonlinear}. \aknote{we need Lipschitz continuity of the coefficient in the process... check}
	 Then, we try a similar proof than Theorem 3 in \cite{jourdain2007nonlinear} for the first r.h.s. in \eqref{eq:decompose_process}:
	 \begin{align}
	 \sup_{k}\|X_k-Y_k\|^2 \le C W_2^2(\widehat{\mu},\mu)
	 \end{align}
	where $\widehat{\mu}=\frac{1}{n}\sum_{i=1, \dots,n}\delta_{y_i}$ and $y_i \sim \mu$. For the first inequality we also need Lipschitz continuity of the coefficient in the process.
\end{proof}

\begin{remark}
	We point out here that algorithm~\eqref{eq:sample_based_stochastic_process} is different from the descent proposed by \cite{mroueh2018regularized}. 
\end{remark}

\begin{remark}
	Birth-Death Dynamics to improve convergence (see \cite{rotskoff2019global}).
\end{remark}


\section{Discretizing the MMD gradient Flow}\label{sec:discretized_flow}



\subsection{Convergence of the time-discretized flow}

The time discretized flow can be written:\aknote{add pushforward $\#$ definitions}
\begin{align}\label{eq:discretized_flow}
\nu_{m+1} = (I -\gamma \phi_m)_{\#}\nu_m
\end{align}
where $\gamma$ is some fixed step-size and $X \mapsto \phi_m(X):=\nabla f_{\mu, \nu_m}(X)$ is the gradient of the witness function between $\mu$ and $\nu_m$.% It is easy to see that the particle version of \cref{eq:discretized_flow} is given by:
%\begin{align}
%X_{n+1} = X_n - \gamma \phi_n(X_n) \quad\forall n\in \mathbb{N}.
%\end{align} 
\cref{prop:almost_convex_optimization} guarantees the existence of a direction of descent that minimizes the functional $\F$ provided that the starting point $\rho_1$ has a potential greater than the barrier $K$.%, i.e:
%\begin{align}\label{eq:barrier_condition}
%	\F(\rho_1)> \inf_{\rho\in \mathcal{P}} \F(\rho) + K
%\end{align}
One natural question to ask is whether the  discretized gradient flow algorithm provides such way to reach the barrier $K$ and at what speed this happens. This subsection will answer that question. Firstly, we state few propositions that will lead us to the final result.


%\begin{proposition}\label{prop:decreasing_functional}
%	Under \cref{assump:bounded_trace,assump:bounded_hessian}, the following inequality holds:
%	\begin{align*}
%	\F(\nu_{n+1})-\F(\nu_n)\leq -\gamma (1-\frac{\gamma}{2}L )\int \Vert \phi_n(X)\Vert^2 d\nu_n
%	\end{align*}
%\end{proposition}

\begin{proposition}\label{prop:evi}
	Consider the sequence of distributions $\nu_m$ obtained from \cref{eq:discretized_flow}. If $\gamma \leq 1/L$, then
	\begin{align}
2\gamma(\F(\nu_{m+1})-\F(\mu))
\leq 
W_2^2(\nu_m,\mu)-W_2^2(\nu_{m+1},\mu)-2\gamma K(\rho^m).
\label{eq:evi}
\end{align}
where $(\rho^m_t)_{0\leq t \leq 1}$ is a constant-speed geodesic from $\nu_n$ to $\mu$ and $K(\rho^m):=\int_0^1 \Lambda(\rho^m_s,\dot{\rho^m}_s)(1-s)ds$.
\end{proposition}

\begin{theorem}\label{th:rates_mmd}
	Consider the sequence of distributions $\nu_n$ obtained from \cref{eq:discretized_flow}. If $\gamma \leq 1/L$, then
	%\begin{align}
%\F(\bar{\nu}_{n})-\F(\mu)\leq  \frac{W_2^2(\nu_0,\mu)}{2 \gamma n} -\bar{K}
%\end{align}
%where $\bar{\nu}=\frac{1}{N}\sum_{n=1}^N \nu_n$. Moreover, 
\begin{align}
\F(\nu_m)-\F(\mu)\leq  \frac{W_2^2(\nu_0,\mu)}{2 \gamma m} -\bar{K}.
\end{align}
\end{theorem}



%\subsection{Approximate Euler-Maruyama scheme}\label{subsec:euler_maruyama}

\subsection{The sample-based approximate scheme}\label{sec:sample_based}

In the gradient descent \eqref{eq:discretized_noisy_flow}, the vector field depends on the target distribution $\mu$ and the current distribution $\nu_n$ at each iteration $n$, to which we do not have access. This suggests a simple approximate scheme, based on samples of the two latter distributions, that can be computed in practice. Specifically, we adopt the common approach (sometimes referred to as \textit{mean-field interaction} in mathematical physics and stochastic analysis) which considers a system of $N$ interacting particles $(X_n^{1,N}, X_n^{2,N}, \dots, X_n^{N,N})$ and their empirical distribution in order to approximate $\nu_n$. 
Given i.i.d. samples $(X^i_0)_{1\leq i\leq N}$ and $(Y^{m})_{1\leq m\leq M}$ from $\nu_0$ and $\mu$ and step-size $\gamma$, the approximate scheme iteratively updates the i-th particle using the following update equation: 
\begin{multline}\label{eq:euler_maruyama}
X_{n+1}^{i} = X_n^i -\gamma \nabla f_{\hat{\mu},\hat{\nu}_n}(X_n^i+\beta_n U_n^i)\; \text{ with }\\ \nabla f_{\hat{\mu},\hat{\nu}_n}(.) = \frac{1}{M}\sum\limits_{m=1}^M \nabla_2 k(Y_m,.)-\frac{1}{N}\sum\limits_{j=1}^N \nabla_2 k(X_n^j,.)
\end{multline}
where $U_{n}^{i}$ are i.i.d standard gaussians, $\hat{\mu}$, $\hat{\nu}_n$ denote respectively the empirical distributions of $(Y^{m})_{1\leq m\leq M}$ and $(X^i_n)_{1\leq i\leq N}$, and $\nabla_2k(x,.)$ denotes the gradient of $k$ w.r.t. the second variable for any $x \in \X$. One nice property about this scheme is that $\nabla f_{\hat{\mu},\hat{\nu}_n}(.)$ can be evaluated easily by the available samples.
%\begin{equation}
%\nabla f_{\hat{\mu},\hat{\nu}_n}(X_n^i+\beta_n U_n^i) = \frac{1}{M}\sum\limits_{m=1}^M \nabla_2 k(Y_m,X_n^i+\beta_n U_n^i)-\frac{1}{N}\sum\limits_{j=1}^N \nabla_2 k(X_n^j+\beta_n U_n^j,X_n^i+\beta_n U_n^i)
%\nabla f_{\hat{\mu},\hat{\nu}_n}(.) = \frac{1}{M}\sum\limits_{m=1}^M \nabla_2 k(Y_m,.)-\frac{1}{N}\sum\limits_{j=1}^N \nabla_2 k(X_n^j,.)
%\end{equation}
%where $\nabla_2k(x,.)$ denotes the gradient of $k$ w.r.t. the second variable for any $x \in \X$.
However, unlike the algorithms derived from the Negative Sobolev distance or the Sliced Wasserstein gradient flows (see respectively \cite{Mroueh:2019,Simsekli:2018}), \cref{eq:euler_maruyama} is computationally less expensive. Indeed the cost of each iteration is $O((M+N)N)$ and uses $O(M+N)$ in memory.
\cref{eq:euler_maruyama} can be seen as a particle version of \cref{eq:euler_scheme} and one would expect it to converge towards the population version as $M$ and $N$ goes to infinity. This is made more formal in the following theorem.
%\begin{theorem}\label{prop:convergence_euler_maruyama}
%Under the same conditions as in \cref{prop:convergence_euler_scheme} and for any $\frac{T}{\gamma}\geq n\geq 0$:
%\begin{equation}
%\mathbb[E][W_2(\hat{\nu}_n,\nu_n)] = \frac{1}{2}(\frac{var(\mu)^\frac{1}{2}}{M} + \frac{(var(\nu_0)^\frac{1}{2})}{N}e^{LT})(e^LT-1)
%\end{equation}
%Where $M(\gamma k)$ is a constant that only depends on $\gamma n$ and the choice of the kernel $k$.
%\end{theorem}
\begin{theorem}\label{prop:convergence_euler_maruyama}
	For any $\frac{T}{\gamma}\geq n\geq 0$:\aknote{check conditions}
\[
\mathbb{E}[W_{2}(\hat{\nu}_{n},\nu_{n})]\leq \frac{1}{2}\left(\frac{1}{\sqrt{N}}(B+var(\nu_{0})^{\frac{1}{2}})e^{LT}+\frac{1}{\sqrt{M}}var(\mu))\right)(e^{LT}-1)
\]
\end{theorem}
%Notice that since we do not have access to true samples of $\nu_n$ at each iteration $n$, we have adopted the common approach (sometimes referred to as \textit{mean-field interaction} in mathematical physics and stochastic analysis) which considers the system of  $N$ interacting particles $(X_n^{1,N}, X_n^{2,N}, \dots, X_n^{N,N})$ and their empirical distribution in order to approximate $\nu_n$.
\cref{prop:convergence_euler_maruyama} tries to control the propagation of the chaos at each iteration and uses techniques from \cite{Jourdain:2007}, however, the proof is much simpler since it is provided in discrete time. Notice also that these rates remain true when no noise is added to the updates, i.e. for the original flow when $B=0$. A proof is provided in \cref{proof:propagation_chaos}. The dependence in $\sqrt{M}$ underlines the fact that our algorithm could be interesting as a sampling algorithm when one only has access to $M$ samples of $\mu$ (see \cref{subsec:kl_flow} for a more detailed discussion).

%\begin{remark}
%Two settings are usually encountered in the sampling literature: \textit{density-based}, i.e. the target $\mu$ is known up to a constant, or \textit{sample-based}, i.e. we only have access to a set of samples $X \sim \mu$. The Unadjusted Langevin Algorithm (ULA), which involves a time-discretized version of the Langevin diffusion (see \cref{remark:gradient_flow}), seems much more suitable for first setting, since it only requires the knowledge of $\nabla \log \mu$, whereas our algorithm requires the knowledge of $\mu$ (since $\nabla f_{\mu, \nu_n}$ involves an integration over $\mu$). However, in the sample-based setting, it may be difficult to adapt the ULA algorithm, since it would require firstly to estimate $\nabla \log(\mu)$ based on a set of samples of $\mu$, before plugging this estimate in the update of the algorithm. This problem, sometimes referred to as \textit{score estimation} in the literature, has been the subject of a lot of work but remains hard especially in high dimensions (see \cite{sutherland2017efficient},\cite{li2018gradient},\cite{shi2018spectral}). In contrast, the discretized flow of the $MMD^2$ presented in this section seems naturally adapted to the sample-based setting.
%\end{remark}












%
%
%
%\cref{eq:mcKean_Vlasov_process} suggests a time discretized approximation to \cref{eq:continuity_mmd} which will be analyzed in \cref{sec:convergence_mmd_flow}:
%\begin{align}\cref{eq:time_discretized_flow}
%	X_{n+1} = X_n - \gamma \nabla f_{\nu_n}(X_n) \qquad X_0\sim \nu_0
%\end{align}
%where $\gamma >0$ is the step-size. Using standard techniques from
% 
%
%\paragraph{Modified gradient flow}
%
%
\textbf{Experiments}


\begin{figure}[ht]
	\centering
	\includegraphics[width=0.8\linewidth]{figures/Gaussians_error_4}
	\caption{An illustration of the performance of the regular versus noisy MMD flow.}
	\label{fig:experiments}
\end{figure}

We finally illustrate \ref{fig:experiments} the behavior of the proposed algorithm, with a simple experiment. The target distribution $\mu$ and initial distribution $\nu_0$ are taken respectively as $\mathcal{N}(0,a)$ and $\mathcal{N}(0,b)$, and the number of particles $N=1000$ and $M=1000$. The middle figure represent the initial samples. Algorithm \eqref{eq:euler_maruyama} is run for $10^4$ iterations, when $\beta_n=0$ for all $n$ (which corresponds to the regular discretized MMD flow), and when $\beta_n=?$. The right figure illustrates that the samples of the regular MMD flow seem stuck in a local minima and did not converge \aknote{however the red curve is not flat yet, beware}, whereas the ones of the Noisy MMD flow did. The results of this experiment confirm that the injected noise seem to regularize the MMD flow and improve convergence.











\section{Conclusion}


We have introduced MMD flow, a novel flow over the space of distributions, with a practical space-time discretized implementation and a regularisation scheme to improve convergence. We provide theoretical results, highlighting intrinsic properties of the regular MMD flow, and guarantees on convergence based on  recent results in optimal transport, probabilistic interpretations of PDEs, and particle algorithms. Future work will focus on a deeper understanding of  regularization for  MMD flow,
and its application in sampling and optimization for large neural networks.

%to which we make a first attempt here, would open to new algorithms for sampling or optimization of big neural networks.% Also, since this flow can be closely related to the optimization of very big neural networks with gradient descent, a deeper understanding of our noisy algorithm in comparison to regularization procedures used in deep learning nowadays, should be of great interest.


\section*{TO DO}

~\cref{sec:mmd_flow}.
\begin{itemize}
	\item Investigate sample-based algorithm
	\item Birth-Death Dynamics
	\item Experiments!!
\end{itemize}

~\cref{sec:theory}
\begin{itemize}
	\item Get rates in $\nu_n$ (change Lyapunov function)
	\item Hard: Refine the bounds, quantify more precisely $K(\rho_n)$
	\item Hard: Polyak Lojasewicz (PL) for MMD
\end{itemize}
everywhere: treat comments if possible

%\subsubsection*{References}
%\renewcommand\refname{\vskip -1cm}
%\bibliographystyle{apalike}
%\bibliography{biblio}

\printbibliography

\clearpage
%\newpage


\section{Appendix}

\subsection{SDE and stochastic processes}

Consider the Itô process, i.e. the stochastic process:
\begin{equation}
dX_t=g(X_t)dt
\end{equation}
Let $f \in \mathcal{C}^2(\Omega)$, Itô's formula can be written:
\begin{equation*}
df(X_t)=\nabla f(X_t).g(X_t)dt
\end{equation*}
Let $\rho_t$ be the distribution of the process $X_t$. We have:
\begin{align*}
\E[\frac{df}{dt}(X_t)]&= \E[\nabla f(X_t).g(X_t)]\\
\Longleftrightarrow \int f(X) \frac{d \rho_t}{dt}(X)&=-\int f(X)div(g(X)\rho_t(X))
\end{align*}
where the second line is obtained by integrating by parts on both sides of the equality. Finally, the distribution $\rho_t$ verifies: 
\begin{equation*}
\frac{d\rho_t}{dt}=div(g\rho_t)
\end{equation*}




\subsection{Displacement convexity}



\begin{proof} \ref{prop:lambda_convexity}
To prove that $\nu\mapsto MMD^{2}(\mu,\nu)$ $\Lambda_{\mu}$-convex
we need to compute the second derivative $\frac{d^{2}}{dt^{2}}MMD^{2}(\mu,\rho_{t})$
where $\rho_{t}$ is a minimizing geodesic between two probability
distributions $\nu_{0}$ and $\nu_{1}$. When $\nu_{0}$ and $\nu_{1}$
both have a density, there exists a convex function such that $\rho_{t}=(1-t)Id+t\nabla\phi)_{\#}\nu_{0}:=(\pi_{t})_{\#}\nu_{0}$
.We start by computing the first derivative:
\[
\frac{dMMD^{2}(\mu,\rho_{t})}{dt}=2\langle f_{t},\frac{df_{t}}{dt}\rangle_{\mathcal{H}}
\]
where $f_{t}=\rho_{t}(k(x,.))-\mu(k(x,.))$. Using the definition
of $\rho_{t}=(1-t)Id+t\nabla\phi)_{\#}\nu_0$ it follows that:
\[
\frac{df_{t}}{dt}=\int(\nabla\phi(x)-x).\nabla k(\pi_{t}(x),.)\nu_{0}(x)dx
\]
hence:
\[
\frac{dMMD^{2}(\mu,\rho_{t})}{dt}=2\int(\nabla\phi(x)-x).\nabla f_{t}(\pi_{t}(x))\nu_{0}(x)dx
\]
Now the second derivative is given by:
\begin{align*}
\frac{d^{2}MMD^{2}(\mu,\rho_{t})}{dt^{2}}= & \int(\nabla\phi(x)-x).Hf_{t}(\pi_{t}(x))(\nabla\phi(x)-x)\nu_{0}(x)dx\\
 & +\int(\nabla\phi(x)-x).\nabla_{1}\nabla_{2}k(\pi_{t}(x),\pi_{t}(x'))(\nabla\phi(x')-x')\nu_{0}(x)\nu_{0}(x')dxdx'
\end{align*}
Here $\nabla_{1}\nabla_{2}k(x,x')$ is the matrix whose components
are given by $\langle\partial_{i}k(x,.),\partial_{j}k(x,.)\rangle$
for $1\leq i,j\leq d$, and $Hf_{t}$ is the hesssian of $f_{t}$
and its components are also given by:
\[
(Hf_{t}(x))_{i,j}=\langle f_{t},\partial_{i}\partial_{j}k(x,.)\rangle.
\]
Denoting by $h(x):=\nabla\phi(x)-x$ it follows that:
\begin{align*}
\frac{d^{2}MMD^{2}(\mu,\rho_{t})}{dt^{2}}= & \langle f_{t},\int\sum_{i,j}h_{i}(x)h_{j}(x)\partial_{i}\partial_{j}k(\pi_{t}(x),.)\nu_{0}(x)dx\rangle\\
 & +\Vert\int\sum_{i}h_{i}(x)\partial_{i}k(\pi_{t}(x),.)\nu_{0}(x)dx\Vert^{2}
\end{align*}
Now we use Cauchy-Schwarz inequality for the first term to get:
\begin{align*}
\frac{d^{2}MMD^{2}(\mu,\rho_{t})}{dt^{2}}\geq & -\Vert f_{t}\Vert_{\mathcal{H}}\Vert\int\sum_{i,j}h_{i}(x)h_{j}(x)\partial_{i}\partial_{j}k(\pi_{t}(x),.)\nu_{0}(x)dx\Vert_{\mathcal{H}}\\
 & +\Vert\int\sum_{i}h_{i}(x)\partial_{i}k(\pi_{t}(x),.)\nu_{0}(x)dx\Vert^{2}.
\end{align*}
After aplying a change of variables $x=\pi_{t}(y)$ one recovers the
velocity vector $v_{t}$ instead of $h$: 
\begin{align*}
\frac{d^{2}MMD^{2}(\mu,\rho_{t})}{dt^{2}}\geq & -\Vert f_{t}\Vert_{\mathcal{H}}\Vert\int\sum_{i,j}v_{t}^{i}(x)v_{t}^{j}(x)\partial_{i}\partial_{j}k(x,.)\rho_{t}(x)dx\Vert_{\mathcal{H}}\\
 & +\Vert\int\sum_{i}v_{t}^{i}(x)\partial_{i}k(x,.)\rho_{t}(x)dx\Vert^{2}.
\end{align*}

One can further note that:
\[
\Vert\int\sum_{i,j}v_{t}^{i}(x)v_{t}^{j}(x)\partial_{i}\partial_{j}k(x,.)\rho_{t}(x)dx\Vert_{\mathcal{H}}\leq\lambda\Vert v_{t}\Vert_{L_{2}(\rho_{t})}^{2}
\]

and that 
\begin{align*}
\Vert\int\sum_{i}v_{t}^{i}(x)\partial_{i}k(x,.)\rho_{t}(x)dx\Vert^{2} & =\int v_{t}(x)^{T}\int\nabla_{1}\nabla_{2}k(x,x')v_{t}(x')\rho_{t}(x')dx'dx.\\
 & =\langle v_{t},C_{\rho_{t}}v_{t}\rangle_{L_{2}(\rho_{t})}
\end{align*}

Hence we have shown that 
\[
\frac{d^{2}MMD^{2}(\mu,\rho_{t})}{dt^{2}}\geq\langle v_{t},(C_{\rho_{t}}-\lambda MMD(\mu,\rho_{t})I)v_{t}\rangle_{L_{2}(\rho_{t})}=\Lambda_{\mu}(\rho_{t},v_{t})
\]
\end{proof}


\end{document}
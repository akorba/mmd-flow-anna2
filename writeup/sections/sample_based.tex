%\subsection{Convergence of the space discretized flow - Sample-based setting}

%Given samples $(u_1, \dots, u_N)\sim \mu, (v_1, \dots, v_N)\sim \nu_t$, the gradient of $f_{\mu, \nu_t}$ can be easily estimated by:
%\begin{equation}
%\nabla f_{\widehat{\mu},\widehat{\nu_t}}(z)= \frac{1}{N}\sum_{i=1}^{N}\nabla_{z}k(u_i,z) -\frac{1}{N}\sum_{i=1}^{N}\nabla_{z}k(v_i,z) 
%\end{equation}
%where $\widehat{ \mu}=\sum_{j=1}^{N}\delta_{u_i}$ and $\widehat{ \nu_t}=\sum_{j=1}^{N}\delta_{v_i}$. However, we do not have access to $(v_1, \dots, v_N)\sim \nu_t$ at each time $t$. A common approach (sometimes referred to as \textit{mean-field interaction} in mathematical physics and stochastic analysis) is to consider the following system of $n$ interacting particles $(X_t^{1,N}, X_t^{2,N}, \dots, X_t^{N,N})$: 
%\begin{equation}\label{eq:sample_based_process}
%%X_t^{j,N}=X_{0}^j+\int_{0}^t \nabla f_{\widehat{\mu}, \nu_s^N}(X_s^{j,N})ds \quad \text{where } \nu_t^N=\frac{1}{N} \sum_{i=1}^N \delta_{X_t^{j,N}}
%\dot{X}_t^{j,N}=-\nabla f_{\widehat{\mu}, \nu_t^N}(X_t^{j,N}) \text{ where } \nu_t^N=\frac{1}{N} \sum_{i=1}^N \delta_{X_t^{j,N}}
%\end{equation}
%%&\forall s \in [0,T]\;,\quad \widehat{\rho}_s^n=\sum_{j=1}^{n} \delta_{\widehat{X}_s^{j,n}} \text{ denotes the empirical measure } 
%The following proposition, whose proof is deferred to the Appendix, quantifies the distance between the target distribution and the one of the latter particle system.
%%Notice that the coefficient $\nabla f_{\mu, \nu_t}$ in \eqref{eq:mcKean_Vlasov_process} has been replaced by $\nabla f_{\widehat{\mu}, \widehat{\nu_t}}$ where $\widehat{\nu}_t$ is the density of the process defined in \eqref{eq:sample_based_process}. 
%\begin{proposition}\label{prop:sample_based_rates}
%	\aknote{the bound could be refined, and there are few mild assumptions - see the proof for details/discussion. In particular, by taking $t=\sqrt{n}$, we could get a $\mathcal{O(\frac{1}{\sqrt{N}})}$ bound.} Let $\nu_t^N$, be the distribution of the particle system \eqref{eq:sample_based_process}. We have, for all $t>0$:
%	\begin{equation}
%	MMD^2(\mu,\nu_t^N)\le \frac{C_1(t)}{N}+ \frac{C_2}{N} + \frac{C_3}{t}
%	\end{equation}
%\end{proposition}
%Let us also introduce the Euler discretizations with time-step $h > 0$ of the SDE \eqref{eq:mcKean_Vlasov_process} (with $\widehat{ \mu}$ as a target distribution) and the particle system:
%\begin{equation}\label{eq:discret_time}
%X_{m+1}=X_m - \gamma \nabla f_{\widehat{\mu}, \nu_m}(X_m) \end{equation}
%\begin{equation}\label{eq:sample_based_discret_time}
%X_{m+1}^{j,N}=X_{m}^{j,N}- \gamma \nabla f_{\widehat{\mu}, \nu_m^N}(X_m^{j,N}) \quad \text{where } \nu_m^N=\frac{1}{N} \sum_{i=1}^N \delta_{X_m^{j,N}}
%\end{equation}
%\begin{proposition}%\label{prop:sample_based_rates}
%	Let $\nu_m$,$\nu_m^N$, be the distributions of \eqref{eq:discret_time} and the particle system \eqref{eq:sample_based_discret_time}. We have, for all $t>0$:\aknote{from Jourdain 2019. Unfortunately I think it's hard to inherit this result in the noisy case}
%	\begin{equation}
%	MMD^2(\nu_m,\nu_m^N)\le \frac{C_1(t)}{N}
%	\end{equation}
%\end{proposition}
%
%\begin{remark}
%	Two settings are usually encountered in the sampling literature: \textit{density-based}, i.e. the target $\mu$ is known up to a constant, or \textit{sample-based}, i.e. we only have access to a set of samples $X \sim \mu$.
%	The Unadjusted Langevin Algorithm (ULA), which involves a time-discretized version of the Langevin diffusion (see \cref{rk:kl_flow}), seems much more suitable for first setting, since it only requires the knowledge of $\nabla \log \mu$, whereas our algorithm requires the knowledge of $\mu$ (since $\nabla f_{\mu, \nu_t}$ involves an integration over $\mu$). However, in the sample-based setting, it may be difficult to adapt the ULA algorithm, since it would require firstly to estimate $\nabla \log(\mu)$ based on a set of samples of $\mu$, before plugging this estimate in the update of the algorithm. This problem, sometimes referred to as \textit{score estimation} in the literature, has been the subject of a lot of work but remains hard especially in high dimensions (see \cite{sutherland2017efficient,li2018gradient,shi2018spectral}). In contrast, the discretized flow of the $MMD^2$ presented in this section seems naturally adapted to the sample-based setting.
%\end{remark}

\subsection{Approximate Euler-Maruyama scheme}\label{subsec:euler_maruyama}
\cref{eq:euler_scheme} suggests a simple approximate scheme that can be computed in practice. Such scheme is known as the Euler-Maruyama scheme.
Given i.i.d. samples $(X^i_0)_{1\leq i\leq N}$ and $(Y^{m})_{1\leq m\leq M}$ from $\nu_0$ and $\mu$ and step-size $\gamma$, the approximate scheme iteratively updates the particles $i$ using the following update equation: 
\begin{align}\label{eq:euler_maruyama}
X_{n+1}^{i} = X_n^i -\gamma \nabla f_{\hat{\mu},\hat{\nu}_n}(X_n^i)
\end{align}

Here $\hat{mu}$ and $\hat{\nu_n}$ denote the empirical distributions of $(Y^{m})_{1\leq m\leq M}$ and $(X^i_n)_{1\leq i\leq N}$. One nice property about this scheme is that $\nabla f_{\hat{\mu},\hat{\nu}_n}(X_n^i)$ can be evaluated easily:
\begin{align}
\nabla f_{\hat{\mu},\hat{\nu_n}}(X_n^i) = \frac{1}{M}\sum_{m=1}^M \nabla_2 k(Y_m,X_n^i)-\frac{1}{N}\sum_{j=1}^N \nabla_2 k(X_n^j,X_n^i)
\end{align}
\cref{eq:euler_maruyama} is based on a similar discretization approaches as in \cite{Mroueh:2019,Simsekli:2018}. However, unlike the algorithms derived from the Negative Sobolev distance or the Sliced Wasserstein gradient flows, \cref{eq:euler_maruyama} is computationally less expensive. Indeed the cost of each iteration is $O((M+N)N)$ and uses $O(M+N)$ in memory.
\cref{eq:euler_maruyama} can be seen as a particle version of \cref{eq:euler_scheme} and one would expect it to converge towards the its population version as $M$ and $N$ goes to infinity. This is made more formal in 
\begin{theorem}\label{prop:convergence_euler_maruyama}
	Under the same conditions as in \cref{prop:convergence_euler_scheme} and for any $\frac{T}{\gamma}\geq n\geq 0$:
	\begin{equation}
	\mathbb[E][W_2(\hat{\nu}_n,\nu_n)] = \frac{1}{2}(\frac{var(\mu)^\frac{1}{2}}{M} + \frac{(var(\nu_0)^\frac{1}{2})}{N}e^{LT})(e^LT-1)
	\end{equation}
	Where $M(\gamma k)$ is a constant that only depends on $\gamma n$ and the choice of the kernel $k$.
\end{theorem}
\cref{prop:convergence_euler_maruyama} tries to control the propagation of the chaos at each iteration and uses techniques from \cite{Jourdain:2007}, however, the proof is much simpler since it is provided in discrete time. A proof is provided in \cref{prop:propagation_chaos}.
%
%
%
%\cref{eq:mcKean_Vlasov_process} suggests a time discretized approximation to \cref{eq:continuity_mmd} which will be analyzed in \cref{sec:convergence_mmd_flow}:
%\begin{align}\cref{eq:time_discretized_flow}
%	X_{n+1} = X_n - \gamma \nabla f_{\nu_n}(X_n) \qquad X_0\sim \nu_0
%\end{align}
%where $\gamma >0$ is the step-size. Using standard techniques from
% 
%
%\paragraph{Modified gradient flow}
%
%















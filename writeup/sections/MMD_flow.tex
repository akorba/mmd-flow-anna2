\section{Theoretical properties of the MMD gradient flow}\label{sec:mmd_flow}

\subsection{MMD gradient flow}

We will consider a flow $(\rho_t)_{t>0}$ as described in \cref{sec:gradient_flows_functionals} and denote $f_t= \int k(.,z)\diff \mu - \int k(.,z)\diff \rho_t$. In this case:
\begin{equation}
\F(\rho_t)=\frac{1}{2}\|f_t\|^2_{\kH}
%&= \E_{\rho_t \otimes \rho_t}[k(X,X')]+\E_{\pi \otimes \pi}[k(Y,Y')] - 2\E_{\rho_t \otimes \pi}[k(X,Y)]
\end{equation} 

We define the potential energy (also called confinement energy) $V$ and interaction energy $W$ as follows:
\begin{align*}
	V(x)=-\int  k(x,x')\mu(x')\text{,} \quad
W(x,x')=\frac{1}{2}k(x,x')
\end{align*}
We have $1/2MMD^2(\rho,\mu)=C+ \int V(x) \rho(x)dx + \int W(x,x')\rho(x)\rho(x')$, where $C=1/2\E_{\mu\otimes \mu}[k(x,x')]$. $MMD^2$ can thus be written as a \textit{Lyapunov functional} (or "free energy" or "entropy") $\F$ as in \cref{eq:lyapunov}.

\begin{proposition}\label{prop:mmd_flow}
 The velocity in \cref{eq:continuity_equation1} is given by $\nabla \frac{\partial{\F}}{\partial{\rho_t}}=\nabla f_t$ and the dissipation of MMD can be written:  
	\begin{equation}
	\frac{d MMD^2(\rho_t, \mu)}{dt}=-\E_{X \sim \rho_t}[\|\nabla f_t(X)\|^2]
	\end{equation}
	where $\nabla f_t(z)= \int \nabla_{z}k(.,z) d\mu -  \int \nabla_{z}k(.,z) d\rho_t$.
\end{proposition}

\begin{remark}
	If the functional $\F$ was the KL divergence and $\rho_t$ a weak solution of the Fokker-Planck equation \cref{eq:Fokker-Planck}, we would obtain the following dissipation (see \cite{wibisono2018sampling}):
	\begin{equation}
	\frac{d KL(\rho_t, \mu)}{dt}=-\E_{X \sim \rho_t}[\|\nabla log(\frac{\rho_t}{\mu}(X))\|^2]
	\end{equation}
\end{remark}


As explained in \cref{sec:gradient_flows_functionals} and according to \cref{prop:mmd_flow}, the gradient flow of the MMD can be written:
\begin{equation}\label{eq:continuity_equation_mmd}
\frac{\partial \rho_t}{\partial t}= div(\rho_t  \nabla f_t)
\end{equation}
The stochastic process whose distribution satisfies \cref{eq:continuity_equation_mmd} can thus be written (see \cref{sec:ito_stochastic}):
\begin{equation}\label{eq:stochastic_process}
dX_t=-\nabla f_t(X_t) = - (\nabla V (X_t) + \nabla W * \rho_t(X_t))
\end{equation}
%Equation \cref{eq:stochastic_process} can be interpreted as the position $X_t$ of a particle at time $t > 0$.
\aknote{The following is based on the formalism and some results of \cite{jourdain2007nonlinear}} Equation \cref{eq:stochastic_process}, which can be interpreted as the position $X_t$ of a particle at time $t > 0$, can be written as a Mac-Kean Vlasov model\aknote{reference}, a particular kind of SDE driven by a Levy process:
\begin{align}\label{eq:theoretical_process}
&X_t=X_{0}+\int_{0}^t \sigma(X_s, \rho_s, \mu)ds \quad \text{for t in [0,T]}\\
&\forall s \in [0,T]\;,\quad \rho_s \text{ denotes the probability distribution of } X_s
\end{align}
with $\sigma(X_s, \rho_s, \mu)=-\nabla f_t(X_s)=\int \nabla_{X_s}k(.,X_s) d\rho_t -  \int \nabla_{X_s}k(.,X_s) d\mu$. Notice that $\sigma$ is bounded \aknote{true?} and Lipschitz continuous in its second and third variable and bounded.\aknote{investigate conditions on the kernel for convergence, uniqueness. e.g. linear growth of the coefficient sigma? or does it relate to lambda convexity as santambrogio says?}

\begin{remark}
	Consider a family of particles such that its density satisfy Equation\cref{eq:continuity_equation}. Both KL and MMD have a non-zero potential energy $V$ which drive these particles to the target distribution $\mu$. While he entropy function $U$ in KL prevents the particle from "crashing" onto the mode of $\mu$, this role could be played by the interaction energy $W$ for MMD. Indeed, when $W$ is convex, this gives raise to a general
	aggregation behavior of the particles, while when it is not, the particles would push each other apart.\aknote{to check, ref malrieu?}
\end{remark}

%\section{Theoretical properties of the MMD flow}\label{sec:theory}



\subsection{Lambda displacement convexity of the MMD}

One important criterion to characterize the convergence of the gradient flow of a functional $\F$ is the notion of \textit{displacement convexity} of such a functional. Displacement convexity (see \cite{Villani:2004}, Definition 1). states that the functional evaluated at any distribution in a geodesic path between two distributions $\nu$ and $\nu'$ will be upper-bounded by a convex mixture of $\F(\nu)$ and $\F(\nu')$, as explained formally in the following definition.
\begin{definition}\label{def:displacement_convexity}
 Let $\mu$
and $\nu$ be two probabilities densities. There exists a $\mu-a.e.$
unique gradient of a convex function, denoted by $\nabla\phi$, such that $\nu$
is equal to $\nabla\phi_{\#}\mu$ and one can define \aknote{the displacement geodesic?} $\rho_{t}=((1-t)Id+t\nabla\phi)_{\#}\mu$
for $0\leq t\leq1$. We say that a functional $\nu\mapsto\mathcal{F}(\nu)$
is displacement convex if 
\[
t\mapsto\mathcal{F}(\rho_{t})
\]
 is convex for any $\mu$ and $\nu$. Moreover, we say that $\mathcal{F}$
is displacement convex in a neighborhood of $\mu$ if there exists a radius $r>0$
such that the above property holds for any $\nu$ with $W_{2}(\mu,\nu)\leq r$.
\end{definition}


This notion of convexity is to be related to the more widely used notion of convexity called \textit{mixture convexity}:
\begin{align}
	\F(t\nu +(1-t)\nu')\leq t\F(\nu)+(1-t)\F(\nu') \qquad t\in [0,1]
\end{align}
%Unlike mixture convexity, displacement convexity is compatible with the $W_2$ metric and is therefore the natural notion to use for characterizing convergence of gradient flows in the $W_2$ metric.
Although mixture convexity holds for $\F$ (see \cref{lem:mixture_convexity}), this property is less critical for characterizing convergence of gradient flows in the $W_2$ metric. On the other hand, displacement convexity is compatible with the $W_2$ metric \cite{Bottou:2017} and is therefore the natural notion to use in our setting. Unfortunately, $\F$ fails to be displacement convex in general. Instead we will show that $\F$ satisfies some weaker notion of convexity called $\Lambda$-displacement convexity:
%
\begin{definition}\label{def:lambda-convexity}
($\Lambda$-convexity \cite{Villani:2009} Definition 16.4). Let $(\mu,v)\mapsto\Lambda(\mu,v)$
be a function that defines for each probability distribution $\mu$
a quadratic form on the set of square integrable vectors valued functions
$v$ , i.e: $v\in L_{2}(\mathbb{R}^{d},\mathbb{R}^{d},\mu)$ . We
further assume that:
\[
\inf_{\mu,v}\frac{\Lambda(\mu,v)}{\Vert v\Vert_{L_{2}(\mu)}^{2}}>-\infty.
\]
We say that a functional $\mu\mapsto\mathcal{F}(\mu)$ is $\Lambda$-convex
if for any $\mu$ and $\nu$ and a minimizing geodesic $\text{\ensuremath{\rho_{t}}}$
between $\mu$ and $\nu$ with velocity vector field $v_{t}$, i.e:
$\partial_{t}\rho_{t}+div(\rho_{t}v_{t})=0;\rho_{0}=\mu;\rho_{1}=\nu$
the following holds:
\begin{equation*}
\frac{d^{2}\mathcal{F}(\rho_{t})}{dt^{2}}\geq\Lambda(\rho_{t},v_{t})\qquad\forall t\in[0,1].
\end{equation*}
\end{definition}

To show the $\Lambda$-convexity of the functional defined in \cref{eq:MMD_functional} we first make the following assumptions on the kernel:
\begin{assumplist} 
\item \label{assump:bounded_trace} $ \vert \sum_{1\leq i\leq d} \partial_i\partial_ik(x,x) \vert\leq \frac{L}{3}  $ for all $x\in \mathbb{R}^d$.
\item \label{assump:bounded_hessian} $\Vert H_xk(x,y) \Vert_{op} \leq \frac{L}{3}$ for all $x,y\in \mathbb{R}^d$, where $H_xk(x,y)$ is the hessian of $x\mapsto k(x,y)$ and $\Vert.\Vert_{op}$ is the operator norm.
\item \label{assump:bounded_fourth_oder} $\Vert Dk(x,y) \Vert\leq \lambda  $ for all $x,y\in \mathbb{R}$, where $Dk(x,y)$ is an $\mathbb{R}^{d^2}\times \mathbb{R}^{d^2}$ matrix with entries given by $\partial_{x_{i}}\partial_{x_{j}}\partial_{x'_{i}}\partial_{x_{j}'}k(x,x')$.
\end{assumplist}\aknote{do we have an order of magnitude for lambda? or just we put a remark to say it's satisfied by the gaussian kernel}
The next proposition states that the functional defined in \cref{eq:MMD_functional} is $\Lambda$-displacement convex and provide and explicit expression for the functional $\Lambda$.

\begin{proposition}
\label{prop:lambda_convexity} Suppose \cref{assump:bounded_fourth_oder} is satisfied for some $\lambda \in \R^+$. The functional $\nu\mapsto \F(\nu)$ is $\text{\ensuremath{\Lambda}}$-convex
with $\Lambda$ given by:
\begin{equation}
\Lambda(\rho,v)=\langle v,(C_{\rho}-\lambda \F(\rho)^{\frac{1}{2}}I)v\rangle_{L_{2}(\rho)}\label{eq:Lambda}
\end{equation}
where $C_{\rho}$ is the (positive) operator defined by:
\begin{align}\label{eq:positive_operator_C}
	(C_{\rho}v)(x)=\int\nabla_{x}\nabla_{x'}k(x,x')v(x')d\rho(x')
\end{align}
\end{proposition}
%
%
Consider the geodesic \aknote{path/geodesic/curve?}$\rho_{t}=((1-t)Id+t\nabla\phi)_{\#}\mu$ of \cref{def:displacement_convexity}. It is worth noting that $\rho_{0}=\mu$ and at time $t=0$ we have
that $\F(\rho_{0})=0$, hence we get:
\[
\frac{d^{2}\F(\rho_{t})}{dt^{2}}\vert_{t=0}=\langle v_{t},C_{\rho_{t}}v_{t}\rangle_{L_{2}(\rho_{t})}\geq0.
\]
This shows that $\nu\mapsto \F(\nu)$ has a non-negative
hessian at $\mu$ which is not surprising since $\mu$ is the global
minimum of this functional.
\begin{corollary}\label{cor:integral_lambda_convexity}
For any geodesic $\rho_{t}$ between two probability distributions
$\rho_{0}$ and $\rho_{1}$ the following holds:
\begin{equation}
\F(\rho_{t})\leq(1-t)\F(\rho_{0})+t\F(\rho_{1})-\int_{0}^{1}\Lambda(\rho_{s},v_{s})G(s,t)ds\label{eq:integral_lambda_convexity}
\end{equation}
where $\Lambda$ is given by \cref{eq:Lambda} and $G$ is given
by:
\[
G(s,t)=\begin{cases}
s(1-t) & s\leq t\\
t(1-s) & s\geq t
\end{cases}
\]
\end{corollary}
%

\begin{corollary}
\label{cor:loser_bound}Assume the distributions are supported on
$\mathcal{X}$ and the kernel is bounded, i.e: $\sup_{x,y\in\mathcal{X}}\vert k(x,y)\vert<\infty$.
Then the following holds:
\begin{equation}
\F(\rho_{t})\leq(1-t)\F(\rho_{0})+t\F(\rho_{1})+t(1-t)K
\end{equation}
where $K$ is a constant depending on $\X$ and the kernel $k$ in $\F$.
\end{corollary}
%
%
\cref{cor:loser_bound}, is a loser bound and does not account for the local
convexity of the MMD. However, it allows to state the following result,
which is inspired from (\cite{Bottou:2017}, Theorem 6.3) but generalizes
it to the case of 'almost convex' functionals.
\begin{proposition}
\label{prop:almost_convex_optimization}
(Almost convex optimization). Let $\mathcal{P}$ be a closed subset
of $\mathcal{P}(\mathcal{X})$ which is displacement convex\aknote{weird for a set to be displacement convex? it was for functionals}. Then
for all $M>\inf_{\rho\in\mathcal{P}}\F(\rho)+K$, the following
holds:
\end{proposition}
\begin{enumerate}
\item The level set $L(\mathcal{P},M)=\{\rho\in\mathcal{P}:\F(\rho)\leq M\}$
is connected
\item For all $\rho_{0}\in\mathcal{P}$ such that $\F(\rho_0)>M$
and all $\epsilon>0$, there exists $\rho\in\mathcal{P}$ such that
$W_{2}(\rho,\rho_{0})=\mathcal{O}(\epsilon)$ and
\[
\F(\rho)\leq \F(\rho_{0})-\epsilon(\F(\rho_{0})-M).
\]
\end{enumerate}
%
\begin{remark}
The result in \Cref{prop:almost_convex_optimization} means that it is possible to optimize the cost function $\rho\mapsto \F(\rho)$
on $\mathcal{P}$ as long as the barrier $\inf_{\rho\in\mathcal{P}}\F(\rho)+K$
is not reached. We provide now a simple proof of this result.
\end{remark}


\begin{remark}
	A possible direction would be to directly leverage the tighter inequality in \cref{eq:integral_lambda_convexity} to get a better description of the loss landscape.
\end{remark}








\subsection{Lojasiewicz type inequality - Convergence of the continuous flow}

Here we would like to derive an inequality between the time derivative of the Lyapounov functional $\mathcal{F}$ along its gradient flow $t\mapsto \nu_t$. For this purpose we first introduce the weighted negative Sobolev distance \manote{cite villani and peyre and Mroueh}:
\begin{align}\label{eq:neg_sobolev}
	\Vert \nu - \mu \Vert_{\dot{H}^{-1}(\nu)} = \sup_{\substack{ f\in W_0^{1,2}(\nu) \\ \nu(\Vert \nabla f \Vert^2) \leq 1 }} \vert \nu(f)-\mu(f)\vert 
\end{align}
Where $W_0^{1,2}(\nu)$ is the space $1$ order Sobolev functions with functions vanishing at the boundary of the domain.
The distance defined in \cref{eq:neg_sobolev} plays a fundamental role in dynamic optimal transport as it linearizes the $W_2$ distance when $\mu$ is arbitrarily close to $\nu$. It can also be seen as the minimum kinetic energy needed to advect the mass $\nu$ to $\mu$. However, this quantity might be infinite \manote{say exactly when it is finite} and one of the key problems would be to control its value during the evolution of the flow. More precisely we will rely on the following statement:
\begin{align}\label{eq:bounded_neg_sobolev}
	\Vert \nu_t  - \mu \Vert_{\dot{H}^{-1}(\nu_t)} \leq C \qquad \forall t\geq 0.
\end{align} 
where $\nu_t$ is defined by the gradient flow and $\mu$ is the target distribution. When \cref{eq:bounded_neg_sobolev}  holds, we have the following proposition:
\begin{proposition}\label{prop:PL_type_inequality}
	When \cref{eq:bounded_neg_sobolev} holds, the following inequality is then satisfied at all times:
	\begin{align}\label{eq:PL_type_inequality}
		\Vert \nabla f_t \Vert_{L_2(\nu_t)} \geq \frac{1}{C} \Vert f_t \Vert^2_{\mathcal{H}} \qquad \forall t\geq 0.
	\end{align}
\end{proposition}
\begin{proof}
	Indeed, this follows simply from the definition of the negative Sobolev distance: Consider $g = \Vert \nabla f_t\Vert^{-1}_{L_2(\nu_t)} f_t$, then $g\in W_0^{1,2}(\nu)$ \manote{this suggests an assumption on the kernel so that all those function satisfy a boundary condition} and $\Vert \nabla g \Vert_{L_2(\nu_t)}\leq 1$. Therefore, we directly have:
	\begin{align}
		\Vert \nu_t - \mu\Vert_{\dot{H}^{-1}(\nu_t)}\geq \vert \nu_t(g) - \mu(g)  \vert.
	\end{align}
Now, recall the definition of $g$, which implies that
\[
\vert \nu_t(g) - \mu(g)  \vert = \Vert \nabla f_t\Vert^{-1}_{L_2(\nu_t)} \vert \nu_t(f_t)-\mu(f_t)\vert.
\]
But since $f_t$  is exactly the witness functions between $\nu_t$ and $\mu$, it follows that $\nu_t(f_t)-\mu(f_t) = \Vert f_t\Vert^2_{\kH}$.
Using \cref{eq:bounded_neg_sobolev}, we get the desired inequality.
\end{proof}

Now we will use the inequality in \cref{prop:PL_type_inequality} to prove a convergence result towards the global optimum $\mu$. This is provided in \cref{prop:convergence}.

\begin{proposition}\label{prop:convergence}
	If \cref{eq:bounded_neg_sobolev} is satisfied for all times then $t\mapsto \mathcal{F}(\nu_t)$ converges to $0$ with a rate of convergence given by:
	\begin{align}
		\mathcal{F}(\nu_t)\leq \frac{1}{\mathcal{F}(\nu_0)^{-1} + \frac{4t}{C}}
	\end{align}
\end{proposition}
\begin{proof}
	The proof is a simple consequence of \cref{prop:mmd_flow,eq:bounded_neg_sobolev}. Indeed, by \cref{prop:mmd_flow} we have that 
	\begin{align}
		\dot{\F}(\nu_t) = - \Vert \nabla f_t \Vert^2_{L_2(\nu_t)} 	
	\end{align}
	Using \cref{eq:PL_type_inequality}, we directly get that:
	\begin{align}
		\dot{\F}(\nu_t) \leq  -\frac{4}{C}\F(\nu_t)^2
	\end{align}
It is clear that if $\mathcal{F}(\nu_0)>0$ then $\F(\nu_t)>0$ at all times by uniqueness of the solution. Hence, one can divide by $\F(\nu_t)^2$ and integrate the inequality from $0$ to some time $t$. The desired inequality is obtained by simple calculations.
\end{proof}

All the difficulty is to see now when \cref{eq:bounded_neg_sobolev} holds. One possible strategy would be to start from initial $\nu_0$ such that $\Vert \nu_0  - \mu \Vert_{\dot{H}^{-1}(\nu_0)} \leq C $  for some finite positive value $C$ and then show that this property is preserved during the dynamics. It is also possible to have a time depended constant $C_t$ as long as its growth is such that:
\begin{align}
	\lim_{t\rightarrow +\infty} \int_0^t C_s^{-1}\diff s = +\infty
\end{align}
For instance $C_t$ could have up to a linear growth in time. In this case the decay of $\F(\nu_t)$ will no longer be in $\frac{1}{t}$ but only in $\frac{1}{\log(t)}$ \manote{This seems unlikely if we end up having convergence of $\nu_t$, but who nows.}.
One possible promising condition for \cref{eq:bounded_neg_sobolev} to hold would be if $\mu \ll \nu_0$ and if this property is preserved during the dynamics.


 










\subsection{MMD flows in the literature}

\begin{remark}
	We point out here that algorithm~\cref{eq:sample_based_process} is different from the descent proposed by \cite{mroueh2018regularized}. 
\end{remark}

\begin{remark}
	Birth-Death Dynamics to improve convergence (see \cite{rotskoff2019global}).
\end{remark}

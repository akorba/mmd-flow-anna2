

\section{Convergence properties of the MMD flow}\label{sec:mmd_flow}




%\subsection{MMD as a free energy}
%
%Interestingly, for a fixed target distribution $\mu$, it appears that the $MMD^2$ can be written
%as a free energy \cref{eq:lyapunov}, by choosing the potential energy $V$ and interaction energy $W$ as follows:
%\begin{align}
%V(x)=-\int  k(x,x')\mu(x')\text{,} \quad
%W(x,x')=\frac{1}{2}k(x,x')
%\end{align}
%Indeed, in this case we have $(1/2)MMD^2(\rho,\mu)=C+ \int V(x) \rho(x)dx + \int W(x,x')\rho(x)\rho(x')$, where $C=(1/2)\E_{\mu\otimes \mu}[k(x,x')]$. 
% Hence, we can consider a flow $(\rho_t)_{t>0}$ as described in \cref{sec:gradient_flows_functionals} and the loss functional defined as:
%\begin{equation}\label{eq:mmd_functional}
%\F(\rho_t)=\frac{1}{2}MMD^2(\rho_t, \mu)=\frac{1}{2}\|f_t\|^2_{\kH}
%%&= \E_{\rho_t \otimes \rho_t}[k(X,X')]+\E_{\pi \otimes \pi}[k(Y,Y')] - 2\E_{\rho_t \otimes \pi}[k(X,Y)]
%\end{equation} 
%where $f_t(z)= \int k(.,z)\diff \mu - \int k(.,z)\diff \rho_t$ to alleviate notations. The following proposition gives the time dissipation of $\F$ as well as the associated gradient flow.
%
%\begin{proposition}\label{prop:mmd_flow} The gradient flow associated to $\F$ and as the dissipation can be written:
%\begin{equation}\label{eq:continuity_equation_mmd}
%\frac{\partial \rho_t}{\partial t}= div(\rho_t  \nabla f_t) ,\quad  \quad \frac{d \F(\rho_t)}{dt}=-\E_{X \sim \rho_t}[\|\nabla f_t(X)\|^2]\quad
%\end{equation}
% where $\nabla f_t$ is the gradient of the witness function $f_t$, defined by $\nabla f_t(z)= \int \nabla_{z}k(.,z) d\mu -  \int \nabla_{z}k(.,z) d\rho_t$. %Hence, the dissipation of $\F$ can be written:  
% %\begin{equation}
% %\frac{d \F(\rho_t)}{dt}=-\E_{X \sim \rho_t}[\|\nabla f_t(X)\|^2]\quad
% %\end{equation}
%\end{proposition}
%\begin{remark}
%	If the functional $\F$ was the KL divergence and $\rho_t$ a weak solution of the Fokker-Planck equation \cref{eq:Fokker-Planck}, we would obtain the following dissipation (see \cite{wibisono2018sampling}):
%	\begin{align}\label{eq:decreasing_mmd}
%	\frac{d KL(\rho_t, \mu)}{dt}=-\E_{X \sim \rho_t}[\|\nabla log(\frac{\rho_t}{\mu}(X))\|^2]
%	\end{align}
%\end{remark}
%A stochastic process whose distribution satisfies \cref{eq:continuity_equation_mmd} can thus be written (see \cref{sec:ito_stochastic} on ItÃŽ's formula):
%\begin{equation}\label{eq:stochastic_process}
%dX_t=-\nabla f_t(X_t) = - (\nabla V (X_t) + \nabla W * \rho_t(X_t))
%\end{equation}
%It can be interpreted as the position $X_t$ of a particle at time $t > 0$, following the velocity vector field $\nabla \frac{\partial{\F}}{\partial{\rho_t}}=\nabla f_t$.  Equation \eqref{eq:stochastic_process} is actually a McKean Vlasov model (see \cite{kac1956foundations}, \cite{mckean1966class}), a particular kind of SDE (Stochastic Differential Equation) driven by a Levy process:
%\begin{align}\label{eq:theoretical_process}
%&X_t=X_{0}+\int_{0}^t \sigma_{\mu}(X_s, \rho_s)ds \quad \text{for t in [0,T]}\\
%&\forall s \in [0,T]\;,\quad \rho_s \text{ denotes the probability distribution of } X_s
%\end{align}
%with $\sigma_{\mu}(X_s, \rho_s)=-\nabla f_s(X_s)=\int \nabla_{X_s}k(.,X_s) d\rho_s -  \int \nabla_{X_s}k(.,X_s) d\mu$, and $X_0$ is distributed according to a given initial measure $\rho_0$. The non-linearity in the SDE \eqref{eq:theoretical_process} appears through the dependency of its coefficients on the law of the process. Suppose that $\nabla k$ is bounded and measurable on $\X$, and that there exists $L_k$ such that $\forall x,y \in \X$, $\|\nabla k(x,.)-\nabla k(y,.) \|_{\kH}\le L'_k \|x-y\|$. Hence, $\sigma$  Lipschitz continuous on $\X \times \mathcal{P}_2(\X)$ (endowed with the product of the canonical metric on $\X$ and $W_2$ on $\mathcal{P}_2(\X)$) and Equation~\eqref{eq:theoretical_process} admits a unique solution (see \cite{jourdain2007nonlinear}).
%In contrast, it is much more difficult to prove the existence and uniqueness of the invariant probabilities of the PDE \eqref{eq:continuity_equation_mmd}. 
%Indeed, to the best of our knowledge, such a property relies generally on the convexity of $V+2W$; if this holds, the process \eqref{eq:theoretical_process} converges
%toward the unique invariant probability as the time goes to infinity. Moreover, if the confining potential is not convex, it has been shown in some cases that the diffusion may admit several invariant probabilities (see \cite{herrmann2010non, tugaut2014phase}).%, tugaut2014self
%\aknote{What about the conditions for existence and uniqueness of the PDE\eqref{eq:continuity_equation_mmd}? does it relate to lambda convexity as santambrogio says? or Following Chizat Bach solutions exist for all t > 0 for appropriate initial $\mu_0$ that are compactly
%	supported in $\X$?}
%\begin{remark}
%	Consider a family of particles whose density satisfy Equation\cref{eq:continuity_equation} for some free energy $\F$. Both KL and $MMD^2$ can be written as free energies \eqref{eq:lyapunov}, with a potential energy $V$ term which drive the particles to the target distribution $\mu$. While the entropy function $U$ in KL prevents the particle from "crashing" onto the mode of $\mu$, this role could be played by the interaction energy $W$ for $MMD^2$. Indeed, when $W$ is convex, this gives raise to a general aggregation behavior of the particles, while when it is not, the particles would push each other apart.\aknote{to check, ref malrieu?}
%\end{remark}

 

%\section{Theoretical properties of the MMD flow}\label{sec:theory}



%\subsection{Lambda displacement convexity of the MMD}
\subsection{Optimization in a ($W_2$) non-convex setting}
\label{subsection:barrier_optimization}
One important criterion to characterize the convergence of the Wasserstein gradient flow of a functional $\F$ is the \textit{displacement convexity} of such a functional. The latter states that $t\mapsto \F(\nu_t)$ is a convex function whenever $t\mapsto\nu_t$ is a path of minimal length from two distributions $\mu$ and $\nu$ as explained in \cref{def:displacement_convexity} (see also \cite[Definition 16.5]{Villani:2004}). %The notion of path of minimal length depends on the choice of the metric. 
Such paths are called  \textit{constant speed displacement geodesics} when additionally their velocity vector has a constant norm. We refer to \cite{Bottou:2017} for a more in-depth discussion.
\textit{Displacement convexity} should not be confused with \textit{mixture convexity} which corresponds to the usual notion of convexity. As a matter of fact, $\F$ is \textit{mixture convex} in that it satisfies: $\F(t\nu +(1-t)\nu')\leq t\F(\nu)+(1-t)\F(\nu')$ for all $t\in [0,1]$ and $\nu,\nu'\in\mathcal{P}_2(\X)$ (see \cref{lem:mixture_convexity}). Unfortunately, $\F$ is not \textit{displacement convex}. Instead, $\F$ only satisfies a weaker notion of displacement convexity called $\Lambda$-displacement convexity (\cite[Definition 16.5]{Villani:2009} and  \cref{def:lambda-convexity}): 
%This implies that the gradient flow of $\F$ might not converge to the optimal solution. 
%This could happen if for instance \cref{eq:Lojasiewicz_inequality} doesn't hold. 
%It can be shown, however, that $\F$ is guaranteed to reach some barrier. This is a consequence of the $\Lambda$-displacement convexity of $\F$ which is 
\begin{proposition}
	\label{prop:lambda_convexity} Suppose $\sup_{(x,y) \in \X, (i,j) \in \llbracket 1, d \rrbracket} \partial_{x_{i}}\partial_{x_{j}}\partial_{x'_{i}}\partial_{x_{j}'}k(x,y)\le B$  holds for some $B \in \R^+$. Then for all $\nu, \nu'\in \mathcal{P}_2(\X)$ and any \textit{displacement geodesic} $(\nu_t)_{1\leq t\leq 1}$ from $\nu$ to $\nu'$ with velocity vectors $(V_t)_{0\leq t\leq 1}$ the following holds:
	\begin{equation}
	\F(\nu_{t})\leq(1-t)\F(\nu)+t\F(\nu')-\int_0^1 \Lambda(\nu_s, V_s ) G(s,t)\diff s
	\end{equation}
	where $s,t\mapsto G(s,t)$ is the one-dimensional Green function: $G(s, t) =  s(1-t) \mathbbm{1}\{s\leq t\}+ t(1-s) \mathbbm{1}\{s\geq t\}$ and $\Lambda$ is defined for any pair $(\nu,V)\in \mathcal{P}_2(\X)\times L_2(\rho)$  by:
	\begin{align}\label{eq:lambda}
		\Lambda(\nu,V) = \Vert \int V(x).\nabla_x k(x,.) \diff \nu(x) \Vert^2_{\mathcal{H}} - B\F(\nu)^{\frac{1}{2}}  \int \Vert  V(x) \Vert^2 \diff \nu(x) 
	\end{align}
	\end{proposition}
\cref{prop:lambda_convexity} can be obtained by computing the second time-derivative of $\F(\rho_t)$ which is then lower-bounded by $\Lambda(\rho_t,V_t)$ (see \cref{proof:prop:lambda_convexity} for a full proof).
In \cref{eq:lambda}, the map $\Lambda$ is a difference of two non-negative terms thus $\int_0^1 \Lambda(\rho_s, V_s ) G(s,t)\diff s$ can become negative, hence displacement convexity doesn't hold in general. However, it is still possible to provide an upper-bound on the asymptotic value of $\F(\nu_n)$ when $\nu_n$ are obtained using \cref{eq:euler_scheme}. Such upper-bound would depend on a scalar $ K(\rho^n) :=  \int_0^1\Lambda(\rho_s^n,V_s^n)(1-s)\diff s$ where $(\rho_s^n)_{0\leq s\leq 1}$ is a \textit{constant speed displacement geodesic} from $\nu_n$ to the optimal value $\mu$ with velocity vectors $(V_s^n)_{0\leq s\leq 1}$. \cref{th:rates_mmd} provides such bound:
\begin{theorem}\label{th:rates_mmd}
	Consider the sequence of distributions $\nu_n$ obtained from \cref{eq:euler_scheme} and let us denote $\bar{K}$ the average value of $(K(\rho^j))_{0\leq j \leq n}$ over iterations from $0$ to $n$. \aknote{show that $\bar{K}$ is bounded!!} If $\gamma \leq 1/L$, then
\begin{align}
\F(\nu_n)\leq  \frac{W_2^2(\nu_0,\mu)}{2 \gamma n} -\bar{K}.
\end{align}
\end{theorem}
\cref{th:rates_mmd} is obtained using techniques from optimal transport and optimization. It relies on \cref{prop:lambda_convexity} and \cref{prop:decreasing_functional} to prove an \textit{extended variational inequality} (see \cref{prop:evi}) and concludes using a suitable Lyapunov function, a full proof is given in\cref{proof:th:rates_mmd}.
When $\bar{K}$ is non-negative, one recovers the usual convergence rate as $O(\frac{1}{n})$ for the gradient descent algorithm. However, $\bar{K}$ can be negative in general and would therefore act as a barrier on the optimal value that $\F(\nu_n)$ can achieve. In that sense, the above result is similar to \cite[Theorem 6.9]{Bottou:2017}. In fact, \cref{th:rates_mmd} provides only a loose bound. In \cref{sec:Lojasiewicz_inequality} we show global convergence when some energy functional remains controlled.
%However, it will be sufficient to show that the flow of $\F$ can decrease until it reaches a barrier. %The size of the barrier depends on a relaxed notion of convexity called $\Lambda$-displacement convexity:

%It can be shown that $\F$ is $\text{\ensuremath{\Lambda}}$-displacement convex under mild assumptions on the kernel $k$:
%\begin{proposition}
%\label{prop:lambda_convexity} Suppose \cref{assump:bounded_fourth_oder} is satisfied for some $\lambda \in \R^+$. Then $\F$ is $\text{\ensuremath{\Lambda}}$-displacement convex with $  \Lambda(\rho,v) = -\lambda \mathcal{\rho}^{\frac{1}{2}} \int \Vert v(x) \Vert^2 \diff\rho(x)$.
%Moreover, for any displacement geodesic $\rho_t$ between two distributions $\nu$ and $\nu'$ it holds 
%\begin{align}
%	\bar{\F}(\rho_t) \geq -\lambda \F{\rho_t}^{\frac{1}{2}} W_2^2(\nu,\nu')
%\end{align}
%\end{proposition}


%%Unlike mixture convexity, displacement convexity is compatible with the $W_2$ metric and is therefore the natural notion to use for characterizing convergence of gradient flows in the $W_2$ metric.
%Although mixture convexity holds for $\F$ (see \cref{lem:mixture_convexity}), this property is less critical for characterizing convergence of gradient flows in the $W_2$ metric. On the other hand, displacement convexity is compatible with the $W_2$ metric \cite{Bottou:2017} and is therefore the natural notion to use in our setting. Unfortunately, $\F$ fails to be displacement convex in general. Instead we will show that $\F$ satisfies some weaker notion of convexity called $\Lambda$-displacement convexity:
%
%\begin{definition}\label{def:lambda-convexity}
%	We say that a functional $\nu\mapsto\mathcal{F}(\nu)$ is $\Lambda$-convex
%	if for any $\nu$ and $\mu$ and a minimizing geodesic $\text{\ensuremath{\nu_{t}}}$
%	between $\nu$ and $\mu$ with velocity vector field $v_{t}$, i.e:
%	$\partial_{t}\nu_{t}+div(\nu_{t}v_{t})=0;\nu_{0}=\nu;\nu_{1}=\mu;$
%	the following holds:
%	\begin{equation}\label{eq:lambda_displacement_convex}
%	\frac{d^{2}\mathcal{F}(\nu_{t})}{dt^{2}}\geq\Lambda(\nu_{t},v_{t})\qquad\forall\; t\in[0,1].
%	\end{equation}
%	where $(\nu,v)\mapsto\Lambda(\nu,v)$
%	is a function that defines for each $\nu \in \mathcal{P}(\X)$
%	a quadratic form on the set of square integrable vectors valued functions
%	$v$ , i.e: $v\in L_{2}(\mathbb{R}^{d},\mathbb{R}^{d},\nu)$. We
%	further assume that $\inf_{\nu,v}\Lambda(\nu,v)/\Vert v\Vert_{L_{2}(\nu)}^{2}>-\infty$. 
%	Also, the following holds:
%	\begin{equation}\label{eq:integral_lambda_convexity}
%	\F(\nu_{t})\leq(1-t)\F(\nu_{0})+t\F(\nu_{1})-\int_{0}^{1}\Lambda(\nu_{s},v_{s})G(s,t)ds
%	\end{equation}
%	where $G(s,t)=s(1-t) \mathbb{I}\{s\leq t\}
%	+t(1-s) \mathbb{I}\{s\geq t\}$.
%\end{definition}
%
%%Then, to show the $\Lambda$-convexity of the functional defined in \cref{sec:gradient_flow} we first make the following assumptions on the kernel:
%%\begin{assumplist} 
%%	\item \label{assump:bounded_trace} $ \vert \sum_{1\leq i\leq d} \partial_i\partial_ik(x,x) \vert\leq \frac{L}{3}  $ for all $x\in \mathbb{R}^d$.
%%	\item \label{assump:bounded_hessian} $\Vert H_xk(x,y) \Vert_{op} \leq \frac{L}{3}$ for all $x,y\in \mathbb{R}^d$, where $H_xk(x,y)$ is the hessian of $x\mapsto k(x,y)$ and $\Vert.\Vert_{op}$ is the operator norm.
%%	\item \label{assump:bounded_fourth_oder} $\Vert Dk(x,y) \Vert\leq \lambda  $ for all $x,y\in \mathbb{R}$, where $Dk(x,y)$ is an $\mathbb{R}^{d^2}\times \mathbb{R}^{d^2}$ matrix with entries given by $\partial_{x_{i}}\partial_{x_{j}}\partial_{x'_{i}}\partial_{x_{j}'}k(x,y)$.
%%\end{assumplist}
%The next proposition states that the functional defined in \cref{sec:gradient_flow} is $\Lambda$-displacement convex and provide and explicit expression for the functional $\Lambda$. Some additional mild assumptions on the derivative of the kernel are also needed but deferred to the appendix for presentation purpose.\aknote{to do!!}
%
%\begin{proposition}
%	\label{prop:lambda_convexity} Suppose $\sup_{x,y} \partial_{x_{i}}\partial_{x_{j}}\partial_{x'_{i}}\partial_{x_{j}'}k(x,y)\le \lambda$ is satisfied for some $\lambda \in \R^+$. The functional $\nu\mapsto \F(\nu)$ is $\text{\ensuremath{\Lambda}}$-convex
%	with $\Lambda$ given by:
%	\begin{equation}
%	\Lambda(\nu,v)=\langle v,(C_{\nu}-\lambda \F(\nu)^{\frac{1}{2}}I)v\rangle_{L_{2}(\nu)}\label{eq:Lambda}
%	\end{equation}
%	where $C_{\nu}$ is the (positive) operator defined by $(C_{\nu}v)(x)=\int\nabla_{x}\nabla_{x'}k(x,x')v(x')d\nu(x')$ for any $x \in \X$.
%	%\begin{align}\label{eq:positive_operator_C}
%	%(C_{\nu}v)(x)=\int\nabla_{x}\nabla_{x'}k(x,x')v(x')d\nu(x')
%	%\end{align}
%\end{proposition}
%%
%%


%\begin{proposition}
%	\label{prop:loser_bound}Assume the distributions are supported on
%	$\mathcal{X}$ and the kernel is bounded, i.e: $\sup_{x,y\in\mathcal{X}}\vert k(x,y)\vert<\infty$.
%	Then the following holds:
%	\begin{equation}
%	\F(\nu_{t})\leq(1-t)\F(\nu_{0})+t\F(\nu_{1})+t(1-t)K
%	\end{equation}
%	where $K$ is a constant depending on $\X$ and the kernel $k$ in $\F$.
%\end{proposition}
%
%

%\cref{prop:loser_bound}, is a loose bound and does not account for the local
%convexity of the MMD (see \cref{subsec:lambda_convexity} for more details on the convexity properties of this flow). However, it allows to state the following result,
%which is inspired from (\cite{Bottou:2017}, Theorem 6.3) but generalizes
%it to the case of 'almost convex' functionals.
%\begin{proposition}
%	\label{prop:almost_convex_optimization}
%	(Almost convex optimization). Let $\mathcal{P}$ be a closed subset
%	of $\mathcal{P}(\mathcal{X})$ which is displacement convex\aknote{weird for a set to be displacement convex? it was for functionals}. Then
%	for all $M>\inf_{\nu\in\mathcal{P}}\F(\nu)+K$, the following
%	holds:
%\end{proposition}
%\begin{enumerate}
%	\item The level set $L(\mathcal{P},M)=\{\nu\in\mathcal{P}:\F(\nu)\leq M\}$
%	is connected
%	\item For all $\nu_{0}\in\mathcal{P}$ such that $\F(\nu_0)>M$
%	and all $\epsilon>0$, there exists $\nu\in\mathcal{P}$ such that
%	$W_{2}(\nu,\nu_{0})=\mathcal{O}(\epsilon)$ and
%	\[
%	\F(\nu)\leq \F(\nu_{0})-\epsilon(\F(\nu_{0})-M).
%	\]
%\end{enumerate}
%%
%%\begin{remark}
%The result in \Cref{prop:almost_convex_optimization} means that it is possible to optimize the cost function $\nu\mapsto \F(\nu)$
%on $\mathcal{P}$ as long as the barrier $\inf_{\nu\in\mathcal{P}}\F(\nu)+K$
%is not reached. However, this barrier remains large, since it depends on particular of the diameter of $\X$. A possible direction to refine the statement in \cref{prop:almost_convex_optimization} would be to directly leverage the local convexity of $\F$ to get a better description of the loss landscape
% %tighter inequality in \cref{eq:integral_lambda_convexity} to get a better description of the loss landscape.\aknote{can we really?}
%
%
%%\cref{prop:almost_convex_optimization} guarantees the existence of a direction of descent that minimizes the functional $\F$ provided that the starting point $\rho_1$ has a potential greater than the barrier $K$. %, i.e:
%%\begin{align}\label{eq:barrier_condition}
%%	\F(\rho_1)> \inf_{\rho\in \mathcal{P}} \F(\rho) + K
%%\end{align}
%One natural question to ask is whether the  discretized gradient flow algorithm provides such way to reach the barrier $K$ and at what speed this happens. %This subsection will answer that question. 
%Firstly, we state few propositions that will lead us to the final result. The proofs exploit in particular the local convexity of $\F$. \aknote{we basically need Proposition 10 to state the final result but we can also not say it}
%
%
%%\begin{proposition}\label{prop:decreasing_functional}
%%	Under \cref{assump:bounded_trace,assump:bounded_hessian}, the following inequality holds:
%%	\begin{align*}
%%	\F(\nu_{n+1})-\F(\nu_n)\leq -\gamma (1-\frac{\gamma}{2}L )\int \Vert \phi_n(X)\Vert^2 d\nu_n
%%	\end{align*}
%%\end{proposition}
%
%\begin{proposition}\label{prop:evi}
%	Consider the sequence of distributions $\nu_m$ obtained from \cref{eq:euler_scheme}. If $\gamma \leq 1/L$, then
%	\begin{align}
%2\gamma(\F(\nu_{n+1})-\F(\mu))
%\leq 
%W_2^2(\nu_n,\mu)-W_2^2(\nu_{n+1},\mu)-2\gamma K(\rho^n).
%\label{eq:evi}
%\end{align}
%where $(\rho^n_t)_{0\leq t \leq 1}$ is a constant-speed geodesic from $\nu_n$ to $\mu$ and $K(\rho^n):=\int_0^1 \Lambda(\rho^n_s,\dot{\rho}^n_s)(1-s)ds$.
%\end{proposition}
%
%%\begin{theorem}\label{th:rates_mmd}
%%	Consider the sequence of distributions $\nu_n$ obtained from \cref{eq:euler_scheme} and let us denote $\bar{K}$ the average value of $(K(\rho^j))_{0\leq j \leq n}$ over iterations from $0$ to $n$. \aknote{show that $\bar{K}$ is bounded} If $\gamma \leq 1/L$, then
%%	%\begin{align}
%%%\F(\bar{\nu}_{n})-\F(\mu)\leq  \frac{W_2^2(\nu_0,\mu)}{2 \gamma n} -\bar{K}
%%%\end{align}
%%%where $\bar{\nu}=\frac{1}{N}\sum_{n=1}^N \nu_n$. Moreover, 
%%\begin{align}
%%\F(\nu_n)-\F(\mu)\leq  \frac{W_2^2(\nu_0,\mu)}{2 \gamma n} -\bar{K}.
%%\end{align}
%%\end{theorem}
%
%The Euler scheme of the MMD flow is thus guaranteed to converge up to a barrier. In practice, we will see in the experiments \cref{sec:discretized_flow} that the algorithm can be stuck in local minimas on simple examples. We point out that the results in the latter section, concerning the rates of convergence, remains in continuous time, but \cref{th:rates_mmd} provides a stronger result. \aknote{not well said but let's tell a story for now}In the next section, we propose a modified algorithm which will be guaranteed to converge to a global optimum.
%
%



\subsection{Lojasiewicz type inequality - Convergence of the continuous flow}\label{sec:Lojasiewicz_inequality}

Here we would like to derive an inequality between the time derivative of the Lyapunov functional $\mathcal{F}$ along its gradient flow $t\mapsto \rho_t$. For this purpose we first introduce the weighted negative Sobolev distance \manote{cite villani and peyre and Mroueh}:
\begin{align}\label{eq:neg_sobolev}
	\Vert \nu - \mu \Vert_{\dot{H}^{-1}(\nu)} = \sup_{\substack{ f\in W_0^{1,2}(\nu), \; \nu(\Vert \nabla f \Vert^2) \leq 1 }} \vert \nu(f)-\mu(f)\vert 
\end{align}
Where $W_0^{1,2}(\nu)$ is the space $1$ order Sobolev functions with functions vanishing at the boundary of the domain.
The distance defined in \cref{eq:neg_sobolev} plays a fundamental role in dynamic optimal transport as it linearizes the $W_2$ distance when $\mu$ is arbitrarily close to $\nu$. It can also be seen as the minimum kinetic energy needed to advect the mass $\nu$ to $\mu$. However, this quantity might be infinite \manote{say exactly when it is finite} and one of the key problems would be to control its value during the evolution of the flow. More precisely we will rely on the following statement:
\begin{align}\label{eq:bounded_neg_sobolev}
	\Vert \nu_t  - \mu \Vert_{\dot{H}^{-1}(\nu_t)} \leq C \qquad \forall t\geq 0.
\end{align} 
where $\nu_t$ is defined by the gradient flow and $\mu$ is the target distribution. When \cref{eq:bounded_neg_sobolev}  holds, we have the following proposition:
\begin{proposition}\label{prop:lojasiewicz}
	When \cref{eq:bounded_neg_sobolev} holds, the following inequality is then satisfied at all times:
	\begin{align}\label{eq:PL_type_inequality}
		\Vert \nabla f_t \Vert_{L_2(\nu_t)} \geq \frac{1}{C} \Vert f_t \Vert^2_{\mathcal{H}} \qquad \forall t\geq 0.
	\end{align}
	Then $t\mapsto \mathcal{F}(\nu_t)$ converges to $0$ with a rate of convergence given by:
	\begin{align}
	\mathcal{F}(\nu_t)\leq \frac{1}{\mathcal{F}(\nu_0)^{-1} + \frac{4t}{C}}
	\end{align}
\end{proposition}



All the difficulty is to see now when \cref{eq:bounded_neg_sobolev} holds. One possible strategy would be to start from initial $\nu_0$ such that $\Vert \nu_0  - \mu \Vert_{\dot{H}^{-1}(\nu_0)} \leq C $  for some finite positive value $C$ and then show that this property is preserved during the dynamics. It is also possible to have a time depended constant $C_t$ as long as its growth is such that:
\begin{align}
	\lim_{t\rightarrow +\infty} \int_0^t C_s^{-1}\diff s = +\infty
\end{align}
For instance $C_t$ could have up to a linear growth in time. In this case the decay of $\F(\nu_t)$ will no longer be in $\frac{1}{t}$ but only in $\frac{1}{\log(t)}$ \manote{This seems unlikely if we end up having convergence of $\nu_t$, but who nows.}.
One possible promising condition for \cref{eq:bounded_neg_sobolev} to hold would be if $\mu \ll \nu_0$ and if this property is preserved during the dynamics.


 












\subsection{A noisy update as a regularization}\label{sec:noisy_flow}

%Although the MMD flow in $W_2$ decreases in time, it can very well
% remain stuck in local minima. This can happen for instance when the negative Sobolev norm in \eqref{eq:inequality_neg_sobolev} diverges. One way to see this, is by looking at the equilibrium condition for \cref{eq:time_evolution_mmd}.
%%\asnote{I think that the same problem happens with the dynamics of SVGD. Because KSD = 0 doesn't imply p = q unless absolute continuity + other requirements} 
%Indeed $t \mapsto \F(\nu_t)$ is a non-negative decreasing function, it must therefore converge to some limit, which implies in turn that its time derivative would also converge to $0$. Assuming that $\nu_t$ also converged to some limit distribution $\nu^{*}$
%\footnote{There are cases when $\nu_t$ does not converge to any $\nu^*$. This would happen if the sequence $(\nu_t)_{t\geq 0}$ is not tight.} 
%one can show under simple regularity conditions \aknote{which ones?} that the equilibrium condition:
%\begin{align}\label{eq:equilibrium_condition}
%	\int \Vert \nabla f_{\mu,\nu^{*}}(x)\Vert^2 \diff \nu^{*}(x) =0  
%\end{align}
%must hold. If $\nu^*$ turns out to have a positive density, then $f_{\mu,\nu^{*}}(x)$ would be constant everywhere. This in turn would mean that $f_{\mu,\nu^{*}}=0$ when the RKHS does not contain constant functions, as for a gaussian kernel. Hence, $\nu^*$ would be a global optimum since $\F(\nu^{*})=0$. However, the limit distribution $\nu^*$  might be very singular, it could even be a dirac distribution. \manote{here a figure would be nice}  This suggests that the gradient flow could converge to a suboptimal solution $\nu^*$ for which \cref{eq:equilibrium_condition} is true. 

%This can happen for instance when the negative Sobolev norm in \eqref{eq:inequality_neg_sobolev} diverges. 
%\asnote{I think that the same problem happens with the dynamics of SVGD. Because KSD = 0 doesn't imply p = q unless absolute continuity + other requirements} 
%Indeed $t \mapsto \F(\nu_t)$ is a non-negative decreasing function, it must therefore converge to some limit, which implies in turn that its time derivative would also converge to $0$. Assuming that $\nu_t$ also converged to some limit distribution $\nu^{*}$
%In \cite[Proposition 3]{Mroueh:2019} such equilibria are excluded by assumption \cite[Assumption A]{Mroueh:2019} to provide convergence towards a global minimizer for the KSD. 
%In \cref{sec:Lojasiewicz_inequality} we show global convergence, under the boundedness at all times $t$ of a specific distance between $\nu_t$ and $\mu$.
%If $\nu^*$ turns out to have a positive density, then $f_{\mu,\nu^{*}}(x)$ would be constant everywhere. This in turn would mean that $f_{\mu,\nu^{*}}=0$ when the RKHS does not contain constant functions, as for a gaussian kernel. Hence, $\nu^*$ would be a global optimum since $\F(\nu^{*})=0$. 
%In \cite{mroueh2018regularized}, it is assumed that the only distribution $\nu^*$ satisfying the equilibrium condition is equal to $\mu$\aknote{true?}.  However, the limit distribution $\nu^*$  might be very singular, it could even be a dirac distribution. This suggests that the gradient flow could converge to a suboptimal solution $\nu^*$ for which
We showed in \cref{subsection:barrier_optimization} that $\F$ was a non-convex functional, then derived a condition in \cref{sec:Lojasiewicz_inequality} to reach the global optimum. Now, we would like to address the case when such condition does not necessarily hold and then provide a regularization of the gradient flow to help achieve global optimality. Our starting point will be the equilibrium condition in \cref{eq:equilibrium_condition}. If an equilibrium $\nu^*$ that satisfies \cref{eq:equilibrium_condition} happens to have a positive density, then $f_{\mu,\nu^{*}}$ would be constant everywhere. This in turn would mean that $f_{\mu,\nu^{*}}=0$ when the RKHS does not contain constant functions, as for a gaussian kernel. Hence, $\nu^*$ would be a global optimum since $\F(\nu^{*})=0$. However, the limit distribution $\nu^*$  might be very singular, it can even be a dirac distribution \cite[Theorem 6]{mei2018mean}. Although, the gradient $\nabla f_{\mu,\nu^{*}}$ is not identically $0$ in that case,  \cref{eq:equilibrium_condition} only evaluate it on the support $\nu^{*}$ on which $\nabla f_{\mu,\nu^{*}}=0$ holds. Hence a possible fix would be to make sure that such gradient is also evaluated at points outside of the support of $\nu^{*}$. 
Here, we propose to regularize the flow by injecting noise into the gradient during updates of \cref{eq:euler_scheme_particles}: %\aknote{abrupt. would be great to have a continuous analog}  
\begin{align}\label{eq:discretized_noisy_flow}
	X_{n+1} = X_{n} -\gamma \nabla f_{\mu,\nu_n}(X_n+ \beta_n U_n) \qquad n\geq 0
\end{align}
where $U_n$ is a standard gaussian variables and $\beta_n$ is the noise level at $n$. Compared to \cref{eq:euler_scheme}, here the sample is blurred first before evaluating the gradient.
Intuitively, if $\nu_n$ approaches a local optimum $\nu^{*}$, $ \nabla f_{\mu,\nu_n}$ would be small on the support of $\nu_n$ but it might be much larger outside of it, hence evaluating $\nabla f_{\mu,\nu_n}$ outside the support of $\nu_n$ might help escaping the local minimum. The stochastic process \cref{eq:discretized_noisy_flow} is different from adding a diffusion term to \cref{eq:continuity_mmd}. Indeed, in the latter case, %the update equation would be:
%\begin{align}\label{eq:diffusion}
%	X_{n+1} = X_{n} -\gamma \nabla f_{\mu,\nu_n}(X_n)+ \beta_n U_n \qquad n\geq 0.
%\end{align}
%to construct, at least formally, a modified gradient flow for which the optimality condition would guarantee reaching the global optimum.
%Ideally, we would like to obtain an optimality condition of the form
%\begin{align}\label{eq:soothed_equilibrium_condition}
%	\int \Vert \nabla f_{\mu,\nu^{*}}(x)\Vert^2 \diff (\nu^{*}\star g)(x) =0  
%\end{align}
%where $\nu^{*}\star g$ means the convolution of $\nu^*$ with a gaussian distribution $g$. The smoothing effect of convolution directly implies that $\nu^{*}\star g$ has a positive density, which falls back in the scenario where the $\nu^*$ must a global optimum.
%We consider, at least formally, the following modified equation for $\nu_t$:
%\begin{align}\label{eq:smoothed_continuity_equation_mmd}
%	\partial_t \nu_t = div((\nu_t \star g) \nabla f_{\mu,\nu_t} )
%\end{align}
%This suggests a particle equation which would be given by:
%\begin{align}\label{eq:noisy_particles}
%	\dot{X}_t = -\nabla f_{\mu,\nu_t}( X_t + W_t  )
%\end{align}
%where $(W_t)$ is a brownian motion. Furthermore, $\F(\nu_t)$ satisfies
%\begin{align}\label{eq:smoothed_decreasing_mmd}
%	\dot{\F}(\nu_t) = -\int \Vert \nabla f_{\mu,\nu_t}(x)\Vert^2 \diff (\nu_t\star g)(x)
%\end{align}
%The existence and uniqueness of a solution to \cref{eq:smoothed_continuity_equation_mmd} for a general $g$ remains an open question to our knowledge. However, we find it useful here to state \cref{eq:smoothed_continuity_equation_mmd,eq:noisy_particles,eq:smoothed_decreasing_mmd} which are the modified analogs of \manote{ref to the analogs}.
 %\cref{eq:diffusion} 
it would correspond to regularizing $\F$ using an entropic term as in \cite{mei2018mean,csimcsekli2018sliced} (see also \cref{subsec:kl_flow} about the Langevin diffusion). \cref{eq:discretized_noisy_flow} is also different from \cite{craig2016blob,carrillo2019blob} where $\F$ (and thus its associated velocity field) is regularized by convolving the interaction potential $W$ in \cref{eq:potentials} with a mollifier. However, the optimal solution of a regularized version of the functional $\F$ will be generally different from the non-regularized one, which is not desirable in our setting. 
%This is not the case for \cref{eq:discretized_noisy_flow} where the global optimum of $\F$ is a fixed point, as it will be shown in this section. 
 %As shown in \manote{add this in the appendix}, \cref{eq:discretized_noisy_flow} is associated to an augmented continuous-time dynamics  which decreases $\F$ under a condition on the noise level $\beta_k$:
In fact \cref{eq:discretized_noisy_flow} is  closely related to the \textit{continuation methods} \cite{Gulcehre:2016a,Gulcehre:2016,Chaudhari:2017}  and \textit{graduated optimization} \cite{Hazan:2015} used for non-convex optimization in Euclidian spaces, which inject noise \textit{into} the gradient as well at each iteration. %Indeed given a non-convex cost function $F$, the graduated descent would lead to updates of the form: $X_{n+1} = X_n - \gamma \nabla F(X_n+\beta U_n )$. The main difference with \cref{eq:discretized_noisy_flow} is the dependence of $f_{\mu, \nu_n}$ on $\nu_n$ which is inherently due to functional optimization.
We show in \cref{thm:convergence_noisy_gradient} that \cref{eq:discretized_noisy_flow} attains the global minimum of $\F$ provided the level of the noise is well controlled.
%\begin{proposition}\label{thm:convergence_noisy_gradient}
%	Let $(\nu_n)_{n\geq 0}$ be defined by \cref{eq:discretized_noisy_flow} with an initial $\nu_0$. Under \cref{assump:lipschitz_gradient_k,assump:Lipschitz_grad_rkhs}, and for a choice of $\beta_n$ such that:
%	\begin{equation}\label{eq:control_level_noise}
%		8\lambda^2\beta_n^2 \F(\nu_n) \leq \int \Vert \nabla f_{\mu,\nu_n}(x+\beta_n u) \Vert^2 g(u) \diff \nu_n(x)\diff u   
%	\end{equation}
%	 the following inequality holds:
%	\begin{equation}\label{eq:decreasing_loss_iterations}
%		\F(\nu_{n+1}) - \F(\nu_n  ) \leq -\frac{\gamma}{2}(1-3\gamma L)\int \Vert \nabla f_{\mu,\nu_n}(x+\beta_n u) \Vert^2 g(u) \diff\nu_n(x) \diff u.
%	\end{equation}
%$\lambda$ and $L$ are defined in \cref{assump:lipschitz_gradient_k,assump:Lipschitz_grad_rkhs} and depend only on the choice of the kernel, and $g$ is the density of the standard gaussian distribution. Moreover, if  $\sum_{i=0}^n \beta_i^2 \rightarrow \infty $ then:
%\begin{equation}
%\F(\nu_n)\leq \F(\nu_0) e^{-4\lambda^2\gamma(1-3\gamma L)\sum_{i=0}^n \beta^2_i}
%\end{equation}
%\end{proposition}
\begin{proposition}\label{thm:convergence_noisy_gradient}
	Let $(\nu_n)_{n\in \mathbb{N}}$ be defined by \cref{eq:discretized_noisy_flow} with an initial $\nu_0$. Denote $\mathcal{D}_{\beta_n}(\nu_n)=\mathbb{E}_{x\sim \nu_n, u\sim g}[\Vert \nabla f_{\mu,\nu_n}(x+\beta_n u) \Vert^2]$ with $g$ the density of the standard gaussian distribution.	Under \cref{assump:lipschitz_gradient_k,assump:Lipschitz_grad_rkhs}, and for a choice of $\beta_n$ such that:
	\begin{equation}\label{eq:control_level_noise}
	8\lambda^2\beta_n^2 \F(\nu_n) \leq \mathcal{D}_{\beta_n}(\nu_n),
	\end{equation}
	\begin{flalign}\label{eq:decreasing_loss_iterations}
\text{The following inequality holds: }\quad\quad	\F(\nu_{n+1}) - \F(\nu_n  ) \leq -\frac{\gamma}{2}(1-3\gamma L)\mathcal{D}_{\beta_n}(\nu_n), &&
	\end{flalign}
	where $\lambda$ and $L$ are defined in \cref{assump:lipschitz_gradient_k,assump:Lipschitz_grad_rkhs} and depend only on the choice of the kernel. Moreover, if  $\sum_{i=0}^n \beta_i^2 \rightarrow \infty $ then:
	\begin{equation}
	\F(\nu_n)\leq \F(\nu_0) e^{-4\lambda^2\gamma(1-3\gamma L)\sum_{i=0}^n \beta^2_i}
	\end{equation}
\end{proposition}



\begin{remark}
	  %This allows the algorithm to use non-local information on the loss landscape by probing the gradient in regions outside of the support of $\nu_k$. Thus this algorithm could potentially escape local optima. 
	At each iteration, the level of the noise needs to be adjusted such that the gradient is not too much blurred. This ensures that each step would decrease the loss functional. However, $\beta_n$ does not need to decrease at each iteration: it could increase adaptively whenever needed, i.e. when  the sequence gets closer to a local optimum, it is helpful to increase the level of the noise to probe the gradient in regions where its value is not flat.
	Notice that for $\beta_n = 0$  in \cref{eq:decreasing_loss_iterations} , we recover a similar bound as in \cref{prop:decreasing_functional}. %However, the interesting cases are when $\beta_n>0$.
	%The second crucial point, is the dependence of the level of the noise on the value of the loss functional itself in \cref{eq:control_level_noise}. This allows some tolerance for high levels of noise when the loss functional is already small. In fact this precise condition provides a Lojasiewicz type inequality for free, which will then be used in  to provide convergence rates in  \cref{sec:Lojasiewicz_inequality}.
 \end{remark}
%The natural question is whether \cref{eq:discretized_noisy_flow} converges towards to global optimum of $\F$. The answer will depend on how much noise is allowed to be injected while still decreasing $\F$. The higher the $\beta_n$ is, the faster it will converge. This is made more precise in \cref{thm:convergence_noisy_gradient}: 
 %\begin{theorem}\label{thm:convergence_noisy_gradient}
 %Under \cref{assump:lipschitz_gradient_k,assump:Lipschitz_grad_rkhs} and if \cref{eq:control_level_noise} is satisfied for a sequence $(\beta_n)_{n\geq0}$ such that $\sum_{i=0}^n \beta_i^2 \rightarrow \infty $ then:
 %\begin{align}
 %	\F(\nu_n)\leq \F(\nu_0) e^{-4\lambda^2\gamma(1-3\gamma L)\sum_{i=0}^n \beta^2_i}
 %\end{align}
 %\end{theorem}
% A proof of \cref{thm:convergence_noisy_gradient} is provided in
A proof of \cref{thm:convergence_noisy_gradient} is provided in \cref{proof:thm:convergence_noisy_gradient} and relies on \cref{eq:control_level_noise} to get a Lojasiewicz inequality which then controls the decay of $\F(\nu_n)$. A particular case when $\sum_{i=0}^n \beta_i^2 \rightarrow \infty$ holds is when $\beta_n$ decays as $1/\sqrt{n}$ while still having \cref{eq:control_level_noise}. In this case, one gets polynomial convergence.
In \cref{sec:sample_based} we provide a practical algorithm for \cref{eq:discretized_noisy_flow} which involves a discretization in space.
 
 
 



%\subsection{MMD flows in the literature}

%\begin{remark}
%	We point out here that algorithm~\cref{eq:sample_based_process} is different from the descent proposed by \cite{mroueh2018regularized}. 
%\end{remark}

%\begin{remark}
%	Birth-Death Dynamics to improve convergence (see \cite{rotskoff2019global}).
%\end{remark}

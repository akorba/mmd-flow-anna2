

\section{MMD gradient flows}\label{sec:mmd_flow}

\subsection{MMD gradient flow}

We will consider a flow $(\rho_t)_{t>0}$ as described in \cref{sec:gradient_flows_functionals} and denote $f_t= \int k(.,z)\diff \mu - \int k(.,z)\diff \rho_t$. In this case:
\begin{equation}
\F(\rho_t)=\frac{1}{2}\|f_t\|^2_{\kH}
%&= \E_{\rho_t \otimes \rho_t}[k(X,X')]+\E_{\pi \otimes \pi}[k(Y,Y')] - 2\E_{\rho_t \otimes \pi}[k(X,Y)]
\end{equation} 

We define the potential energy (also called confinement energy) $V$ and interaction energy $W$ as follows:
\begin{align*}
	V(x)=-\int  k(x,x')\mu(x')\text{,} \quad
W(x,x')=\frac{1}{2}k(x,x')
\end{align*}
We have $1/2MMD^2(\rho,\mu)=C+ \int V(x) \rho(x)dx + \int W(x,x')\rho(x)\rho(x')$, where $C=1/2\E_{\mu\otimes \mu}[k(x,x')]$. $MMD^2$ can thus be written as a \textit{Lyapunov functional} (or "free energy" or "entropy") $\F$ as in \cref{eq:lyapunov}.

\begin{proposition}\label{prop:mmd_flow}
 The velocity in \cref{eq:continuity_equation1} is given by $\nabla \frac{\partial{\F}}{\partial{\rho_t}}=\nabla f_t$ and the dissipation of MMD can be written:  
	\begin{equation}
	\frac{d MMD^2(\rho_t, \mu)}{dt}=-\E_{X \sim \rho_t}[\|\nabla f_t(X)\|^2]
	\end{equation}
	where $\nabla f_t(z)= \int \nabla_{z}k(.,z) d\mu -  \int \nabla_{z}k(.,z) d\rho_t$.
\end{proposition}

\begin{remark}
	If the functional $\F$ was the KL divergence and $\rho_t$ a weak solution of the Fokker-Planck equation \cref{eq:Fokker-Planck}, we would obtain the following dissipation (see \cite{wibisono2018sampling}):
	\begin{align}\label{eq:decreasing_mmd}
	\frac{d KL(\rho_t, \mu)}{dt}=-\E_{X \sim \rho_t}[\|\nabla log(\frac{\rho_t}{\mu}(X))\|^2]
	\end{align}
\end{remark}


As explained in \cref{sec:gradient_flows_functionals} and according to \cref{prop:mmd_flow}, the gradient flow of the MMD can be written:
\begin{equation}\label{eq:continuity_equation_mmd}
\frac{\partial \rho_t}{\partial t}= div(\rho_t  \nabla f_t)
\end{equation}
The stochastic process whose distribution satisfies \cref{eq:continuity_equation_mmd} can thus be written (see \cref{sec:ito_stochastic}):
\begin{equation}\label{eq:stochastic_process}
dX_t=-\nabla f_t(X_t) = - (\nabla V (X_t) + \nabla W * \rho_t(X_t))
\end{equation}
%Equation \cref{eq:stochastic_process} can be interpreted as the position $X_t$ of a particle at time $t > 0$.
\aknote{The following is based on the formalism and some results of \cite{jourdain2007nonlinear}} Equation \cref{eq:stochastic_process}, which can be interpreted as the position $X_t$ of a particle at time $t > 0$, can be written as a Mac-Kean Vlasov model\aknote{reference}, a particular kind of SDE driven by a Levy process:
\begin{align}\label{eq:theoretical_process}
&X_t=X_{0}+\int_{0}^t \sigma(X_s, \rho_s, \mu)ds \quad \text{for t in [0,T]}\\
&\forall s \in [0,T]\;,\quad \rho_s \text{ denotes the probability distribution of } X_s
\end{align}
with $\sigma(X_s, \rho_s, \mu)=-\nabla f_t(X_s)=\int \nabla_{X_s}k(.,X_s) d\rho_t -  \int \nabla_{X_s}k(.,X_s) d\mu$. Notice that $\sigma$ is bounded \aknote{true?} and Lipschitz continuous in its second and third variable and bounded.\aknote{investigate conditions on the kernel for convergence, uniqueness. e.g. linear growth of the coefficient sigma? or does it relate to lambda convexity as santambrogio says?}

\begin{remark}
	Consider a family of particles such that its density satisfy Equation\cref{eq:continuity_equation}. Both KL and MMD have a non-zero potential energy $V$ which drive these particles to the target distribution $\mu$. While he entropy function $U$ in KL prevents the particle from "crashing" onto the mode of $\mu$, this role could be played by the interaction energy $W$ for MMD. Indeed, when $W$ is convex, this gives raise to a general
	aggregation behavior of the particles, while when it is not, the particles would push each other apart.\aknote{to check, ref malrieu?}
\end{remark}


\subsection{Noisy MMD flow}

Although the Wasserstein flow of the MMD decreases the MMD in time, it can very well remain stuck in local minima. One way to see how this could happen, at least formally, is by looking at \cref{eq:decreasing_mmd} at equilibrium. Indeed $\F(n_t)$ is a non-negative decreasing function of time, it must therefore converge to some limit, this implies that its time derivative would also converge to $0$. Assuming that $\nu_t$ also converged to some limit distribution $\nu^{*}$ one can show that under simple regularity conditions that the equilibrium condition
\begin{align}\label{eq:equilibrium_condition}
	\int \Vert \nabla f_{\mu,\nu^{*}}(x)\Vert^2 \diff \nu^{*}(x) =0  
\end{align} 
must hold. If $\nu^*$ had a positive density then this would imply that $f_{\mu,\nu^{*}}(x)$ is constant everywhere. If the set of functions spanned by the RKHS associated to the MMD do not include constant functions, then it must hold that $f_{\mu,\nu^{*}}=0$ which in turn means that $MMD(\mu,\nu^{*})=0$, hence $\nu^*$ would be a global solution. However, the limit distribution $\nu^*$  might be very singular, it could even be a dirac distribution. In that case, the optimality condition \cref{eq:equilibrium_condition} is of little use. Moreover, it suggests that the gradient flow could converge to some suboptimal configuration as the gradient is only evaluated near the support of $\nu^*$.
Since \cref{eq:equilibrium_condition} seems to be the main obstruction to reach global optimality, we propose to construct, at least formally, a modified gradient flow for which the optimality condition would guarantee reaching the global optimum.
Ideally, we would like to obtain an optimality condition of the form
\begin{align}\label{eq:soothed_equilibrium_condition}
	\int \Vert \nabla f_{\mu,\nu^{*}}(x)\Vert^2 \diff (\nu^{*}\star g)(x) =0  
\end{align}
where $\nu^{*}\star g$ means the convolution of $\nu^*$ with a gaussian distribution $g$. The smoothing effect of convolution directly implies that $\nu^{*}\star g$ has a positive density, which falls back in the scenario where the $\nu^*$ must a global optimum.
We consider, at least formally, the following modified equation for $\nu_t$:
\begin{align}\label{eq:smoothed_continuity_equation_mmd}
	\partial_t \nu_t = div((\nu_t \star g) \nabla f_{\mu,\nu_t} )
\end{align}
This suggests a particle equation which would be given by:
\begin{align}\label{eq:noisy_particles}
	\dot{X}_t = -\nabla f_{\mu,\nu_t}( X_t + W_t  )
\end{align}
where $(W_t)$ is a brownian motion. Furthermore, $\F(\nu_t)$ satisfies
\begin{align}\label{eq:smoothed_decreasing_mmd}
	\dot{\F}(\nu_t) = -\int \Vert \nabla f_{\mu,\nu_t}(x)\Vert^2 \diff (\nu_t\star g)(x)
\end{align}
The existence and uniqueness of a solution to \cref{eq:smoothed_continuity_equation_mmd} for a general $g$ remains an open question to our knowledge. However, we find it useful here to state \cref{eq:smoothed_continuity_equation_mmd,eq:noisy_particles,eq:smoothed_decreasing_mmd} which are the modified analogs of \manote{ref to the analogs}.

We provide now the time discretized version of such flow which is well defined at every iteration and for which we prove the convergence towards the global optimum \manote{ref to section}.
\manote{so all of this is not consistent with the rest anymore!}
For a time discretization step $\gamma>0$ we consider the following algorithm:
\begin{align}\label{eq:discretized_noisy_flow}
	X_{k+1} = X_{k} + U_k -\gamma \nabla f_{\mu,\nu_k}(X_k+U_k) \qquad k\geq 0
\end{align}

Here $U_k$ is a sample from the gaussian $g$ noise while $X_k$ is a sample at iteration $k$. Unlike the original flow where the gradient is evaluated at the current sample, here the sample is blurred first before evaluating the gradient. We would like to emphasize that this algorithm is different from adding noise to the samples themselves which would correspond to adding a diffusion term in \cref{eq:noisy_particles}. We show that \cref{eq:discretized_noisy_flow} decreases the loss functional at every iteration in the following proposition:
\begin{proposition}\label{prop:decreasing_loss_iterations}
	Let $(\nu_k)_{k\geq 0}$ be sequence of distributions defined by \cref{eq:discretized_noisy_flow} with an initial condition $\nu_0$, then the following inequality holds:
	\begin{align}\label{eq:decreasing_loss_iterations}
		\F(\nu_{k+1}) - \F(\nu_k \star g ) \leq -\gamma(1-\gamma L)\int \Vert \nabla f_{\mu,\nu_k}(x) \Vert^2 \diff (\nu_k\star g)(x)
	\end{align}
\end{proposition}




 


%\section{Theoretical properties of the MMD flow}\label{sec:theory}



%\subsection{Lambda displacement convexity of the MMD}
\subsection{Optimization in a ($W_2$) non-convex setting}
\label{subsection:barrier_optimization}
One important criterion to characterize the convergence of the Wasserstein gradient flow of a functional $\F$ is the \textit{displacement convexity} of such a functional. The latter states that $t\mapsto \F(\nu_t)$ is a convex function whenever $t\mapsto\nu_t$ is a path of minimal length from two distributions $\mu$ and $\nu$ as explained in \cref{def:displacement_convexity} (see also \cite[Definition 16.5]{Villani:2004}). %The notion of path of minimal length depends on the choice of the metric. 
Such paths are called  \textit{constant speed displacement geodesics} when additionally their velocity vector has a constant norm. We refer to \cite{Bottou:2017} for a more in-depth discussion.
\textit{Displacement convexity} should not be confused with \textit{mixture convexity} which corresponds to the usual notion of convexity. As a matter of fact, $\F$ is \textit{mixture convex} in that it satisfies: $\F(t\nu +(1-t)\nu')\leq t\F(\nu)+(1-t)\F(\nu')$ for all $t\in [0,1]$ and $\nu,\nu'\in\mathcal{P}_2(\X)$ (see \cref{lem:mixture_convexity}). Unfortunately, $\F$ is not \textit{displacement convex}. Instead, $\F$ only satisfies a weaker notion of displacement convexity called $\Lambda$-displacement convexity (\cite[Definition 16.5]{Villani:2009} and  \cref{def:lambda-convexity}): 
%This implies that the gradient flow of $\F$ might not converge to the optimal solution. 
%This could happen if for instance \cref{eq:Lojasiewicz_inequality} doesn't hold. 
%It can be shown, however, that $\F$ is guaranteed to reach some barrier. This is a consequence of the $\Lambda$-displacement convexity of $\F$ which is 
\begin{proposition}
	\label{prop:lambda_convexity} Suppose $\sup_{(x,y) \in \X, (i,j) \in \llbracket 1, d \rrbracket} \partial_{x_{i}}\partial_{x_{j}}\partial_{x'_{i}}\partial_{x_{j}'}k(x,y)\le B$  holds for some $B \in \R^+$. Then for all $\nu, \nu'\in \mathcal{P}_2(\X)$ and any \textit{displacement geodesic} $(\nu_t)_{1\leq t\leq 1}$ from $\nu$ to $\nu'$ with velocity vectors $(V_t)_{0\leq t\leq 1}$ the following holds:
	\begin{equation}
	\F(\nu_{t})\leq(1-t)\F(\nu)+t\F(\nu')-\int_0^1 \Lambda(\nu_s, V_s ) G(s,t)\diff s
	\end{equation}
	where $s,t\mapsto G(s,t)$ is the one-dimensional Green function: $G(s, t) =  s(1-t) \mathbbm{1}\{s\leq t\}+ t(1-s) \mathbbm{1}\{s\geq t\}$ and $\Lambda$ is defined for any pair $(\nu,V)\in \mathcal{P}_2(\X)\times L_2(\rho)$  by:
	\begin{align}\label{eq:lambda}
		\Lambda(\nu,V) = \Vert \int V(x).\nabla_x k(x,.) \diff \nu(x) \Vert^2_{\mathcal{H}} - B\F(\nu)^{\frac{1}{2}}  \int \Vert  V(x) \Vert^2 \diff \nu(x) 
	\end{align}
	\end{proposition}
\cref{prop:lambda_convexity} can be obtained by computing the second time-derivative of $\F(\rho_t)$ which is then lower-bounded by $\Lambda(\rho_t,V_t)$ (see \cref{proof:prop:lambda_convexity} for a full proof).
In \cref{eq:lambda}, the map $\Lambda$ is a difference of two non-negative terms thus $\int_0^1 \Lambda(\rho_s, V_s ) G(s,t)\diff s$ can become negative, hence displacement convexity doesn't hold in general. However, it is still possible to provide an upper-bound on the asymptotic value of $\F(\nu_n)$ when $\nu_n$ are obtained using \cref{eq:euler_scheme}. Such upper-bound would depend on a scalar $ K(\rho^n) :=  \int_0^1\Lambda(\rho_s^n,V_s^n)(1-s)\diff s$ where $(\rho_s^n)_{0\leq s\leq 1}$ is a \textit{constant speed displacement geodesic} from $\nu_n$ to the optimal value $\mu$ with velocity vectors $(V_s^n)_{0\leq s\leq 1}$. \cref{th:rates_mmd} provides such bound:
\begin{theorem}\label{th:rates_mmd}
	Consider the sequence of distributions $\nu_n$ obtained from \cref{eq:euler_scheme} and let us denote $\bar{K}$ the average value of $(K(\rho^j))_{0\leq j \leq n}$ over iterations from $0$ to $n$. \aknote{show that $\bar{K}$ is bounded!!} If $\gamma \leq 1/L$, then
\begin{align}
\F(\nu_n)\leq  \frac{W_2^2(\nu_0,\mu)}{2 \gamma n} -\bar{K}.
\end{align}
\end{theorem}
\cref{th:rates_mmd} is obtained using techniques from optimal transport and optimization. It relies on \cref{prop:lambda_convexity} and \cref{prop:decreasing_functional} to prove an \textit{extended variational inequality} (see \cref{prop:evi}) and concludes using a suitable Lyapunov function, a full proof is given in\cref{proof:th:rates_mmd}.
When $\bar{K}$ is non-negative, one recovers the usual convergence rate as $O(\frac{1}{n})$ for the gradient descent algorithm. However, $\bar{K}$ can be negative in general and would therefore act as a barrier on the optimal value that $\F(\nu_n)$ can achieve. In that sense, the above result is similar to \cite[Theorem 6.9]{Bottou:2017}. In fact, \cref{th:rates_mmd} provides only a loose bound. In \cref{sec:Lojasiewicz_inequality} we show global convergence when some energy functional remains controlled.
%However, it will be sufficient to show that the flow of $\F$ can decrease until it reaches a barrier. %The size of the barrier depends on a relaxed notion of convexity called $\Lambda$-displacement convexity:

%It can be shown that $\F$ is $\text{\ensuremath{\Lambda}}$-displacement convex under mild assumptions on the kernel $k$:
%\begin{proposition}
%\label{prop:lambda_convexity} Suppose \cref{assump:bounded_fourth_oder} is satisfied for some $\lambda \in \R^+$. Then $\F$ is $\text{\ensuremath{\Lambda}}$-displacement convex with $  \Lambda(\rho,v) = -\lambda \mathcal{\rho}^{\frac{1}{2}} \int \Vert v(x) \Vert^2 \diff\rho(x)$.
%Moreover, for any displacement geodesic $\rho_t$ between two distributions $\nu$ and $\nu'$ it holds 
%\begin{align}
%	\bar{\F}(\rho_t) \geq -\lambda \F{\rho_t}^{\frac{1}{2}} W_2^2(\nu,\nu')
%\end{align}
%\end{proposition}


%%Unlike mixture convexity, displacement convexity is compatible with the $W_2$ metric and is therefore the natural notion to use for characterizing convergence of gradient flows in the $W_2$ metric.
%Although mixture convexity holds for $\F$ (see \cref{lem:mixture_convexity}), this property is less critical for characterizing convergence of gradient flows in the $W_2$ metric. On the other hand, displacement convexity is compatible with the $W_2$ metric \cite{Bottou:2017} and is therefore the natural notion to use in our setting. Unfortunately, $\F$ fails to be displacement convex in general. Instead we will show that $\F$ satisfies some weaker notion of convexity called $\Lambda$-displacement convexity:
%
%\begin{definition}\label{def:lambda-convexity}
%	We say that a functional $\nu\mapsto\mathcal{F}(\nu)$ is $\Lambda$-convex
%	if for any $\nu$ and $\mu$ and a minimizing geodesic $\text{\ensuremath{\nu_{t}}}$
%	between $\nu$ and $\mu$ with velocity vector field $v_{t}$, i.e:
%	$\partial_{t}\nu_{t}+div(\nu_{t}v_{t})=0;\nu_{0}=\nu;\nu_{1}=\mu;$
%	the following holds:
%	\begin{equation}\label{eq:lambda_displacement_convex}
%	\frac{d^{2}\mathcal{F}(\nu_{t})}{dt^{2}}\geq\Lambda(\nu_{t},v_{t})\qquad\forall\; t\in[0,1].
%	\end{equation}
%	where $(\nu,v)\mapsto\Lambda(\nu,v)$
%	is a function that defines for each $\nu \in \mathcal{P}(\X)$
%	a quadratic form on the set of square integrable vectors valued functions
%	$v$ , i.e: $v\in L_{2}(\mathbb{R}^{d},\mathbb{R}^{d},\nu)$. We
%	further assume that $\inf_{\nu,v}\Lambda(\nu,v)/\Vert v\Vert_{L_{2}(\nu)}^{2}>-\infty$. 
%	Also, the following holds:
%	\begin{equation}\label{eq:integral_lambda_convexity}
%	\F(\nu_{t})\leq(1-t)\F(\nu_{0})+t\F(\nu_{1})-\int_{0}^{1}\Lambda(\nu_{s},v_{s})G(s,t)ds
%	\end{equation}
%	where $G(s,t)=s(1-t) \mathbb{I}\{s\leq t\}
%	+t(1-s) \mathbb{I}\{s\geq t\}$.
%\end{definition}
%
%%Then, to show the $\Lambda$-convexity of the functional defined in \cref{sec:gradient_flow} we first make the following assumptions on the kernel:
%%\begin{assumplist} 
%%	\item \label{assump:bounded_trace} $ \vert \sum_{1\leq i\leq d} \partial_i\partial_ik(x,x) \vert\leq \frac{L}{3}  $ for all $x\in \mathbb{R}^d$.
%%	\item \label{assump:bounded_hessian} $\Vert H_xk(x,y) \Vert_{op} \leq \frac{L}{3}$ for all $x,y\in \mathbb{R}^d$, where $H_xk(x,y)$ is the hessian of $x\mapsto k(x,y)$ and $\Vert.\Vert_{op}$ is the operator norm.
%%	\item \label{assump:bounded_fourth_oder} $\Vert Dk(x,y) \Vert\leq \lambda  $ for all $x,y\in \mathbb{R}$, where $Dk(x,y)$ is an $\mathbb{R}^{d^2}\times \mathbb{R}^{d^2}$ matrix with entries given by $\partial_{x_{i}}\partial_{x_{j}}\partial_{x'_{i}}\partial_{x_{j}'}k(x,y)$.
%%\end{assumplist}
%The next proposition states that the functional defined in \cref{sec:gradient_flow} is $\Lambda$-displacement convex and provide and explicit expression for the functional $\Lambda$. Some additional mild assumptions on the derivative of the kernel are also needed but deferred to the appendix for presentation purpose.\aknote{to do!!}
%
%\begin{proposition}
%	\label{prop:lambda_convexity} Suppose $\sup_{x,y} \partial_{x_{i}}\partial_{x_{j}}\partial_{x'_{i}}\partial_{x_{j}'}k(x,y)\le \lambda$ is satisfied for some $\lambda \in \R^+$. The functional $\nu\mapsto \F(\nu)$ is $\text{\ensuremath{\Lambda}}$-convex
%	with $\Lambda$ given by:
%	\begin{equation}
%	\Lambda(\nu,v)=\langle v,(C_{\nu}-\lambda \F(\nu)^{\frac{1}{2}}I)v\rangle_{L_{2}(\nu)}\label{eq:Lambda}
%	\end{equation}
%	where $C_{\nu}$ is the (positive) operator defined by $(C_{\nu}v)(x)=\int\nabla_{x}\nabla_{x'}k(x,x')v(x')d\nu(x')$ for any $x \in \X$.
%	%\begin{align}\label{eq:positive_operator_C}
%	%(C_{\nu}v)(x)=\int\nabla_{x}\nabla_{x'}k(x,x')v(x')d\nu(x')
%	%\end{align}
%\end{proposition}
%%
%%


%\begin{proposition}
%	\label{prop:loser_bound}Assume the distributions are supported on
%	$\mathcal{X}$ and the kernel is bounded, i.e: $\sup_{x,y\in\mathcal{X}}\vert k(x,y)\vert<\infty$.
%	Then the following holds:
%	\begin{equation}
%	\F(\nu_{t})\leq(1-t)\F(\nu_{0})+t\F(\nu_{1})+t(1-t)K
%	\end{equation}
%	where $K$ is a constant depending on $\X$ and the kernel $k$ in $\F$.
%\end{proposition}
%
%

%\cref{prop:loser_bound}, is a loose bound and does not account for the local
%convexity of the MMD (see \cref{subsec:lambda_convexity} for more details on the convexity properties of this flow). However, it allows to state the following result,
%which is inspired from (\cite{Bottou:2017}, Theorem 6.3) but generalizes
%it to the case of 'almost convex' functionals.
%\begin{proposition}
%	\label{prop:almost_convex_optimization}
%	(Almost convex optimization). Let $\mathcal{P}$ be a closed subset
%	of $\mathcal{P}(\mathcal{X})$ which is displacement convex\aknote{weird for a set to be displacement convex? it was for functionals}. Then
%	for all $M>\inf_{\nu\in\mathcal{P}}\F(\nu)+K$, the following
%	holds:
%\end{proposition}
%\begin{enumerate}
%	\item The level set $L(\mathcal{P},M)=\{\nu\in\mathcal{P}:\F(\nu)\leq M\}$
%	is connected
%	\item For all $\nu_{0}\in\mathcal{P}$ such that $\F(\nu_0)>M$
%	and all $\epsilon>0$, there exists $\nu\in\mathcal{P}$ such that
%	$W_{2}(\nu,\nu_{0})=\mathcal{O}(\epsilon)$ and
%	\[
%	\F(\nu)\leq \F(\nu_{0})-\epsilon(\F(\nu_{0})-M).
%	\]
%\end{enumerate}
%%
%%\begin{remark}
%The result in \Cref{prop:almost_convex_optimization} means that it is possible to optimize the cost function $\nu\mapsto \F(\nu)$
%on $\mathcal{P}$ as long as the barrier $\inf_{\nu\in\mathcal{P}}\F(\nu)+K$
%is not reached. However, this barrier remains large, since it depends on particular of the diameter of $\X$. A possible direction to refine the statement in \cref{prop:almost_convex_optimization} would be to directly leverage the local convexity of $\F$ to get a better description of the loss landscape
% %tighter inequality in \cref{eq:integral_lambda_convexity} to get a better description of the loss landscape.\aknote{can we really?}
%
%
%%\cref{prop:almost_convex_optimization} guarantees the existence of a direction of descent that minimizes the functional $\F$ provided that the starting point $\rho_1$ has a potential greater than the barrier $K$. %, i.e:
%%\begin{align}\label{eq:barrier_condition}
%%	\F(\rho_1)> \inf_{\rho\in \mathcal{P}} \F(\rho) + K
%%\end{align}
%One natural question to ask is whether the  discretized gradient flow algorithm provides such way to reach the barrier $K$ and at what speed this happens. %This subsection will answer that question. 
%Firstly, we state few propositions that will lead us to the final result. The proofs exploit in particular the local convexity of $\F$. \aknote{we basically need Proposition 10 to state the final result but we can also not say it}
%
%
%%\begin{proposition}\label{prop:decreasing_functional}
%%	Under \cref{assump:bounded_trace,assump:bounded_hessian}, the following inequality holds:
%%	\begin{align*}
%%	\F(\nu_{n+1})-\F(\nu_n)\leq -\gamma (1-\frac{\gamma}{2}L )\int \Vert \phi_n(X)\Vert^2 d\nu_n
%%	\end{align*}
%%\end{proposition}
%
%\begin{proposition}\label{prop:evi}
%	Consider the sequence of distributions $\nu_m$ obtained from \cref{eq:euler_scheme}. If $\gamma \leq 1/L$, then
%	\begin{align}
%2\gamma(\F(\nu_{n+1})-\F(\mu))
%\leq 
%W_2^2(\nu_n,\mu)-W_2^2(\nu_{n+1},\mu)-2\gamma K(\rho^n).
%\label{eq:evi}
%\end{align}
%where $(\rho^n_t)_{0\leq t \leq 1}$ is a constant-speed geodesic from $\nu_n$ to $\mu$ and $K(\rho^n):=\int_0^1 \Lambda(\rho^n_s,\dot{\rho}^n_s)(1-s)ds$.
%\end{proposition}
%
%%\begin{theorem}\label{th:rates_mmd}
%%	Consider the sequence of distributions $\nu_n$ obtained from \cref{eq:euler_scheme} and let us denote $\bar{K}$ the average value of $(K(\rho^j))_{0\leq j \leq n}$ over iterations from $0$ to $n$. \aknote{show that $\bar{K}$ is bounded} If $\gamma \leq 1/L$, then
%%	%\begin{align}
%%%\F(\bar{\nu}_{n})-\F(\mu)\leq  \frac{W_2^2(\nu_0,\mu)}{2 \gamma n} -\bar{K}
%%%\end{align}
%%%where $\bar{\nu}=\frac{1}{N}\sum_{n=1}^N \nu_n$. Moreover, 
%%\begin{align}
%%\F(\nu_n)-\F(\mu)\leq  \frac{W_2^2(\nu_0,\mu)}{2 \gamma n} -\bar{K}.
%%\end{align}
%%\end{theorem}
%
%The Euler scheme of the MMD flow is thus guaranteed to converge up to a barrier. In practice, we will see in the experiments \cref{sec:discretized_flow} that the algorithm can be stuck in local minimas on simple examples. We point out that the results in the latter section, concerning the rates of convergence, remains in continuous time, but \cref{th:rates_mmd} provides a stronger result. \aknote{not well said but let's tell a story for now}In the next section, we propose a modified algorithm which will be guaranteed to converge to a global optimum.
%
%



\subsection{Lojasiewicz type inequality - Convergence of the continuous flow}\label{sec:Lojasiewicz_inequality}

Here we would like to derive an inequality between the time derivative of the Lyapunov functional $\mathcal{F}$ along its gradient flow $t\mapsto \rho_t$. For this purpose we first introduce the weighted negative Sobolev distance \manote{cite villani and peyre and Mroueh}:
\begin{align}\label{eq:neg_sobolev}
	\Vert \nu - \mu \Vert_{\dot{H}^{-1}(\nu)} = \sup_{\substack{ f\in W_0^{1,2}(\nu), \; \nu(\Vert \nabla f \Vert^2) \leq 1 }} \vert \nu(f)-\mu(f)\vert 
\end{align}
Where $W_0^{1,2}(\nu)$ is the space $1$ order Sobolev functions with functions vanishing at the boundary of the domain.
The distance defined in \cref{eq:neg_sobolev} plays a fundamental role in dynamic optimal transport as it linearizes the $W_2$ distance when $\mu$ is arbitrarily close to $\nu$. It can also be seen as the minimum kinetic energy needed to advect the mass $\nu$ to $\mu$. However, this quantity might be infinite \manote{say exactly when it is finite} and one of the key problems would be to control its value during the evolution of the flow. More precisely we will rely on the following statement:
\begin{align}\label{eq:bounded_neg_sobolev}
	\Vert \nu_t  - \mu \Vert_{\dot{H}^{-1}(\nu_t)} \leq C \qquad \forall t\geq 0.
\end{align} 
where $\nu_t$ is defined by the gradient flow and $\mu$ is the target distribution. When \cref{eq:bounded_neg_sobolev}  holds, we have the following proposition:
\begin{proposition}\label{prop:lojasiewicz}
	When \cref{eq:bounded_neg_sobolev} holds, the following inequality is then satisfied at all times:
	\begin{align}\label{eq:PL_type_inequality}
		\Vert \nabla f_t \Vert_{L_2(\nu_t)} \geq \frac{1}{C} \Vert f_t \Vert^2_{\mathcal{H}} \qquad \forall t\geq 0.
	\end{align}
	Then $t\mapsto \mathcal{F}(\nu_t)$ converges to $0$ with a rate of convergence given by:
	\begin{align}
	\mathcal{F}(\nu_t)\leq \frac{1}{\mathcal{F}(\nu_0)^{-1} + \frac{4t}{C}}
	\end{align}
\end{proposition}



All the difficulty is to see now when \cref{eq:bounded_neg_sobolev} holds. One possible strategy would be to start from initial $\nu_0$ such that $\Vert \nu_0  - \mu \Vert_{\dot{H}^{-1}(\nu_0)} \leq C $  for some finite positive value $C$ and then show that this property is preserved during the dynamics. It is also possible to have a time depended constant $C_t$ as long as its growth is such that:
\begin{align}
	\lim_{t\rightarrow +\infty} \int_0^t C_s^{-1}\diff s = +\infty
\end{align}
For instance $C_t$ could have up to a linear growth in time. In this case the decay of $\F(\nu_t)$ will no longer be in $\frac{1}{t}$ but only in $\frac{1}{\log(t)}$ \manote{This seems unlikely if we end up having convergence of $\nu_t$, but who nows.}.
One possible promising condition for \cref{eq:bounded_neg_sobolev} to hold would be if $\mu \ll \nu_0$ and if this property is preserved during the dynamics.


 










\subsection{MMD flows in the literature}

\begin{remark}
	We point out here that algorithm~\cref{eq:sample_based_process} is different from the descent proposed by \cite{mroueh2018regularized}. 
\end{remark}

\begin{remark}
	Birth-Death Dynamics to improve convergence (see \cite{rotskoff2019global}).
\end{remark}

%\section{Theoretical properties of the MMD flow}\label{sec:theory}



\subsection{Lambda displacement convexity of the MMD}

One important criterion to characterize the convergence of the gradient flow of a functional $\F$ is the notion of \textit{displacement convexity} of such a functional. Displacement convexity (see \cite{Villani:2004}, Definition 1) states that the functional evaluated at any distribution in a geodesic path between two distributions $\nu$ and $\mu$ will be upper-bounded by a convex mixture of $\F(\nu)$ and $\F(\mu)$, as explained formally in the following definition.
\begin{definition}\label{def:displacement_convexity}
 Let $\mu$
and $\nu$ in $\mathcal{P}(\X)$. There exists a $\mu-a.e.$
unique gradient of a convex function, denoted by $\nabla\phi$, such that $\mu$
is equal to $\nabla\phi_{\#}\nu$ and one can define the displacement geodesic $\nu_{t}=((1-t)Id+t\nabla\phi)_{\#}\nu$
for $0\leq t\leq1$. We say that a functional $\nu\mapsto\mathcal{F}(\nu)$
is displacement convex if 
\begin{equation}
t\mapsto\mathcal{F}(\nu_{t})
\end{equation}
 is convex for any $\nu$ and $\mu$. %Moreover, we say that $\mathcal{F}$ is displacement convex in a neighborhood of $\mu$ if there exists a radius $r>0$ such that the above property holds for any $\nu$ with $W_{2}(\mu,\nu)\leq r$.
\end{definition}


This notion of convexity is to be related to the more widely used notion of convexity called \textit{mixture convexity}:
\begin{align}
	\F(t\nu +(1-t)\nu')\leq t\F(\nu)+(1-t)\F(\nu') \qquad t\in [0,1]
\end{align}
%Unlike mixture convexity, displacement convexity is compatible with the $W_2$ metric and is therefore the natural notion to use for characterizing convergence of gradient flows in the $W_2$ metric.
Although mixture convexity holds for $\F$ (see \cref{lem:mixture_convexity}), this property is less critical for characterizing convergence of gradient flows in the $W_2$ metric. On the other hand, displacement convexity is compatible with the $W_2$ metric \cite{Bottou:2017} and is therefore the natural notion to use in our setting. Unfortunately, $\F$ fails to be displacement convex in general. Instead we will show that $\F$ satisfies some weaker notion of convexity called $\Lambda$-displacement convexity (see also \cite{Villani:2009} Definition 16.4).
%
\begin{definition}\label{def:lambda-convexity}
%($\Lambda$-convexity \cite{Villani:2009} Definition 16.4). 
%\[
%\inf_{\nu,v}\frac{\Lambda(\nu,v)}{\Vert v\Vert_{L_{2}(\nu)}^{2}}>-\infty.
%\]
We say that a functional $\nu\mapsto\mathcal{F}(\nu)$ is $\Lambda$-convex
if for any $\nu$ and $\mu$ and a minimizing geodesic $\text{\ensuremath{\nu_{t}}}$
between $\nu$ and $\mu$ with velocity vector field $v_{t}$, i.e:
$\partial_{t}\nu_{t}+div(\nu_{t}v_{t})=0;\nu_{0}=\nu;\nu_{1}=\mu;$
the following holds:
\begin{equation}\label{eq:lambda_displacement_convex}
\frac{d^{2}\mathcal{F}(\nu_{t})}{dt^{2}}\geq\Lambda(\nu_{t},v_{t})\qquad\forall\; t\in[0,1].
\end{equation}
where $(\nu,v)\mapsto\Lambda(\nu,v)$
is a function that defines for each $\nu \in \mathcal{P}(\X)$
a quadratic form on the set of square integrable vectors valued functions
$v$ , i.e: $v\in L_{2}(\mathbb{R}^{d},\mathbb{R}^{d},\nu)$. We
further assume that $\inf_{\nu,v}\Lambda(\nu,v)/\Vert v\Vert_{L_{2}(\nu)}^{2}>-\infty$. 
Also, the following holds:
\begin{equation}
\F(\nu_{t})\leq(1-t)\F(\nu_{0})+t\F(\nu_{1})-\int_{0}^{1}\Lambda(\nu_{s},v_{s})G(s,t)ds\label{eq:integral_lambda_convexity}
\end{equation}
where $G(s,t)=s(1-t) \mathbb{I}\{s\leq t\}
+t(1-s) \mathbb{I}\{s\geq t\}$.
\end{definition}

%Then, to show the $\Lambda$-convexity of the functional defined in \cref{sec:gradient_flow} we first make the following assumptions on the kernel:
%\begin{assumplist} 
%	\item \label{assump:bounded_trace} $ \vert \sum_{1\leq i\leq d} \partial_i\partial_ik(x,x) \vert\leq \frac{L}{3}  $ for all $x\in \mathbb{R}^d$.
%	\item \label{assump:bounded_hessian} $\Vert H_xk(x,y) \Vert_{op} \leq \frac{L}{3}$ for all $x,y\in \mathbb{R}^d$, where $H_xk(x,y)$ is the hessian of $x\mapsto k(x,y)$ and $\Vert.\Vert_{op}$ is the operator norm.
%	\item \label{assump:bounded_fourth_oder} $\Vert Dk(x,y) \Vert\leq \lambda  $ for all $x,y\in \mathbb{R}$, where $Dk(x,y)$ is an $\mathbb{R}^{d^2}\times \mathbb{R}^{d^2}$ matrix with entries given by $\partial_{x_{i}}\partial_{x_{j}}\partial_{x'_{i}}\partial_{x_{j}'}k(x,y)$.
%\end{assumplist}
The next proposition states that the functional defined in \cref{sec:gradient_flow} is $\Lambda$-displacement convex and provide and explicit expression for the functional $\Lambda$. Some additional mild assumptions on the derivative of the kernel are also needed but deferred to the appendix for presentation purpose.

\begin{proposition}
\label{prop:lambda_convexity} Suppose $\sup_{x,y} \partial_{x_{i}}\partial_{x_{j}}\partial_{x'_{i}}\partial_{x_{j}'}k(x,y)\le \lambda$ is satisfied for some $\lambda \in \R^+$. The functional $\nu\mapsto \F(\nu)$ is $\text{\ensuremath{\Lambda}}$-convex
with $\Lambda$ given by:
\begin{equation}
\Lambda(\nu,v)=\langle v,(C_{\nu}-\lambda \F(\nu)^{\frac{1}{2}}I)v\rangle_{L_{2}(\nu)}\label{eq:Lambda}
\end{equation}
where $C_{\nu}$ is the (positive) operator defined by:
\begin{align}\label{eq:positive_operator_C}
	(C_{\nu}v)(x)=\int\nabla_{x}\nabla_{x'}k(x,x')v(x')d\nu(x')
\end{align}
\end{proposition}
%
%
Consider the geodesic $\nu_{t}=((1-t)Id+t\nabla\phi)_{\#}\nu$ of \cref{def:displacement_convexity}, so that $\nu_{1}=\mu$ and at time $t=1$ and thus $\F(\nu_{1})=0$. It is worth noting that by \cref{prop:lambda_convexity}, we get that the non-negative hessian \eqref{eq:lambda_displacement_convex} at the global minimum $\mu$, is greater than $\langle v_{t},C_{\nu_{t}}v_{t}\rangle_{L_{2}(\nu_{t})}\geq0$ which is itself positive. Also, we can now write the following convexity inequalities along the gradient flow of $\F$.
%\[
%\frac{d^{2}\F(\nu_{t})}{dt^{2}}\vert_{t=1}=\langle v_{t},C_{\nu_{t}}v_{t}\rangle_{L_{2}(\nu_{t})}\geq0.
%\]
%This shows that $\nu\mapsto \F(\nu)$ has a non-negative
%hessian at $\mu$ which is not surprising since $\mu$ is the global
%minimum of this functional.
%\begin{corollary}\label{cor:integral_lambda_convexity}
%For any geodesic $\nu_{t}$ between
%$\nu_{0}$ and $\nu_{1}$ in $\mathcal{P}(\X)$ the following holds:
%\begin{equation}
%\F(\nu_{t})\leq(1-t)\F(\nu_{0})+t\F(\nu_{1})-\int_{0}^{1}\Lambda(\nu_{s},v_{s})G(s,t)ds\label{eq:integral_lambda_convexity}
%\end{equation}
%where $\Lambda$ is given by \cref{eq:Lambda} and $G(s,t)=s(1-t) \mathbb{I}\{s\leq t\}
%+t(1-s) \mathbb{I}\{s\geq t\}$.
%\end{corollary}


\begin{corollary}
\label{cor:loser_bound}Assume the distributions are supported on
$\mathcal{X}$ and the kernel is bounded, i.e: $\sup_{x,y\in\mathcal{X}}\vert k(x,y)\vert<\infty$.
Then the following holds:
\begin{equation}
\F(\nu_{t})\leq(1-t)\F(\nu_{0})+t\F(\nu_{1})+t(1-t)K
\end{equation}
where $K$ is a constant depending on $\X$ and the kernel $k$ in $\F$.
\end{corollary}
%
%
\cref{cor:loser_bound}, is a loser bound and does not account for the local
convexity of the MMD. However, it allows to state the following result,
which is inspired from (\cite{Bottou:2017}, Theorem 6.3) but generalizes
it to the case of 'almost convex' functionals.
\begin{proposition}
\label{prop:almost_convex_optimization}
(Almost convex optimization). Let $\mathcal{P}$ be a closed subset
of $\mathcal{P}(\mathcal{X})$ which is displacement convex\aknote{weird for a set to be displacement convex? it was for functionals}. Then
for all $M>\inf_{\nu\in\mathcal{P}}\F(\nu)+K$, the following
holds:
\end{proposition}
\begin{enumerate}
\item The level set $L(\mathcal{P},M)=\{\nu\in\mathcal{P}:\F(\nu)\leq M\}$
is connected
\item For all $\nu_{0}\in\mathcal{P}$ such that $\F(\nu_0)>M$
and all $\epsilon>0$, there exists $\nu\in\mathcal{P}$ such that
$W_{2}(\nu,\nu_{0})=\mathcal{O}(\epsilon)$ and
\[
\F(\nu)\leq \F(\nu_{0})-\epsilon(\F(\nu_{0})-M).
\]
\end{enumerate}
%
%\begin{remark}
The result in \Cref{prop:almost_convex_optimization} means that it is possible to optimize the cost function $\nu\mapsto \F(\nu)$
on $\mathcal{P}$ as long as the barrier $\inf_{\nu\in\mathcal{P}}\F(\nu)+K$
is not reached. However, this barrier remains large, since it depends on particular of the diameter of $\X$\aknote{recall $K=2\lambda Cdiam(\mathcal{X})$}. A possible direction to refine the statement in \cref{prop:almost_convex_optimization} would be to directly leverage the tighter inequality in \cref{eq:integral_lambda_convexity} to get a better description of the loss landscape.\aknote{can we really?}



\subsection{Regularized MMD flow - convergence to a global optimum}

Although the Wasserstein flow of the MMD decreases the MMD in time, it can very well remain stuck in local minima.\asnote{I think that the same problem happens with the dynamics of SVGD. Because KSD = 0 doesn't imply p = q unless absolute continuity + other requirements} One way to see how this could happen, at least formally, is by looking at \cref{eq:time_evolution_mmd} at equilibrium. Indeed $\F(n_t)$ is a non-negative decreasing function of time, it must therefore converge to some limit, this implies that its time derivative would also converge to $0$. Assuming that $\nu_t$ also converged to some limit distribution $\nu^{*}$ one can show that under simple regularity conditions that the equilibrium condition
\begin{align}\label{eq:equilibrium_condition}
	\int \Vert \nabla f_{\mu,\nu^{*}}(x)\Vert^2 \diff \nu^{*}(x) =0  
\end{align} 
must hold. If $\nu^*$ had a positive density then this would imply that $f_{\mu,\nu^{*}}(x)$ is constant everywhere. If the set of functions spanned by the RKHS associated to the MMD do not include constant functions, then it must hold that $f_{\mu,\nu^{*}}=0$ which in turn means that $MMD(\mu,\nu^{*})=0$, hence $\nu^*$ would be a global solution. However, the limit distribution $\nu^*$  might be very singular, it could even be a dirac distribution. In that case, the optimality condition \cref{eq:equilibrium_condition} is of little use. Moreover, it suggests that the gradient flow could converge to some suboptimal configuration as the gradient is only evaluated near the support of $\nu^*$.
Since \cref{eq:equilibrium_condition} seems to be the main obstruction to reach global optimality, we propose to construct, at least formally, a modified gradient flow for which the optimality condition would guarantee reaching the global optimum.
Ideally, we would like to obtain an optimality condition of the form
\begin{align}\label{eq:soothed_equilibrium_condition}
	\int \Vert \nabla f_{\mu,\nu^{*}}(x)\Vert^2 \diff (\nu^{*}\star g)(x) =0  
\end{align}
where $\nu^{*}\star g$ means the convolution of $\nu^*$ with a gaussian distribution $g$. The smoothing effect of convolution directly implies that $\nu^{*}\star g$ has a positive density, which falls back in the scenario where the $\nu^*$ must a global optimum.
We consider, at least formally, the following modified equation for $\nu_t$:
\begin{align}\label{eq:smoothed_continuity_equation_mmd}
	\partial_t \nu_t = div((\nu_t \star g) \nabla f_{\mu,\nu_t} )
\end{align}
This suggests a particle equation which would be given by:
\begin{align}\label{eq:noisy_particles}
	\dot{X}_t = -\nabla f_{\mu,\nu_t}( X_t + W_t  )
\end{align}
where $(W_t)$ is a brownian motion. Furthermore, $\F(\nu_t)$ satisfies
\begin{align}\label{eq:smoothed_decreasing_mmd}
	\dot{\F}(\nu_t) = -\int \Vert \nabla f_{\mu,\nu_t}(x)\Vert^2 \diff (\nu_t\star g)(x)
\end{align}
The existence and uniqueness of a solution to \cref{eq:smoothed_continuity_equation_mmd} for a general $g$ remains an open question to our knowledge. However, we find it useful here to state \cref{eq:smoothed_continuity_equation_mmd,eq:noisy_particles,eq:smoothed_decreasing_mmd} which are the modified analogs of \manote{ref to the analogs}. This regularize flow will lead to a new algorithm in the next section.
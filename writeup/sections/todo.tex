
\section*{TO DO}

Proofs:
\begin{itemize}
	%\item Get rates in $\nu_n$ (change Lyapunov function)
	\item Hard: Refine the bounds, quantify more precisely $K(\rho_n)$
	\item Hard: Polyak Lojasewicz (PL) for MMD. When is the condition on Negative Sobolev norms true?
	\item Finish to write the proof for Lambda convexity and Sample based
	\item refine the bound for sample-based
\end{itemize}


Experiments/Literature
\begin{itemize}
	\item Talk about Birth-Death Dynamics and Mroueh
	\item Experiments!!
\end{itemize}
everywhere: treat comments if possible

Add to appendix:\\

Given a probability distribution $\nu$ the \textit{weighted Sobolev semi-norm} is defined for all squared integrable functions $f$ in $L_2(\nu)$ as $ \Vert f \Vert_{\dot{H}(\nu)} = \left(\int \Vert \nabla f(x) \Vert^2 \diff \nu(x) \right)^{\frac{1}{2}}$ with the convention $\Vert f \Vert_{\dot{H}^1(\nu)} = +\infty$ if $f$ doesn't have a square integrable gradient.

\subsubsection{Proof of \cref{prop:existence_uniqueness}}

\begin{proof}\aknote{this will go in the Appendix. please comment the other proposition in the appendix if this stays in the main.tex}
	Under Lipschitzness of $\nabla k$, the map $(x,\nu)\mapsto \nabla f_{\mu,\nu}(x)=\int \nabla k(x,.)d \nu - \int \nabla k(x,.) d \mu$ is Lipschitz continuous on $\X \times \mathcal{P}_2(\X)$ (endowed with the product of the canonical metric on $\X$ and $W_2$ on $\mathcal{P}_2(\X)$), hence we benefit from standard existence and uniqueness results of McKean-Vlasov processes (see \cite{Jourdain:2007}). Then, it is straightforward to verify that the distribution of \eqref{eq:mcKean_Vlasov_process} is solution of \eqref{eq:continuity_mmd} by Itô's formula (see \cref{sec:ito_stochastic}). The uniqueness of a gradient flow, given a starting distribution $\nu_0$, results from the $\lambda$-convexity of $\F$ which will be proven later, and from Theorem 11.1.4 of \cite{ambrosio2008gradient}. The existence derive from the fact that the subdifferential of $\F$ is single-valued, and stated by \cref{prop:differential_mmd}, and that any $\nu_0$ in $\mathcal{P}(\X)$ is in the domain of $\F$.\aknote{check} The existence then results from Theorem 11.1.6 and Corollary 11.1.8 from \cite{ambrosio2008gradient}.
\end{proof}
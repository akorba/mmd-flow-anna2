\section{MMD flow}


\subsection{Background}

Endow the space of probability measures $\mathcal{P}(\X)$ on a domain $\X \subset \R^d$ with a distance (e.g, optimal transport distances), and then deal with gradient flows of suitable functionals on such a metric space.



\subsubsection{Optimal transport tools}

Let $\X \subset \R^d$. Let $\mathcal{M}^{+}$ and $\mathcal{M}_2^{+}$ be the set of probability distributions on $\X$, and the set of probability distributions on $\X$ with finite second moment respectively.
Let $T: \X \rightarrow \X$ be a measurable map, and $\mu \in \mathcal{M}^+$. The measure $T_{\#}\mu$
is characterized by:
\begin{align*}
	&\quad T_{\#}\mu(A) = \mu(T^{-1}(A)) \text{ for every measurable set A,}\\
\text{or}& \int_{y \in \X} \phi(y) d(T_{\#}\mu)(y) =\int_{x \in \X}\phi(T(x)) d\mu(x) \text{ for every measurable function $\phi$.}
\end{align*}


\begin{definition}\label{def:mk}
	The quadratic \textit{Monge-Kantorovitch distance} of $\rho_0, \rho_1 \in \mathcal{M}_2^{+}$ is:
	\begin{equation}
		W^2(\rho_0, \rho_1)= \min_{\gamma \in \Gamma[\rho_0, \rho_1]} \int_{\X \times \X} |x-y|^2 d\gamma(x,y)
	\end{equation}
where $\Gamma(\rho_0,\rho_1)$ is the set of transport plans of $\rho_0, \rho_1$ consists of probability measures $\gamma \in \mathcal{M}^{+}(\X \times \X)$ with marginals $\rho_0, \rho_1$ : for every measurable set $A \subset \bR^d$, $\gamma(A \times \bR^d) = \rho_0(A)$ and $\gamma(\bR^d \times A) = \rho_1(A)$. 
	The minimizer is unique and is called an \textit{optimal plan}. 
\end{definition}

Add minimizing geodesic.

Given a functional $\F : \mathcal{M}^+ \rightarrow \R$ we call $\frac{\partial{\F}}{\partial{\rho}}$ if it exists, the unique (up to additive constants) function such that $\frac{d}{d\epsilon}\F(\rho+\epsilon  f)_{\epsilon=0}=\int\frac{\partial{\F}}{\partial{\rho}}(\rho) df$ for every perturbation $f$ such that, at least for $\epsilon \in [0, \epsilon_0]$, the measure $\rho +\epsilon f$ belongs to $\mathcal{M}^+$. The function $\frac{\partial{\F}}{\partial{\rho}}$ is called first variation of the functional $\F$ at $\rho$.

\subsubsection{Gradient flows on the space of probability measures}

Consider a \textit{Lyapunov functional} (or "free energy" or "entropy") $\F$ (see \citep{Villani:2004}), i.e. a functional of the form:
\begin{equation}\label{eq:lyapunov}
\F(\rho)=\int U(\rho(x)) \rho(x)dx + \int V(x)\rho(x)dx + \int W(x,y)\rho(x)\rho(y)dxdy
\end{equation}
where  $U$ is the internal energy, $V$ the potential energy and $W$ the
interaction energy. The formal gradient flow equation associated to this functional can be written:
\begin{equation}\label{eq:continuity_equation}
\frac{\partial \rho}{\partial t}= div( \rho \nabla \frac{\partial \F}{\partial \rho}) \quad x \in \R^d , t>0
\end{equation}
And the dissipation of entropy is defined as %see  http://wwwf.imperial.ac.uk/~jcarrill/RICAM/CharlaRICAM2014-1.pdf
\begin{align}
&        \frac{d \F(\rho)}{dt} =-D(\rho) \quad \text{ with } D(q)= \int |\xi|^2 \rho(x)dx\\
&\text{ and } \xi= \nabla (U'(\rho) + V + W * \rho)= \nabla \frac{\partial \F}{\partial \rho}
\end{align}
Standard considerations from fluid mechanics tell us that the continuity equation \eqref{eq:continuity_equation} may be interpreted as the equation ruling the evolution of the density $\rho_t$ of a family of particles initially distributed according to some $\rho_0$ and each of which follows the velocity/vector field $v_t=\nabla \frac{\partial{\F}}{\partial{\rho_t}}$.

\begin{remark} 
	A famous example of a functional \eqref{eq:lyapunov} is the Kullback-Leibler divergence, defined for $\rho, \pi \in \mathcal{M}^+$ by
	$KL(\rho,\pi)=\int log(\frac{\rho(x)}{\pi(x)})\rho(x)dx$. Indeed, $KL(\rho, \pi)=\int U(\rho(x))\rho(x)dx + \int V(x) \rho(x)dx$ with $U(\rho(x))=\rho(x)log(\rho(x))$ and $V(x)=-log(\pi(x))$. In this case, $\nabla \frac{\partial \F}{\partial \rho}= \nabla \log(\rho) + \nabla V=  \nabla \log(\frac{\rho}{\pi})$ and equation \eqref{eq:continuity_equation} becomes the classical Fokker-Planck equation:
	\begin{equation}\label{eq:Fokker-Planck}
	\frac{\partial{\rho}}{\partial t}= div(\rho \nabla V )+ \Delta \rho
	\end{equation}
	It can be shown that the distribution $\rho_t$ of the Langevin diffusion~\eqref{eq:langevin_diffusion} is a solution of \eqref{eq:Fokker-Planck}.
\end{remark}

\subsection{MMD flow}



Let $\kH$ a Reproducing Kernel Hilbert Space (RKHS) and $k$ its reproducing kernel. This means that for all $f \in \kH$, $x \in \X$, we can write the \textit{reproducing property} $f(x)=\psh{f, k(x,.)}$. The kernel Maximum Mean Discrepancy between two distributions $\rho,\pi$ is defined as:
\begin{equation}
MMD(\rho,\pi)=\sup_{f \in \kH,  \|f\|_{\kH}\le 1} (\E_{X \sim \rho}[f(X)]-\E_{Y \sim \pi}[f(Y)])
\end{equation}
Under some appropriate assumptions on the kernel (see \cite{gretton2012kernel}):
\begin{align}
MMD^2(\rho,\pi)&=\|\E_{\rho}[k(X,.)] - \E_{\pi}[k(Y,.)]\|^2_{\kH}\\
&=\E_{\rho \otimes \rho}[k(X,X')]+\E_{\pi \otimes \pi}[k(Y,Y')] - 2\E_{\rho \otimes \pi}[k(X,Y)]
\end{align}

We will consider a flow $(\rho_t)_{t>0}$ as described in the previous subsection and denote $f_t= \E_{\rho_t}[k(X,.)]- \E_{\pi}[k(Y,.)]$. In this case:
\begin{equation}
MMD^2(\rho_t,\pi)=\|f_t\|^2_{\kH}
%&= \E_{\rho_t \otimes \rho_t}[k(X,X')]+\E_{\pi \otimes \pi}[k(Y,Y')] - 2\E_{\rho_t \otimes \pi}[k(X,Y)]
\end{equation} 

We define the potential energy (also called confinement energy) $V$ and interaction energy $W$ as follows:
\begin{equation}
V(X)=-\int 2 k(X,x')\pi(x') \quad \text{,} \quad
W(X,Y)=k(X,Y)
\end{equation}
We have $MMD^2(\rho,\pi)=C+ \int V(x) \rho(x)dx + \int W(x,x')\rho(x)\rho(x')$, where $C=\E_{\pi\otimes \pi}[k(Y,Y')]$. $MMD^2$ can thus be written as a \textit{Lyapunov functional} (or "free energy" or "entropy") $\F$.


\begin{proposition}\label{prop:mmd_flow}
 The velocity $\nabla \frac{\partial{\F}}{\partial{\rho_t}}=2 \nabla f_t$ and the dissipation of MMD is given by:  
	\begin{equation}
	\frac{d MMD^2(\rho_t, p)}{dt}=-2 \E_{X \sim \rho_t}[\|\nabla f_t(X)\|^2]
	\end{equation}
	where $\nabla f_t(Y)= \E_{X \sim \rho_t}[\nabla_{Y}k(X,Y)] -  \E_{X \sim \pi}[\nabla_{Y}k(X,Y)]$.
\end{proposition}

\begin{remark}
	If the functional $\F$ was the KL divergence and $\rho_t$ a weak solution of the Fokker-Planck equation \eqref{eq:Fokker-Planck}, we would obtain the following dissipation (see \cite{wibisono2018sampling}):
	\begin{equation}
	\frac{d KL(\rho_t, \pi)}{dt}=-\E_{X \sim \rho_t}[\|\nabla log(\frac{\rho_t}{\pi}(X))\|^2]
	\end{equation}.
\end{remark}




\subsection{Algorithm}

The gradient flow of MMD can be written:
\begin{equation*}
\frac{\partial \rho}{\partial t}= 2 div(\rho  \nabla f_t)
\end{equation*}
which is the density of the stochastic process:
\begin{equation}\label{eq:stochastic_process}
dX_t=-2\nabla f_t(X_t) 
\end{equation}
It represents the position $X_t$ of a particle at time $t > 0$.


We can thus consider the Euler discretization of \eqref{eq:stochastic_process}:
\begin{equation}\label{eq:discretization}
X_{k+1}=X_k - \gamma_{k+1} \nabla f_k(X_k)
\end{equation}
where $\nabla f_k(X_k)= \E_{X \sim \rho_k}[\nabla_{X_k}k(X,X_k)] -  \E_{X \sim \pi}[\nabla_{X_k}k(X,X_k)]$.
In practice, since we don't have access to the target distribution $\pi$, we estimate this gradient by:
\begin{equation*}
\widehat{\nabla f_k}(X_k)=\frac{1}{n}\sum_{i=1,\dots,n}\nabla_{X_k}k(x_i,X_k) - \frac{1}{n}\sum_{i=1,\dots,n}\nabla_{X_k}k(y_i,X_k)
\end{equation*}
where $(x_1, \dots, x_n)\sim \rho_k$ and $(y_1, \dots, y_n)\sim \pi$.


In this subsection we assume that $MMD^2$ is $\lambda$-geodesically-convex. Conditions under which this holds will be provided in the next section.

\subsection{Analysis of the theoretical algorithm}

Equation~\eqref{eq:discretization} provides a theoretical algorithm to minimize $MMD^2(\cdot,\pi)$. The algorithm is only theoretical because it requires to compute $\nabla f_k(X_k)$.

This algorithm is the discretization of the Gradient flow associated to $MMD^2$. Since $MMD^2$ is $\lambda$-convex, using Theorem 11.1.4 of~\cite{ambrosio2008gradient}, the gradient flow $(\rho_t)$ satisfies
\begin{equation}
    \label{eq:evi}
    \frac12 \frac{d}{dt} W^2(\rho_t,\nu) + \frac{\lambda}{2}W^2(\rho_t,\nu) \leq MMD^2(\nu,\mu) - MMD^2(\rho_t,\pi)
\end{equation}
Unfortunately, $\lambda \leq 0$ (otherwise it would mean that $MMD^2$ is strongly-geodesically-convex and hence geodesically convex).

For the theoretical algorithm~\eqref{eq:discretization} we can expect a discretized version of~\eqref{eq:evi} to hold : \asnote{This should hold, I haven't proved it yet but I will do it later}
\begin{equation}
    \label{eq:evi-discrete}
    W^2(\rho_{k+1},\pi) \leq  W^2(\rho_{k},\pi) -2\gamma_{k+1}\left( \frac{\lambda}{2}W^2(\rho_{k},\pi) + MMD^2(\rho_{k+1},\pi) - MMD^2(\pi,\pi)\right)
\end{equation}
Since $\lambda \leq 0$ we cannot have a rate from this inequality (I think).
However, if $\sum \gamma_k < \infty$, using Robbins Siegmund lemma, we know that $W^2(\rho_{k},\pi)$ converges to some $\ell \geq 0$. \asnote{From this it might be possible to prove that $W^2(\overline{\rho_{k}},\pi)$ converges to zero, but it would be a lot of work. It looks like Pakes Hasminskii criterion}

\subsection{Another Lyapunov function}

In this section we try to use $MMD^2(\cdot,\pi)$ as a Lyapunov function (instead of $W^2(\cdot,\pi)$), like in Theorem 3.3 of Liu 2017. Once this is done, we can use the Gradient Lojasiewicz inequality to get a rate (see Bolte, it's like log sobolev inequality)

Taylor : 


\subsection{Sample-based setting}

Two settings are usually encountered in the sampling literature:
\begin{itemize}
	\item Density-based: $\mu$ is known up to a constant
	\item Sample-based: we only have access to a set of samples $X \sim \mu$.
\end{itemize}

\aknote{to investigate much further} 
The Unadjusted Langevin Algorithm (ULA) seems much more adapted to the first setting, since it only requires the knowledge of $\nabla \log \mu$ while our algorithm requires the knowledge of $\mu$ (since $\nabla f_t$ involves an integration over $\mu$). However, in the sample-based setting, it may be difficult to adapt ULA by replacing $\nabla \log \mu$ in \eqref{eq:langevin_algorithm} by an estimator based on samples. Indeed, it has been the subject of a lot of work (see \cite{li2017gradient})\aknote{check reference}. In contrast, the gradient of $f_t$ can be 'easily' estimated by:
\begin{equation}
\widehat{\nabla f_k}(X_k)=\frac{1}{n}\sum_{i=1,\dots,n}\nabla_{X_k}k(x_i,X_k) - \frac{1}{n}\sum_{i=1,\dots,n}\nabla_{X_k}k(y_i,X_k)
\end{equation}
where $(x_1, \dots, x_n)\sim \rho_k$ and $(y_1, \dots, y_n)\sim \mu$. 
It is thus natural to consider the stochastic process:
\begin{equation}\label{eq:sample_based_stochastic_process}
X_{k+1}=X_k-\gamma_{k+1}\widehat{\nabla f_k}(X_k) 
\end{equation}
Let $\widetilde{\nu}_k$ be the distribution of~\eqref{eq:sample_based_stochastic_process}. Can we say something about $MMD^2(\widetilde{\nu_k}, \nu_k)$? If yes, we have rates as well.

\begin{remark}
	We point out here that algorithm~\eqref{eq:sample_based_stochastic_process} is different from the descent proposed by \cite{mroueh2018regularized}. 
\end{remark}

\begin{remark}
	Birth-Death Dynamics to improve convergence (see \cite{rotskoff2019global}).
\end{remark}

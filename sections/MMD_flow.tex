\section{MMD Gradient flow - Algorithms, experiments}\label{sec:mmd_flow}

\subsection{MMD Descent - Theoretical algorithm}

We will consider a flow $(\rho_t)_{t>0}$ as described in \cref{sec:gradient_flows_functionals} and denote $f_t= \int k(.,z)\diff \mu - \int k(.,z)\diff \rho_t$. In this case:
\begin{equation}
\F(\rho_t)=\frac{1}{2}\|f_t\|^2_{\kH}
%&= \E_{\rho_t \otimes \rho_t}[k(X,X')]+\E_{\pi \otimes \pi}[k(Y,Y')] - 2\E_{\rho_t \otimes \pi}[k(X,Y)]
\end{equation} 

We define the potential energy (also called confinement energy) $V$ and interaction energy $W$ as follows:
\begin{equation}
V(x)=-\int 2 k(x,x')\mu(x')\text{,} \quad
W(x,x')=k(x,x')
\end{equation}
We have $MMD^2(\rho,\mu)=C+ \int V(x) \rho(x)dx + \int W(x,x')\rho(x)\rho(x')$, where $C=\E_{\mu\otimes \mu}[k(x,x')]$. $MMD^2$ can thus be written as a \textit{Lyapunov functional} (or "free energy" or "entropy") $\F$. 

\begin{remark}
	Consider a family of particles such that its density satisfy Equation\eqref{eq:continuity_equation}. Both KL and MMD have a non-zero potential energy $V$ which drive these particles to the target distribution $\mu$. While he entropy function $U$ in KL prevents the particle from "crashing" onto the mode of $\mu$, this role could be played by the interaction energy $W$ for MMD.
\end{remark}



\begin{proposition}\label{prop:mmd_flow}
 The velocity in \eqref{eq:continuity_equation1} is given by $\nabla \frac{\partial{\F}}{\partial{\rho_t}}=2 \nabla f_t$ and the dissipation of MMD can be written:  
	\begin{equation}
	\frac{d MMD^2(\rho_t, \mu)}{dt}=-\E_{X \sim \rho_t}[\|\nabla f_t(X)\|^2]
	\end{equation}
	where $\nabla f_t(z)= \int \nabla_{z}k(.,z) d\mu -  \int \nabla_{z}k(.,z) d\rho_t$.
\end{proposition}

\begin{remark}
	If the functional $\F$ was the KL divergence and $\rho_t$ a weak solution of the Fokker-Planck equation \eqref{eq:Fokker-Planck}, we would obtain the following dissipation (see \cite{wibisono2018sampling}):
	\begin{equation}
	\frac{d KL(\rho_t, \mu)}{dt}=-\E_{X \sim \rho_t}[\|\nabla log(\frac{\rho_t}{\mu}(X))\|^2]
	\end{equation}
\end{remark}


As explained in \cref{sec:gradient_flows_functionals} and according to \cref{prop:mmd_flow}, the gradient flow of the MMD can be written:
\begin{equation*}
\frac{\partial \rho_t}{\partial t}= div(\rho_t  \nabla f_t)
\end{equation*}
which is the density of the stochastic process (see \cref{sec:ito_stochastic}):
\begin{equation}\label{eq:stochastic_process}
dX_t=-\nabla f_t(X_t) 
\end{equation}
\eqref{eq:stochastic_process} represents the position $X_t$ of a particle at time $t > 0$.
%We naturally consider the Euler discretization of \eqref{eq:stochastic_process}, which gives:
%\begin{equation}\label{eq:discretized_process}
%X_{k+1}=X_k - \gamma_{k+1} \nabla f_k(X_k)
%\end{equation}
%where $\nabla f_k(X_k)= \int \nabla_{X_k}k(X,X_k)\diff \mu - \int \nabla_{X_k}k(X,X_k) \diff \rho_k$ with $\rho_k$ the distribution of the process \eqref{eq:discretized_process} and $(\gamma_k)_{k\ge1}$ is a sequence of step sizes.




In this subsection we assume that $MMD^2$ is $\lambda$-geodesically-convex. Conditions under which this holds will be provided in the next section.

\subsection{Analysis of the theoretical algorithm}

Equation~\eqref{eq:discretization} provides a theoretical algorithm to minimize $MMD^2(\cdot,\pi)$. The algorithm is only theoretical because it requires to compute $\nabla f_k(X_k)$.

This algorithm is the discretization of the Gradient flow associated to $MMD^2$. Since $MMD^2$ is $\lambda$-convex, using Theorem 11.1.4 of~\cite{ambrosio2008gradient}, the gradient flow $(\rho_t)$ satisfies
\begin{equation}
    \label{eq:evi}
    \frac12 \frac{d}{dt} W^2(\rho_t,\nu) + \frac{\lambda}{2}W^2(\rho_t,\nu) \leq MMD^2(\nu,\mu) - MMD^2(\rho_t,\pi)
\end{equation}
Unfortunately, $\lambda \leq 0$ (otherwise it would mean that $MMD^2$ is strongly-geodesically-convex and hence geodesically convex).

For the theoretical algorithm~\eqref{eq:discretization} we can expect a discretized version of~\eqref{eq:evi} to hold : \asnote{This should hold, I haven't proved it yet but I will do it later}
\begin{equation}
    \label{eq:evi-discrete}
    W^2(\rho_{k+1},\pi) \leq  W^2(\rho_{k},\pi) -2\gamma_{k+1}\left( \frac{\lambda}{2}W^2(\rho_{k},\pi) + MMD^2(\rho_{k+1},\pi) - MMD^2(\pi,\pi)\right)
\end{equation}
Since $\lambda \leq 0$ we cannot have a rate from this inequality (I think).
However, if $\sum \gamma_k < \infty$, using Robbins Siegmund lemma, we know that $W^2(\rho_{k},\pi)$ converges to some $\ell \geq 0$. \asnote{From this it might be possible to prove that $W^2(\overline{\rho_{k}},\pi)$ converges to zero, but it would be a lot of work. It looks like Pakes Hasminskii criterion}

\subsection{Another Lyapunov function}

In this section we try to use $MMD^2(\cdot,\pi)$ as a Lyapunov function (instead of $W^2(\cdot,\pi)$), like in Theorem 3.3 of Liu 2017. Once this is done, we can use the Gradient Lojasiewicz inequality to get a rate (see Bolte, it's like log sobolev inequality)

Taylor : 


\subsection{Sample-based setting}

Two settings are usually encountered in the sampling literature:
\begin{itemize}
	\item Density-based: $\mu$ is known up to a constant
	\item Sample-based: we only have access to a set of samples $X \sim \mu$.
\end{itemize}

\aknote{to investigate much further} 
The Unadjusted Langevin Algorithm (ULA) seems much more adapted to the first setting, since it only requires the knowledge of $\nabla \log \mu$ while our algorithm requires the knowledge of $\mu$ (since $\nabla f_t$ involves an integration over $\mu$). However, in the sample-based setting, it may be difficult to adapt ULA by replacing $\nabla \log \mu$ in \eqref{eq:langevin_algorithm} by an estimator based on samples. Indeed, it has been the subject of a lot of work (see \cite{li2017gradient})\aknote{check reference}. In contrast, the gradient of $f_t$ can be 'easily' estimated by:
\begin{equation}
\widehat{\nabla f_k}(X_k)=\frac{1}{n}\sum_{i=1,\dots,n}\nabla_{X_k}k(x_i,X_k) - \frac{1}{n}\sum_{i=1,\dots,n}\nabla_{X_k}k(y_i,X_k)
\end{equation}
where $(x_1, \dots, x_n)\sim \rho_k$ and $(y_1, \dots, y_n)\sim \mu$. 
It is thus natural to consider the stochastic process:
\begin{equation}\label{eq:sample_based_stochastic_process}
X_{k+1}=X_k-\gamma_{k+1}\widehat{\nabla f_k}(X_k) 
\end{equation}
Let $\widetilde{\nu}_k$ be the distribution of~\eqref{eq:sample_based_stochastic_process}. Can we say something about $MMD^2(\widetilde{\nu_k}, \nu_k)$? If yes, we have rates as well.

\begin{remark}
	We point out here that algorithm~\eqref{eq:sample_based_stochastic_process} is different from the descent proposed by \cite{mroueh2018regularized}. 
\end{remark}

\begin{remark}
	Birth-Death Dynamics to improve convergence (see \cite{rotskoff2019global}).
\end{remark}

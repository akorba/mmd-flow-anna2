
\documentclass{article}
\usepackage{nips_2018,graphicx}

 \usepackage[english]{babel}   % "babel.sty" + "french.sty"
% \usepackage[english,francais]{babel} % "babel.sty"
% \usepackage{french}                  % "french.sty"
  \usepackage{times}			% ajout times le 30 mai 2003
 
%% --------------------------------------------------------------
%% CODAGE DE POLICES ?
%% Si votre moteur Latex est francise, il est conseille
%% d'utiliser le codage de police T1 pour faciliter la césure,
%% si vous disposez de ces polices (DC/EC)
\usepackage[T1]{fontenc}
\usepackage[utf8]{inputenc} 
\DeclareUnicodeCharacter{00A0}{~}

%% ==============================================================
% Packages divers (mathématiques, etc.)   
\usepackage{amssymb,amsmath,amscd,amsfonts,amsthm,bbm,mathrsfs,yhmath}
%\usepackage{enumerate}
\usepackage[shortlabels]{enumitem}
% \usepackage{showkeys}
\usepackage{hyperref}

% Example definitions.
% --------------------
\def\x{{\mathbf x}}
\def\L{{\cal L}}

\DeclareMathOperator{\var}{\mathbb Var}
\DeclareMathOperator{\cov}{cov}
\DeclareMathOperator{\rank}{rank}
\DeclareMathOperator{\dom}{dom}
\DeclareMathOperator{\zer}{zer}
\DeclareMathOperator{\aver}{av}
\DeclareMathOperator{\inter}{int}
\DeclareMathOperator{\relint}{ri}
\DeclareMathOperator{\epi}{epi}
\DeclareMathOperator{\graph}{gr}
\DeclareMathOperator{\prox}{prox}
\DeclareMathOperator{\tr}{tr}
\DeclareMathOperator{\support}{supp}
\DeclareMathOperator{\dist}{dist}
\DeclareMathOperator{\lev}{lev}
\DeclareMathOperator{\rec}{rec}
\DeclareMathOperator{\cl}{cl}
\DeclareMathOperator{\co}{co}
\DeclareMathOperator{\clo}{\overline co}
\DeclareMathOperator{\distC}{\mathsf d}
\DeclareMathOperator*{\diag}{diag}
\newcommand{\KL}{\mathop{\mathrm{KL}}\nolimits}


\newcommand{\leftnorm}{\left|\!\left|\!\left|}
\newcommand{\rightnorm}{\right|\!\right|\!\right|}

%\newcommand{\eqdef}{{\stackrel{\text{def}}{=}}} 
\newcommand{\eqdef}{:=} 

\newcommand{\1}{\mathbbm 1}
\newcommand{\bs}{\boldsymbol}

\newcommand{\itpx}{{\mathsf x}}
\newcommand{\sx}{{\mathsf x}}
\newcommand{\sy}{{\mathsf y}}
\newcommand{\sz}{{\mathsf z}}
\newcommand{\sw}{{\mathsf w}}
\newcommand{\sF}{{\mathsf F}}
\newcommand{\sH}{{\mathsf H}}

\newcommand{\ZZ}{\mathbb Z}
\newcommand{\CC}{\mathbb{C}}
\newcommand{\bP}{{{\mathbb P}}} 
\newcommand{\bE}{{{\mathbb E}}} 
\newcommand{\bV}{{{\mathbb V}}} 
\newcommand{\bN}{{{\mathbb N}}} 

% Operators, domains, etc.  
\newcommand{\mA}{{\mathcal A}} 
\newcommand{\mB}{{\mathcal B}} 
\newcommand{\mC}{{\mathcal C}} 
\newcommand{\mD}{{\mathcal D}} 
\newcommand{\mO}{{\mathcal O}} 
\newcommand{\mU}{{\mathcal U}}
\newcommand{\mX}{{\mathcal X}}
\newcommand{\mY}{{\mathcal Y}}
\newcommand{\mZ}{{\mathcal Z}} 
\newcommand{\bmD}{\cl({\mathcal D})} 

\newcommand{\sA}{{\mathsf A}}
\newcommand{\sB}{{\mathsf B}}
\newcommand{\sJ}{{\mathsf J}}
\newcommand{\sX}{{\mathsf X}}
\newcommand{\sG}{{\mathsf G}}
\newcommand{\sY}{{\mathsf Y}}

\newcommand{\maxmon}{{\mathscr M}} 
\newcommand{\Selec}{{\mathfrak S}} 

% Sigma fields
\newcommand{\mcA}{{\mathscr A}} 
\newcommand{\mcB}{{\mathscr B}} 
\newcommand{\mcN}{{\mathscr N}} 
\newcommand{\mcT}{{\mathscr T}} 
\newcommand{\mcI}{{\mathscr I}} 
\newcommand{\mcF}{{\mathscr F}} 
\newcommand{\mcG}{{\mathscr G}} 
\newcommand{\mcX}{{\mathscr X}} 
\newcommand{\cP}{{{\mathcal P}}} 
\newcommand{\cS}{{{\mathcal S}}} 
\newcommand{\cZ}{{{\mathcal Z}}} 
\newcommand{\cF}{{{\mathcal F}}} 
\newcommand{\cG}{{{\mathcal G}}} 
\newcommand{\cM}{{{\mathcal M}}} 
\newcommand{\cD}{{{\mathcal D}}} 
\newcommand{\cE}{{{\mathcal E}}} 
\newcommand{\cL}{{{\mathcal L}}}
\newcommand{\cT}{{{\mathcal T}}} 
\newcommand{\cN}{{{\mathcal N}}} 
\newcommand{\cK}{{{\mathcal K}}} 
\newcommand{\cI}{{{\mathcal I}}} 

% Spaces 
\newcommand{\R}{{{\mathbb R}}} 
\newcommand{\E}{{{\mathbb E}}} 
\newcommand{\kH}{{{\mathcal H}}} 
\newcommand{\F}{{{\mathcal F}}} 
\newcommand{\Hil}{E}                % Hilbert   
\newcommand{\Ban}{E}                % Banach   
\newcommand{\RN}{{{\mathbb R}^N}} 
\newcommand{\bR}{{{\mathbb R}}} 

\newcommand{\m}{\mathfrak{m}}
\newcommand{\toL}{\xrightarrow[]{{\mathcal L}}}
\newcommand{\toweak}{\xrightharpoonup[]{{\mathcal L}}}

\newcommand{\ps}[1]{\langle #1 \rangle}

% 
% Almost sure convergence
\newcommand{\toasshort}{\stackrel{\text{as}}{\to}}
\newcommand{\toaslong}{\xrightarrow[n\to\infty]{\text{a.s.}}}

% Convergence in probability 
\newcommand{\toprobashort}{\,\stackrel{\mathcal{P}}{\to}\,}
\newcommand{\toprobalong}{\xrightarrow[n\to\infty]{\mathcal P}}
%
% Convergence in law 
\newcommand{\todistshort}{{\stackrel{\mathcal{D}}{\to}}}
\newcommand{\todistlong}{\xrightarrow[n\to\infty]{\mathcal D}}

\usepackage[textwidth=2cm, textsize=footnotesize]{todonotes}  
\setlength{\marginparwidth}{1.5cm}               %  this goes with todonotes
\newcommand{\pbnote}[1]{\todo[color=cyan!20]{#1}}
\newcommand{\asnote}[1]{\todo[color=green!20]{#1}}
\newcommand{\whnote}[1]{\todo[color=magenta]{#1}}
\newcommand{\wh}[1]{{\color{red} #1}}

%Moreau
\newcommand{\my}{{{\nabla ^\gamma g}}}
\newcommand{\myn}{{{\nabla ^{\gamma_{n+1}} g}}}
%% ==============================================================

\theoremstyle{definition}
\newtheorem{theorem}{Theorem}
\newtheorem{lemma}[theorem]{Lemma}
\newtheorem{corollary}[theorem]{Corollary}
\newtheorem{proposition}[theorem]{Proposition}
\newtheorem{definition}{Definition}
\newtheorem{remark}{Remark}
\newtheorem{condition}{Condition}
\newtheorem{assumption}{Assumption}
\newtheorem{example}{Example}

\title{MMD Flow}

\begin{document}
\maketitle


\begin{abstract} 

\end{abstract}



\section{Introduction}

%OLD INTRO ON LANGEVIN MONTE CARLO
%This paper deals with the problem of sampling from a probability measure $\mu$ on $(\R^d,\mathcal{B}(\R^d))$ which admits a density, still denoted by $\mu$, with respect to the Lebesgue measure.
%This problem appears in machine learning, Bayesian inference, computational physics... Classical methods to tackle this issue are Markov Chain Monte Carlo methods, for instance Metropolis-Hastings algorithm, Gibbs sampling. The main drawback of these methods is that one needs to choose an appropriate proposal distribution, which is not trivial. Consequently, other algorithms based on continuous dynamics have been proposed, such as the over-damped Langevin diffusion:
%\begin{equation}\label{eq:langevin_diffusion}
%dX_t= -\nabla \log \mu (X_t)dt+\sqrt{2}dB_t
%\end{equation}
%where $(B_t)_{t\ge0}$ is a $d$-dimensional Brownian motion. The Langevin Monte-Carlo (LMC) algorithm, or Unadjusted Langevin algorithm (ULA) considers the Markov chain $(X_k)_{k\ge1 }$ given by the Euler-Maruyama discretization of the diffusion \eqref{eq:langevin_diffusion}:
%\begin{equation}\label{eq:langevin_algorithm}
%X_{k+1} = X_k - \gamma_{k+1}\nabla \log \mu(X_k) + \sqrt{2\gamma_{k+1}G_{k+1}}
%\end{equation}
%where $(\gamma_k)_{k\ge1}$ is a sequence of step sizes (constant or convergent to zero), and
%$(G_k)_{k \ge 1}$ is a sequence of i.i.d. standard $d$-dimensional Gaussian random variables. This algorithm has attracted a lot of attention... But....\aknote{say something about the requirement of the knowledge of gradient of log target and how it is difficult to estimate?}

Proposal for the intro:
\begin{itemize}
	\item gradient flows / SDE (ie link between fokker planck and gradient flows in W2 discovered by Otto I think?)
	\item gradient flows in Machine learning (eg: sampling/ Langevin monte carlo,  optimization/neural networks \cite{chizat2018global}, \cite{rotskoff2019global})
\end{itemize}

Interestingly, it has been shown in \cite{jordan1998variational} that the family of distributions $(\rho_t)_{t\ge 0}$ where $\rho_t$ is the distribution of the process \eqref{eq:langevin_diffusion} is the solution of a gradient
flow equation in the Wasserstein space of order 2 associated with a particular functional, the KL-divergence. Langevin Monte-Carlo can thus be formulated as a first order optimization algorithm of the KL-divergence defined on the Wasserstein space of order 2 (see also \cite{durmus2018analysis,bernton2018langevin}). Inspired by this interpretation, we propose a new method to sample from a distribution, discretizing a gradient
flow equation in the Wasserstein space of order 2 associated with another well-chosen functional, the Maximum Mean Discrepancy, introduced in \cite{gretton2012kernel}. To the best of our knowledge, this work is the first one which investigates the flow of a discrepancy between distributions different from the Kullback-Leibler divergence.\asnote{We will probably reviewed by Mroueh so we need to compare their results to ours} \aknote{true? (also, reformulate this sentence)}


This paper is organized as follows. In \cref{sec:preliminaries}, definitions and mathematical background needed in the paper are introduced, and \cref{sec:mmd_flow} is devoted to deriving the MMD flow and the associated sampling algorithm.
\cref{sec:theory} investigates at length the theoretical properties of this flow. 


\input{sections/Mmd_flow}




\section{Lambda displacement convexity of the MMD}
In all what follows we consider a fixed target distribution $\mu$ and define the following functional 
\begin{align}\label{eq:MMD_functional}
	\nu \mapsto \F(\nu):=\frac{1}{2} MMD^2(\mu,\nu)\textbf{}
\end{align}. 

We investigate some theoretical properties of the MMD flow. In particular, we are interested in characterizing the convergence of the time discretized flow:
	\begin{align}\label{eq:discretized_flow}
		\nu_{n+1} = (I -\gamma \phi_n)_{\#}\nu_n
	\end{align}
where $\gamma$ is some fixed step-size, $X \mapsto \phi_n(X):=\nabla f_{n}(X)$ is the gradient of the witness function between $\mu$ and $\nu_n$. It is easy to see that the particle version of \cref{eq:discretized_flow_particles} is given by:
\begin{align}
	X_{n+1} = X_n - \gamma \phi_n(X_n) \forall n\in \mathbb{N}.
\end{align} 
One important criterion to characterize the convergence of the gradient flow of a functional $\F$ is the notion of displacement convexity of such functional. Displacement convexity states that the functional evaluated at any distribution in a geodesic path between two distributions $\nu$ and $\nu'$ will be upper-bounded by a convex mixture of $\F(\nu)$ and $\F(\nu')$:
\begin{definition}
(displacement convexity \cite{Villani:2004} Definition 1 ). Let $\mu$
and $\nu$ be two probabilities densities. There exists a $\mu-a.e.$
unique gradient of a convex function ,$\nabla\phi$, such that $\nu$
is equal to $\nabla\phi_{\#}\mu$ and one can define $\rho_{t}=((1-t)Id+t\nabla\phi)_{\#}\mu$
for $0\leq t\leq1$. We say that a functional $\nu\mapsto\mathcal{F}(\nu)$
is displacement convex if 
\[
t\mapsto\mathcal{F}(\rho_{t})
\]
 is convex for any $\mu$ and $\nu$.Moreover, we say that $\mathcal{F}$
is convex in a neighborhood of $\mu$ if there exists a radius $r>0$
such that the above property holds for any $\nu$ with $W_{2}(\mu,\nu)\leq r$
.
\end{definition}


This notion of convexity is to be related to the more widely used notion of convexity called mixture convexity:
\begin{align}
	\F(t\nu +(1-t)\nu')\leq t\F(\nu)+(1-t)\F(\nu') \qquad t\in [0,1]
\end{align}
Unlike, mixture convexity, displacement convexity is compatible with the $W_2$ metric and is therefore the natural notion to use for characterizing convergence of gradient flows in the $W_2$ metric.
Although mixture convexity holds for $\F$ (\cref{lem:mixture_convexity}), this property is less critical for characterizing convergence of gradient flows in the $W_2$ metric. On the other hand, displacement convexity is compatible with the $W_2$ metric \cite{Bottou:2017} and is therefore the natural notion to use in our setting. Unfortunately, $\F$ fails to be displacement convex in general. Instead we will show that $\F$ satisfies some weaker notion of convexity called $\Lambda$-displacement convexity:
%
\begin{definition}
($\Lambda$-convexity \cite{Villani:2009} Definition 16.4). Let $(\mu,v)\mapsto\Lambda(\mu,v)$
be a function that defines for each probability distribution $\mu$
a quadratic form on the set of square integrable vectors valued functions
$v$ , i.e: $v\in L_{2}(\mathbb{R}^{d},\mathbb{R}^{d},\mu)$ . We
further assume that:
\[
\inf_{\mu,v}\frac{\Lambda(\mu,v)}{\Vert v\Vert_{L_{2}(\mu)}^{2}}>-\infty.
\]

We say that a functional $\mu\mapsto\mathcal{F}(\mu)$ is $\Lambda$-convex
if for any $\mu$ and $\nu$ and a minimizing geodesic $\text{\ensuremath{\rho_{t}}}$
between $\mu$ and $\nu$ with velocity vector field $v_{t}$, i.e:
$\partial_{t}\rho_{t}+div(\rho_{t}v_{t})=0;\rho_{0}=\mu;\rho_{1}=\nu$
the following holds:
\end{definition}
\[
\frac{d^{2}\mathcal{F}(\rho_{t})}{dt^{2}}\geq\Lambda(\rho_{t},v_{t})\qquad\forall t\in[0,1].
\]

To show the $\Lambda$-convexity of the functional defined in \cref{eq:MMD_functional} we first make the following assumptions on the kernel:
\begin{assumplist} 
\item \label{assump:bounded_trace} $ \vert \sum_{1\leq i\leq d} \partial_i\partial_ik(x,x) \vert\leq \frac{L}{3}  $ for all $x\in \mathbb{R}^d$.
\item \label{assump:bounded_hessian} $\Vert H_xk(x,y) \Vert_{op} \leq \frac{L}{3}$ for all $x,y\in \mathbb{R}^d$, where $H_xk(x,y)$ is the hessian of $x\mapsto k(x,y)$.
\item \label{assump:bounded_fourth_oder} $\Vert Dk(x,y) \Vert\leq \lambda  $ for all $x,y\in \mathbb{R}$, where $Dk(x,y)$ is an $\mathbb{R}^{d^2}\times \mathbb{R}^{d^2}$ matrix with entries given by $\partial_{x_{i}}\partial_{x_{j}}\partial_{x'_{i}}\partial_{x'j}k(x,x')$.
\end{assumplist}\aknote{do we have an order of magnitude for lambda?}
The next proposition states that the functional defined in \cref{eq:MMD_functional} is $\Lambda$-displacement convex and provide and explicit expression for the functional $\Lambda$:

\begin{proposition}
\label{prop:lambda_convexity} Under \cref{assump:bounded_fourth_oder}, the functional $\nu\mapsto \F(\nu)$ is $\text{\ensuremath{\Lambda}}$-convex
with $\Lambda$ given by:
\begin{equation}
\Lambda(\rho,v)=\langle v,(C_{\rho}-\lambda \F(\rho)^{\frac{1}{2}}I)v\rangle_{L_{2}(\rho)}\label{eq:Lambda}
\end{equation}

where $C_{\rho}$ is the operator defined by:\aknote{sign matters?}
\[
(C_{\rho}v)(x)=\int\nabla_{x}\nabla_{x'}k(x,x')v(x')d\rho(x')
\] and $\lambda$ is defined in \cref{assump:bounded_fourth_oder}.
\end{proposition}
%
%
It is worth noting that $\nu_{0}=\mu$ and at time $t=0$ we have
that $\F(\rho_{0})=0$ and hence we get:
\[
\frac{d^{2}\F(\rho_{t})}{dt^{2}}\vert_{t=0}=\langle v_{t},C_{\rho_{t}}v_{t}\rangle_{L_{2}(\rho_{t})}\geq0.
\]
This shows that $\nu\mapsto \F(\nu)$ has a non-negative
hessian at $\mu$ which is not surprising since $\mu$ is the global
minimum of this functional.
\begin{corollary}\label{cor:integral_lambda_convexity}
For any geodesic $\rho_{t}$ between two probability distributions
$\rho_{0}$ and $\rho_{1}$ the following holds:
\begin{equation}
\F(\rho_{t})\leq(1-t)\F(\rho_{0})+t\F(\rho_{1})-\int_{0}^{1}\Lambda_{\mu}(\rho_{s},v_{s})G(s,t)ds\label{eq:integral_lambda_convexity}
\end{equation}

where $\Lambda_{\mu}$ is given by \cref{eq:Lambda} and $G$ is given
by:
\[
G(s,t)=\begin{cases}
s(1-t) & s\leq t\\
t(1-s) & s\geq t
\end{cases}
\]
\end{corollary}
%
\begin{proof}
This is a direct consequence of the general identity (\cite{Villani:2009},
proposition 16.2). Indeed, for any continuous function $\phi$ on
$[0,1]$ with second derivative $\ddot{\phi}$ that is bounded below
in distribution sense the following identity holds:
\[
\phi(t)=(1-t)\phi(0)+t\phi(1)-\int_{0}^{1}\ddot{\phi}(s)G(s,t)ds
\]
Hence, one can choose $\phi(t)=\F(\rho_{t})$ therefore, \aknote{do we have second derivative bounded below for mmd?}
it follows that:
\[
\F(\rho_{t})=(1-t)\F(\rho_{0})+t\F(\rho_{1})-\int_{0}^{1}\frac{d^{2}\F(\rho_{s})}{ds^{2}}G(s,t)ds
\]
Now using the inequality from \cref{prop:lambda_convexity}, \cref{eq:integral_lambda_convexity}
follows directly. 
\end{proof}
%
\begin{corollary}
\label{cor:loser_bound}Assume the distributions are supported on
$\mathcal{X}$ and the kernel is bounded, i.e: $\sup_{x,y\in\mathcal{X}}\vert k(x,y)\vert<\infty$.
Then the following holds:

\[
\F(\rho_{t})\leq(1-t)\F(\rho_{0})+t\F(\rho_{1})+t(1-t)K
\]
\end{corollary}
%
\begin{proof}
Recall the expression of $\Lambda_{\mu}(\rho_{s},v_{s}):$

\[
\Lambda_{\mu}(\rho_{s},v_{s})=\langle v_{t},(C_{\rho_{t}}-\lambda \F(\rho_{t})^\frac{1}{2} I)v_{t}\rangle_{L_{2}(\rho_{t})}\geq-\lambda \F(\rho_{t})^\frac{1}{2}\Vert v_{t}\Vert_{L_{2}(\rho_{t})}^{2}
\]
However, $\F(\rho_{t})^\frac{1}{2}\leq4C$ where $C=\sup_{x,y\in\mathcal{X}}\vert k(x,y)\vert$
. Moreover, if $\rho_{t}$ is a constant speed geodesic\aknote{give the def at some point} then $\Vert v_{t}\Vert_{L_{2}(\rho_{t})}^{2}=W_{2}^{2}(\rho_{0},\rho_{1})$,
hence: 
\[
-\int_{0}^{1}\Lambda_{\mu}(\rho_{s},v_{s})G(s,t)ds\leq4\lambda CW_{2}^{2}(\rho_{0},\rho_{1})\leq2t(1-t)\lambda Cdiam(\mathcal{X})^{2}
\]\aknote{check}
where $diam(\mathcal{X})$ is the diameter of $\mathcal{X}$. The rest of the proof follows by directly using \cref{cor:integral_lambda_convexity}
and by setting $K=2\lambda Cdiam(\mathcal{X})$.
\end{proof}
%
\cref{cor:loser_bound}, is a loser bound and doesn't account for local
convexity of the MMD. However, it allows to state the following result,
which is inspired from (\cite{Bottou:2017}, Theorem 6.3) but generalizes
it to the case of 'almost convex' functionals.
\begin{proposition}
\label{prop:almost_convex_optimization}
(Almost convex optimization). Let $\mathcal{P}$ be a closed subset
of $\mathcal{P}(\mathcal{X})$ which is displacement convex\aknote{weird for a set to be displacement convex? it was for functionals}. Then
for all $M>\inf_{\rho\in\mathcal{P}}\F(\rho)+K$, the following
holds:
\end{proposition}
\begin{enumerate}
\item The level set $L(\mathcal{P},M)=\{\rho\in\mathcal{P}:\F(\rho)\leq M\}$
is connected
\item For all $\rho_{0}\in\mathcal{P}$ such that $\F(\rho_0)>M$
and all $\epsilon>0$, there exists $\rho\in\mathcal{P}$ such that
$W_{2}(\rho,\rho_{0})=\mathcal{O}(\epsilon)$ and
\[
\F(\rho)\leq \F(\rho_{0})-\epsilon(\F(\rho_{0})-M).
\]
\end{enumerate}
%
\begin{remark}
The result in \cref{prop:almost_convex_optimization} means that it is possible to optimize the cost function $\rho\mapsto \F(\rho)$
on $\mathcal{P}$ as long as the barrier $\inf_{\rho\in\mathcal{P}}\F(\rho)+K$
is not reached. We provide now a simple proof of this result:
\end{remark}
%
\begin{proof}
The proof is very similar to (\cite{Bottou:2017}, Theorem 6.3 and
Theorem 6.9): 
\end{proof}\aknote{$\rho_1$ already taken}
\begin{enumerate}
\item First choose $\rho_{1}\in\mathcal{P}$ such that $\F(\rho_{1})<M-K$.
For any $\rho_{0},\rho_{0}'\in L(\mathcal{P},M)$ there exist a displacement
geodesic joining $\rho_{1}$ and $\rho_{0}$ without leaving $\mathcal{P}$,
since $\mathcal{P}$ is by assumption discplacement convex. By \cref{cor:loser_bound}
we have:
\begin{align*}
\F(\rho_{t}) & \leq(1-t)\F(\rho_{0})+t\F(\rho_{1})+t(1-t)K\\
 & \leq(1-t)M+t(M-K)+t(1-t)K\leq M-t^{2}K\leq M
\end{align*}
Hence $\rho_{t}\in L(\mathcal{P},M)$. The same can be done for a
path joining $\rho_{0}'$ and $\rho_{1}$. Hence we can find a path
in $L(\mathcal{P},M)$ joining $\rho_{0}$ and $\rho_{0}'$ , which
means that the level set $L(\mathcal{P},M)$ is connected.
\item Consider now $\rho_{1}\in L(\mathcal{P},M-K)$, note that such an
element exists since $M>\inf_{\rho\in\mathcal{P}}\F(\rho)+K$.
By convexity of $\mathcal{P}$ there exists a constant speed geodesic
$\rho_{t}$ connecting $\rho_{0}$ and $\rho_{1}$. Since it is a
constant speed curve then one has:
\[
W_{2}(\rho_{0},\rho_{t})\leq tW_{2}(\rho_{0},\rho_{1}).
\]
But we also have by \cref{cor:loser_bound}:
\begin{align*}
\F(\rho_{t}) & \leq(1-t)\F(\rho_{0})+t\F(\rho_{1})+t(1-t)K\\
 & \leq \F(\rho_{0})-t(\F(\rho_{0})-M+tK)\\
 & \leq \F(\rho_{0})-t(\F(\rho_{0})-M)
\end{align*}
Here we simply used the fact that $\rho_{1}\in L(\mathcal{P},M-K)$. 
\end{enumerate}
%

\begin{remark}
	A possible direction would be to directly leverage the tighter inequality in \cref{eq:integral_lambda_convexity} to get a better description of the loss landscape.
\end{remark}

\cref{prop:almost_convex_optimization} guarantees the existence of a direction of descent that minimizes the $\F$ provided that the starting point $\rho_1$ has a potential greater than the barrier $K$, i.e:
\begin{align}\label{eq:barrier_condition}
	\F(\rho_1)> \inf_{\rho\in \mathcal{P}} \F(\rho) + K
\end{align}
One natural question to ask is whether the  discretized gradient flow algorithm provides such way to reach the barrier $K$ and at what speed this happens. The next result answers that question:   

\begin{theorem}
	Consider the sequence of distributions $\nu_n$ obtained from \cref{eq:discretized_flow}. Then the following holds:
	\begin{align}
		\F(\bar{\nu})\leq \frac{1}{2\gamma N}W_2^2(\nu_0,\mu) +K
	\end{align}
	where $\bar{\nu}=\frac{1}{N}\sum_{n=1}^N \nu_n$.
\end{theorem}
\begin{proof}
Let $\Pi^n$ be the optimal coupling between $\nu_n$ and $\mu$, then the optimal transport between $\nu_n$ and $\mu$ is given by:

\begin{align}
	W_2^2(\mu,\nu_n)=\int \Vert X-Y \Vert^2 d\Pi^n(\nu_n,\mu)
\end{align}

Moreover, consider $Z=X-\gamma \phi_n(X)$ where $(X,Y)$ are samples from $\Pi^n$. It is easy to see that $(Z,Y)$ is a coupling between $\nu_{n+1}$ and $\mu$, therefore, by definition of the optimal transport map between $\nu_{n+1}$ and $\mu$ it follows that:
\begin{align}\label{eq:optimal_upper-bound}
	W_2^2(\nu_{n+1},\mu)\leq \int \Vert X-\gamma \phi_{n}(X)-Y\Vert^2 \Pi^n(\nu_n,\mu)
\end{align}
By expanding the r.h.s in \cref{eq:optimal_upper-bound}, the following inequality holds:
\begin{align}\label{eq:main_inequality}
	W_2^2(\nu_{n+1},\mu)\leq W_2^2(\nu_{n},\mu) -2\gamma \int \langle \phi_n(X), X-Y \rangle d\Pi^n(\nu_n,\mu)+ \gamma^2D(\nu_n)
\end{align}
where $D(\nu_n) = \int \Vert \Phi_n(X)\vert^2 d\nu_n $.
An upper-bound on $-2\gamma \int \langle \phi_n(X), X-Y \rangle d\Pi^n(\nu_n,\mu) $ in terms of the loss functional can be obtained using the $\Lambda$ displacement convexity of $\nu\mapsto \F(\nu)$. Indeed, by \cref{lem:grad_flow_lambda_version} it holds that:
\begin{align}\label{eq:flow_upper-bound}
	-2\gamma \int \langle \phi(X),X-Y \rangle d\Pi(\nu,\mu)
	\leq
	-2\gamma\left(\F(\nu)- \F(\mu) +K(\rho^n)\right)
\end{align}
where $(\rho^n_t)_{0\leq t \leq 1}$ is a constant-speed geodesic from $\nu_n$ to $\mu$ and $K(\rho^n):=\int_0^1 \Lambda(\rho^n_s,\dot{\rho^n}_s)(1-s)ds$. Note that when $K(\rho^n)\leq 0$ it falls back to the convex setting.
Therefore, the following inequality holds:
\begin{align}
	W_2^2(\nu_{n+1},\mu)\leq W_2^2(\nu_{n},\mu) - 2\gamma\left(\F(\nu_n)- \F(\mu) +K(\rho^n)\right) +\gamma^2 D(\nu_n)
\end{align}
Now let's introduce a term involving $\F(\nu_{n+1})$. The above inequality becomes:
\begin{align}
	W_2^2(\nu_{n+1},\mu)\leq & W_2^2(\nu_{n},\mu) - 2\gamma\left(\F(\nu_{n+1})- \F(\mu) +K(\rho^n)\right) \\
		&+\gamma^2 D(\nu_n) -2\gamma (\F(\nu_n)-\F(\nu_{n+1}))
	\label{eq:main_ineq_2}
\end{align}
 
It is possible to upper-bound the last two terms by a negative quantity when the step-size is small enough. This is mainly a consequence of the smoothness of the functional $\F$ and the fact that $\nu_{n+1}$ is obtained by following the steepest direction of $\F$ starting from $\nu_n$. \cref{lem:decreasing_functional} makes this statement more precise and allows to get the following inequality:
\begin{align}
	\gamma^2 D(\nu_n) -2\gamma (\F(\nu_n)-\F(\nu_{n+1})\leq -\gamma (1-\gamma L)D(\nu_n).
	\label{eq:decreasing_functional}
\end{align}
Here $L$ is a constant that depends only on the choice of the kernel $k$ in $\F$. Combining \cref{eq:decreasing_functional} and \cref{eq:main_ineq_2} it follows that:
\begin{align}
\F(\nu_{n+1})-\F(\nu)+\frac{\gamma}{2}(1-\gamma L)D(\nu_n)
\leq 
\frac{1}{2\gamma} (W_2^2(\nu_n,\mu)-W_2^2(\nu_{n+1},\mu)-K(\rho^n)
\label{eq:main_final}
\end{align}
Averaging \cref{eq:main_final} for $1\leq n\leq N$ it follows that:
\begin{align}
	\bar{\F}-\F(\mu)+\frac{\gamma}{2}(1-\gamma L)\bar{D} \leq \frac{W_2^2(\nu_0,\mu)}{2 \gamma N} -\bar{K}
\end{align}
where $\bar{\F}$,$\bar{D}$ and $\bar{K}$ denotes the averaged values of $(\F(\nu_n))_{1\leq n \leq N}$, $(D(\nu_n))_{1\leq n \leq N}$  and $(K(\rho^n))_{1\leq n \leq N}$ over iterations from $1$ to $N$. Finally, by \cref{lem:mixture_convexity} it follows that:
\begin{align}
\F(\bar{\nu})-\F(\mu)\leq  \frac{W_2^2(\nu_0,\mu)}{2 \gamma N} -\bar{K}
\end{align}

\end{proof}








\subsubsection*{References}
\renewcommand\refname{\vskip -1cm}
\bibliographystyle{apalike}
\bibliography{biblio}

\newpage


\section{Appendix}

\subsection{Notation and background}
In all what follows, $\X$ is a convex subset of $\R^d$ and $\mathcal{P}_2(\X)$ denotes the set of all probability distributions supported on $\X$ with finite second moment.
For a given distributions $\nu\in\mathcal{P}_2(\X)$ and an integrable function $f$ under $\nu$, the expectation of $f$ under $\nu$ will be written either as $\nu(f)$ or $\int f \diff\nu$ depending on the context. 

\subsubsection{Maximum Mean Discrepancy}\label{subsec:MMD}
For a given characteristic kernel $k$ defined on $\X$, we denote by $\kH$ its corresponding Reproducing Kernel Hilbert Space \manote{some reference here needed}. $\kH$ is a Hilbert space with inner product $\langle .,. \rangle_{\kH}$ and corresponding norm $\Vert . \Vert_{\kH}$. The unit ball in $\kH$ which will be denoted as $\mathcal{B}$ is simply the set of functions $f$ in $\kH$ such that $\Vert f\Vert_{\kH}\leq 1 $:
\begin{align}\label{eq:unit_ball_RKHS}
	\mathcal{B} = \{ f\in \kH : \quad \Vert f\Vert_{\kH}\leq 1 \}
\end{align}
Under mild conditions \manote{write conditions} on the kernel $k$, it is possible to define a distance on $\mathcal{P}_2(\X)$ by finding a function $f$ in $\mathcal{B}$ that maximizes the mean difference between two given distributions $\mu$ and $\nu$. Such distance is called the Maximum Mean Discrepancy  (MMD) \cite{Gretton:2012}:
\begin{align}\label{eq:MMD}
	MMD(\mu,\nu) = \sup_{g\in \mathcal{B}} \int g\diff\mu - \int g \diff\nu
\end{align}
The maximization problem in \cref{eq:MMD} is achieved for an optimal $g^*$ in $\mathcal{B}$ that is proportional to the  witness function between $\nu$ and $\mu$:
\begin{align}\label{eq:witness_function}
	f_{\nu,\mu}(z) = \int k(.,z)\diff \mu - \int k(.,z)\diff \nu  \qquad z\in \X
\end{align}
This allows to express the $MMD$ as the norm of \cref{eq:witness_function} $f_{\nu,\mu}$:
\begin{align}\label{eq:mmd_norm_witness}
	MMD(\mu,\nu) = \Vert f_{\nu,\mu} \Vert_{\mathcal{H}} 
\end{align}
Furthermore, a closed form expression in terms of expectations of the kernel under $\mu$ and $\nu$ can be obtained \cite{gretton2012kernel}:
\begin{align}\label{eq:closed_form_MMD}
	MMD^2(\mu,\nu) = \int k\diff\mu \diff\mu + \int k\diff\nu \diff \nu - 2\int k\diff\mu \diff \nu
\end{align}
When samples from both $\mu$ and $\nu$ are available \cref{eq:closed_form_MMD} can be estimated using those samples. For a fixed target distributions $\mu$ we will consider the loss functional defined as:
\begin{align}\label{eq:loss_functional}
	\F(\nu) = \frac{1}{2} MMD^2(\mu,\nu) \qquad \forall \nu \in \mathcal{P}_2(\X).
\end{align}
We are interested in describing the dynamics of the gradient flow of \cref{eq:loss_functional} under the $2$-Wasserstein metric as defined in \cref{subsec:wasserstein_flow}.
%The MMD was successfully used for training generative models (\cite{mmd-gan,Binkowski:2018,Arbel:2018}) where it is used in a loss functional to learn the parameters of the generator network. This motivate the  
\subsubsection{$2$-Wasserstein geometry}\label{subsec:wasserstein_flow}
For two given probability distributions $\nu$ and $\mu$ in $\mathcal{P}_2(\X)$ we denote by $\Pi(\nu,\mu)$ the set of possible couplings between $\nu$ and $\mu$. In other words $\Pi(\nu,\mu)$ contains all possible distributions $\pi$ on $\X\times \X$ such that if $(X,Y) \sim \pi $ then $X \sim \nu $ and $Y\sim \mu$. The $2$-Wasserstein distance on $\mathcal{P}_2(\X)$ is defined by means of optimal coupling between $\nu$ and $\mu$ in the following way:
\begin{align}\label{eq:wasserstein_2}
	W_2^2(\nu,\mu) := \inf_{\pi\in\Pi(\nu,\mu)} \int \Vert x - y\Vert^2 d\pi(x,y) \qquad \forall \nu, \mu\in \mathcal{P}_2(\X)
\end{align}
It is a well established fact that such optimal coupling $\pi^*$ exists. Moreover, it can be used to define a path $(\rho_t)_{t\in [0,1]}$ between $\nu$ and $\mu$ in $\mathcal{P}_2(\X)$. For a given time $t$ in $[0,1]$ and given a sample $(x,y)$ from $\pi^{*}$, it possible to construct a sample $z_t$ from $\rho_t$ by taking the convex combination of $x$ and $y$: $z_t = s_t(x,y)$ where $s_t$ is given by \cref{eq:convex_combination}
\begin{align}\label{eq:convex_combination}
	s_t = (1-t)x+ty \qquad \forall x,y\in \X, \forall t\in [0,1].
\end{align}
The function $s_t$ is well defined since $\X$ is a convex set. More formally, $\rho_t$ can be written as the projection or push-forward of $\pi^{*}$ by $s_t$:    
  \begin{align}\label{eq:displacement_geodesic}
	\rho_t = (s_t)_{\#}\pi^{*}
\end{align}
It is easy to see that \cref{eq:displacement_geodesic} satisfies the following boundary conditions:
\begin{align}\label{eq:boundary_conditions}
	\rho_0 = \nu \qquad \rho_1 = \mu.
\end{align}
Paths of the form of \cref{eq:displacement_geodesic} are called displacement geodesics. They can be seen as the shortest paths from $\nu$ to $\mu$ in terms of mass transport (\cite{Santambrogio:2015} Theorem 5.27). It can be shown that there exists a vector field $(t,x)\mapsto v_t(x)$ with values in $\R^d$ such that $\rho$ satisfies the continuity equation \manote{reference} :
\begin{align}\label{eq:continuity_equation}
	\partial_t \rho_t + div(\rho_t v_t ) = 0 \qquad \forall t\in[0,1].
\end{align}
\cref{eq:continuity_equation} is well defined in distribution sense even when $\rho_t$ doesn't have a density. $v_t$ can be interpreted as a tangent vector to the curve $(\rho_t)_{t\in[0,1]}$ at time $t$ so that the length $l(\rho)$ of the curve $\rho$ would be given by:
\begin{align}
	l(\rho)^2 = \int_0^1 \Vert v_t \Vert^2_{L_2(\rho_t)} \diff t
\end{align}
where \[
\Vert v_t \Vert^2_{L_2(\rho_t)} =  \int \Vert v_t(x) \Vert^2 \diff \rho_t(x)
\]
This perspective allows to provide a dynamical interpretation of the $W_2$ as the length  of the shortest path from $\nu$ to $\mu$ and is summarized by the celebrated Benamou-Brenier formula (\cite{Santambrogio:2015} 5.28 ):
\begin{align}\label{eq:benamou-brenier-formula}
	W_2(\nu,\mu) = \inf_{(\rho,v)} l(\rho)
\end{align}
where the infimum is taken  over all couples  $\rho$ and $v$ satisfying  \cref{eq:continuity_equation}  with boundary conditions given by \cref{eq:boundary_conditions}.

\begin{remark}
Such paths should not be confused with another kind of paths called mixture geodesics. The mixture geodesic $(\mu_t)_{t\in[0,1]}$ from $\nu$ to $\mu$ is obtained by first choosing either $\nu$ or $\mu$ according to a Bernoulli distribution of parameter $t$ and then sampling from the chosen distribution:
\begin{align}\label{eq:mixture_geodesic}
m_t = (1-t)\nu + t\mu \qquad \forall t \in [0,1].
\end{align}
Paths of the form \cref{eq:mixture_geodesic} can be thought as the shortest paths between two distributions when distances on $\mathcal{P}_2(\X)$ are measured using the $MMD$ (\cite{Bottou:2017} Theorem 5.3). We refer to \cite{Bottou:2017} for an overview of the notion of shortest paths in probability spaces and for the differences between mixture geodesics and displacement geodesics.
Although, we will be interested in the $MMD$ as a loss function, we will not consider the geodesics that are naturally associated to it and we will rather consider the displacement geodesics defined in \cref{eq:displacement_geodesic} for reason that will become clear in \cref{subsec:wasserstein_flow}.
\end{remark}







\subsection{Proof of Proposition~\ref{prop:mmd_flow}}

In the case where $\F=MMD^2$:
\begin{align}
\nabla \frac{\partial \F}{\partial \rho}&= \nabla \frac{\partial \|f_t\|^2_{\kH}}{\partial \rho_t}\\
&=2 \nabla \langle \frac{\partial f_t}{\partial \rho_t}, f_t \rangle_{\kH}\\
&=2 \nabla \langle \frac{\partial \E_{q_t}[k(Y,.)]}{\partial \rho_t}, f_t \rangle_{\kH}\\
&=2 \nabla \langle k(Y,.), f_t \rangle_{\kH}\\
&= 2 \nabla f_t(Y)
\end{align}
where $\nabla f_t(Y)= \E_{X \sim \rho_t}[\nabla_{Y}k(X,Y)] -  \E_{X \sim \pi}[\nabla_{Y}k(X,Y)]$.

\subsection{SDE and stochastic processes}

Consider the Itô process, i.e. the stochastic process:
\begin{equation}
dX_t=g(X_t)dt
\end{equation}
Let $f \in \mathcal{C}^2(\X)$, Itô's formula can be written:
\begin{equation*}
df(X_t)=\nabla f(X_t).g(X_t)dt
\end{equation*}
Let $\rho_t$ be the distribution of the process $X_t$. We have:
\begin{align*}
\E[\frac{df}{dt}(X_t)]&= \E[\nabla f(X_t).g(X_t)]\\
\Longleftrightarrow \int f(X) \frac{d \rho_t}{dt}(X)&=-\int f(X)div(g(X)\rho_t(X))
\end{align*}
where the second line is obtained by integrating by parts on both sides of the equality. Finally, the distribution $\rho_t$ verifies: 
\begin{equation*}
\frac{d\rho_t}{dt}=div(g\rho_t)
\end{equation*}




\subsection{Displacement convexity}


\begin{proof} \ref{prop:lambda_convexity}
To prove that $\nu\mapsto \F(\nu)$ is $\Lambda$-convex
we need to compute the second derivative $\ddot{\F}(\rho_{t})$
where $\rho_{t}$ is a displacement geodesic between two probability
distributions $\nu_{0}$ and $\nu_{1}$ as defined in \cref{eq:displacement_geodesic}. Such minimizing geodesic always exists and can be written as $\rho_t = (s_t)_{\#}\pi$ with $s_t$ defined in \cref{eq:convex_combination} and $\pi$ is an optimal coupling between $\nu_0$ and $\nu_1$ (\cite{Santambrogio:2015}, Theorem 5.27). Moreover, we denote by $v_t$ the corresponding velocity vector as defined in \cref{eq:continuity_equation}. Recall from \cref{eq:mmd_norm_witness} that $\F(\rho_t) = \frac{1}{2} \Vert f_{\mu,\rho_t}\Vert^2_{\mathcal{H}}$, with $f_{\mu,\rho_t}$ defined in \cref{eq:witness_function}. To simplify notations we will write $f_t:= f_{\mu,\rho_t}$. We start by computing the first derivative of $ t\mapsto \F(\rho_t) $. By \cref{lem:derivatives_witness},\cref{eq:derivatives_witness}, we know that $\dot{f}_t$ and $\ddot{f}_t $ are well defined elements of $\kH$ for any given $t\in [0,1]$, hence 
\[
 \dot{\F}(\rho_t) = \langle f_t, \dot{f_t}\rangle_{\kH};\qquad \ddot{\F}(\rho_t) = \Vert \dot{f_t}\Vert^2_{\kH} + \langle f_t, \ddot{f_t}\rangle_{\kH}.
 \]
While $\Vert \dot{f_t}\Vert^2_{\kH}$ is non-negative, $\langle f_t, \ddot{f_t}\rangle_{\kH}$ can in general be negative. We are only interested in quantifying how negative it can get, for this purpose we use Cauchy-Schwartz inequality which directly gives:
\[
\ddot{\F}(\rho_t)\geq  \Vert \dot{f}_t \Vert^2_{\kH} - \Vert f_t \Vert_{\kH}\Vert \ddot{f}_t\Vert_{\kH} 
\]

Finally by \cref{lem:derivatives_witness}, \cref{eq:norm_derivative_witness}, we can conclude that:
\[
	\ddot{\F}(\rho_t)\geq  \langle v_t,(C_{\rho_t} - \lambda \F(\rho_t)^{\frac{1}{2}}) v_t \rangle_{L_2(\rho_t)} 
\]
with $C_{\rho_t}$ given by \cref{eq:positive_operator_C} and $I$ is the identity operator in $L_2(\rho_t)$. Now we can introduce the function:
\begin{align}
	\Lambda(\nu,v) = \langle v ,( C_{\nu} -\lambda \F(\nu)^{\frac{1}{2}} I) v \rangle_{L_2(\nu)} 
\end{align}
which is defined for any pair $(\nu,v)$ with  $\nu\in \mathcal{P}_2(\X)$ and $v$ a square integrable vector field in $L_2(\nu)$. It is clear that $\Lambda(\nu,.)$  is a quadratic form on $L_2(\nu)$. Therefore, form the definition of $\Lambda$ convexity, we conclude that $\F$ is $\Lambda$-convex.
\end{proof}

%
%
%
%By \cref{lem:derivatives_witness}, we have that $\dot{f_t}\in \kH$ and 
%
% it follows from \manote{some assumption to exchange orders}
%\[
%\frac{df_{t}}{dt}=\int(\nabla\phi(x)-x).\nabla k(\pi_{t}(x),.)\nu_{0}(x)dx
%\]
%hence:
%\[
%\frac{dMMD^{2}(\mu,\rho_{t})}{dt}=2\int(\nabla\phi(x)-x).\nabla f_{t}(\pi_{t}(x))\nu_{0}(x)dx
%\]
%Now the second derivative is given by:
%\begin{align*}
%\frac{d^{2}MMD^{2}(\mu,\rho_{t})}{dt^{2}}= & \int(\nabla\phi(x)-x).Hf_{t}(\pi_{t}(x))(\nabla\phi(x)-x)\nu_{0}(x)dx\\
% & +\int(\nabla\phi(x)-x).\nabla_{1}\nabla_{2}k(\pi_{t}(x),\pi_{t}(x'))(\nabla\phi(x')-x')\nu_{0}(x)\nu_{0}(x')dxdx'
%\end{align*}
%Here $\nabla_{1}\nabla_{2}k(x,x')$ is the matrix whose components
%are given by $\langle\partial_{i}k(x,.),\partial_{j}k(x,.)\rangle$
%for $1\leq i,j\leq d$, and $Hf_{t}$ is the hesssian of $f_{t}$
%and its components are also given by:
%\[
%(Hf_{t}(x))_{i,j}=\langle f_{t},\partial_{i}\partial_{j}k(x,.)\rangle.
%\]
%Denoting by $h(x):=\nabla\phi(x)-x$ it follows that:
%\begin{align*}
%\frac{d^{2}MMD^{2}(\mu,\rho_{t})}{dt^{2}}= & \langle f_{t},\int\sum_{i,j}h_{i}(x)h_{j}(x)\partial_{i}\partial_{j}k(\pi_{t}(x),.)\nu_{0}(x)dx\rangle\\
% & +\Vert\int\sum_{i}h_{i}(x)\partial_{i}k(\pi_{t}(x),.)\nu_{0}(x)dx\Vert^{2}
%\end{align*}
%Now we use Cauchy-Schwartz inequality for the first term to get:
%\begin{align*}
%\frac{d^{2}MMD^{2}(\mu,\rho_{t})}{dt^{2}}\geq & -\Vert f_{t}\Vert_{\kH}\Vert\int\sum_{i,j}h_{i}(x)h_{j}(x)\partial_{i}\partial_{j}k(\pi_{t}(x),.)\nu_{0}(x)dx\Vert_{\kH}\\
% & +\Vert\int\sum_{i}h_{i}(x)\partial_{i}k(\pi_{t}(x),.)\nu_{0}(x)dx\Vert^{2}.
%\end{align*}
%After applying a change of variables $x=\pi_{t}(y)$ one recovers the
%velocity vector $v_{t}$ instead of $h$: 
%\begin{align*}
%\frac{d^{2}MMD^{2}(\mu,\rho_{t})}{dt^{2}}\geq & -\Vert f_{t}\Vert_{\kH}\Vert\int\sum_{i,j}v_{t}^{i}(x)v_{t}^{j}(x)\partial_{i}\partial_{j}k(x,.)\rho_{t}(x)dx\Vert_{\kH}\\
% & +\Vert\int\sum_{i}v_{t}^{i}(x)\partial_{i}k(x,.)\rho_{t}(x)dx\Vert^{2}.
%\end{align*}
%
%One can further note that:
%\[
%\Vert\int\sum_{i,j}v_{t}^{i}(x)v_{t}^{j}(x)\partial_{i}\partial_{j}k(x,.)\rho_{t}(x)dx\Vert_{\kH}\leq\lambda\Vert v_{t}\Vert_{L_{2}(\rho_{t})}^{2}
%\]
%
%and that 
%\begin{align*}
%\Vert\int\sum_{i}v_{t}^{i}(x)\partial_{i}k(x,.)\rho_{t}(x)dx\Vert^{2} & =\int v_{t}(x)^{T}\int\nabla_{1}\nabla_{2}k(x,x')v_{t}(x')\rho_{t}(x')dx'dx.\\
% & =\langle v_{t},C_{\rho_{t}}v_{t}\rangle_{L_{2}(\rho_{t})}
%\end{align*}
%
%Hence we have shown that 
%\[
%\frac{d^{2}MMD^{2}(\mu,\rho_{t})}{dt^{2}}\geq\langle v_{t},(C_{\rho_{t}}-\lambda MMD(\mu,\rho_{t})I)v_{t}\rangle_{L_{2}(\rho_{t})}=\Lambda(\rho_{t},v_{t})
%\]


\begin{lemma}\label{lem:derivatives_witness}
Let  $\mu$ ,$\nu_0$ and $\nu_1$ be three distributions in $\mathcal{P}_2(\X)$ and consider a displacement geodesic $(\rho_t)_{t\in[0,1]}$ between $\nu_0$ and $\nu_1$  defined by \cref{eq:displacement_geodesic} 
and its corresponding velocity vector $(v_t)_{t\in [0,1]}$ as defined in \cref{eq:continuity_equation}. The following $3$ statements hold:
\begin{enumerate}
	\item The first and second time derivatives of the witness function $f_{\mu,\rho_t}$ between $\mu$ and $\rho_t$ are well defined elements in $ \kH$ and are given by:
  \begin{align}\label{eq:derivatives_witness}
 	\dot{f}_{\mu,\rho_t} = \int \nabla_1 k(x,.).v_t(x) \diff \rho_t(x); \qquad
 	 \ddot{f}_{\mu,\rho_t} = \int v_t(x)^T\nabla_1^2 k(x,.).v_t(x) \diff \rho_t(x)
 \end{align}
 where $ x \mapsto \nabla_1 k(x,z)$ and $x\mapsto \nabla_1^2 k(x,z)$ denote the gradient of and hessian of $x\mapsto k(x,z)$ for a fixed $z$ in $\X$.
 \item For all $g\in \kH$:
 \begin{align}\label{eq:inner_prod_deriative_witness}
 	\langle g,\dot{f}_{\mu,\rho_t}\rangle_{\kH} = \int \nabla_1 g.v_t \diff \rho_t; \qquad
 	\langle g,  \ddot{f}_{\mu,\rho_t}\rangle_{\kH} = \int v_t^T\nabla_1^2 g.v_t \diff \rho_t
 \end{align}
	\item The RKHS norms of $\dot{f}_{\mu,\rho_t}$ and $\ddot{f}_{\mu,\rho_t}$ satisfy:
	\begin{align}\label{eq:norm_derivative_witness}
 	\Vert \dot{f}_{\mu,\rho_t}\Vert_{\kH}^2 = \langle v_t,C_{\rho_t} v_t \rangle_{L_2(\rho_t)}; \qquad  \Vert \ddot{f}_{\mu,\rho_t} \Vert\leq \lambda \Vert v_t \Vert^2_{L_2(\rho_t)}  
 \end{align}
 with $\lambda$ given by \cref{assump:bounded_fourth_oder} and $C_{\nu}$ defined in \cref{prop:lambda_convexity}. 
\end{enumerate} 
\end{lemma}
\begin{proof}
By definition of $\rho_{t}$:
\[
f_t(z)= \int k(x,z)\diff \mu(x) - \int k(s_t(x,y),z)\diff \pi(x,y)
\]
	\manote{proof}
\end{proof}

\begin{lemma}	\label{lem:grad_flow_lambda_version}
Let $\nu$ be a distributions in $\mathcal{P}_2(\X)$ and $\mu$ be the target distribution such that $\F(\mu)=0$.  Let $\pi$ be an optimal transport from $\nu$ and $\mu$ and $\rho_t$ the displacement geodesic defined by \cref{eq:displacement_geodesic} with its corresponding velocity vector  $v_t$ as defined in \cref{eq:continuity_equation}. The following inequality holds: \manote{This should be a standard result, just need to cite it}
\begin{align*}
	\int \langle \phi(x),y-x \rangle d\pi(x,y)
	\leq
	\F(\mu)- \F(\nu) -\int_0^1 \Lambda(\rho_s,v_s)(1-s)ds
\end{align*}

\end{lemma}
\begin{proof}
Recall that $rho_t$ is given by $\rho_t = (s_t)_{\#}\pi$. By $\Lambda$-convexity of $\mathcal{F}$ the following inequality holds:
	\begin{align*}
		\mathcal{F}(\rho_{t})\leq (1-t)\mathcal{F}(\nu)+t \mathcal{F}(\mu) - \int_0^1 \Lambda(\rho_s,v_s)G(s,t)ds
	\end{align*}
	Hence by bringing $\mathcal{F}(\nu)$ to the l.h.s and dividing by $t$ and then taking its limit at $0$ it follows that:
	\begin{align*}
	\dot{\F}(\rho_t)\vert_{t=0}\leq \mathcal	{F}(\mu)-\mathcal{F}(\nu)-\int_0^1 \Lambda(\rho_s,v_s)(1-s)ds.	
	\end{align*}
	Moreover, by \cref{lem:derivatives_witness}, the time derivative of the witness function between $\nu$ and $\mu$ is well defined, so that $\dot{\F}(\rho_t)$ can be written as:
	\[
	\dot{\F}(\rho_t) = \langle f_{\mu,\rho_t},\dot{f}_{\mu,\rho_t} \rangle_{\kH}
	\]
	Now by \cref{lem:derivatives_witness},\cref{eq:inner_prod_deriative_witness} it follows that:
\[
\dot{\F}(\rho_t) = \int \nabla f_{\mu,\rho_t}(x).v_t(x)\diff \rho_t(x)
\]
By definition of $\rho_t$,  one can further write:
\[
\dot{\F}(\rho_t) = \int \nabla f_{\mu,\rho_t}(s_t(x,y)).(y-x)\diff \pi_(x,y)
\]
We used the fact that $v_t(s_t(x,y))=(y-x)$.\manote{cite something}
 Hence at $t=0$ we get:
\[
\dot{\F}(\rho_t)\vert_{t=0} = \int \nabla f_{\mu,\nu}(x).(y-x)\diff \pi(x,y)
\]
which shows the desired result.
\end{proof}

\begin{lemma}\label{lem:decreasing_functional}
	Under \cref{assump:bounded_trace,assump:bounded_hessian}, the following inequality holds:
	\begin{align*}
		\F(\nu_{n+1})-\F(\nu_n)\leq -\gamma (1-\frac{\gamma}{2}L )\int \Vert \phi_n(X)\Vert^2 d\nu_n
	\end{align*}
\end{lemma}

\begin{proof}
	
Here we consider a path between $\nu_n$ and $\nu_{n+1}$ of the form:
	\begin{align*}
		\rho_t	=(I-\gamma t\phi_n)_{\#}\nu_n
\end{align*}

The function $t\mapsto \mathcal{F}(\rho_t)$ is twice differentiable, hence one can use a Taylor expansion with integral remainder to get:
\begin{align}\label{eq:taylor_expansion}
	\mathcal{F}(\nu_{n+1})-\mathcal{F}(\nu_{n})=\mathcal{F}(\rho_1)-\mathcal{F}(\rho_0) = \frac{d \mathcal{F}(\rho_t) }{dt}\vert_{t=0}+ \frac{1}{2} \int_0^1 \frac{d^2 \mathcal{F}(\rho_t)}{dt^2}(1-t)^2 dt 
\end{align} 
	By \cref{lem:derivative_mmd} we have that:
	\begin{align*}
		\frac{d \mathcal{F}(\rho_t) }{dt} = -\gamma \int \nabla f_n(X).\phi_n(X)d\nu_n(X)=-\gamma \int \Vert \phi_n(X) \Vert^2 d\nu_n(X)
	\end{align*}
	
	Moreover, by \cref{assump:bounded_trace,assump:bounded_hessian} it follows from \cref{lem:derivative_mmd} that:
	\begin{align}\label{eq:upper_bound_1}
		\vert \frac{d^2 \mathcal{F}(\rho_t) }{dt^2}   \vert\leq L\int \Vert \phi_n(X) \Vert^2 d\nu_n(X)
	\end{align}
	Using \cref{eq:taylor_expansion,eq:upper_bound_1} the result follows.
\end{proof}


\begin{lemma}\label{lem:mixture_convexity}
The functional $\F$ is mixture convex: for any probability distributions $\nu_1$ and $\nu_2$ and scalar $1\leq \lambda\leq 1$:
\begin{align*}
	\F(\lambda \nu_1+(1-\lambda)\nu_2)\leq \lambda \F(\nu_1)+ (1-\lambda)\F(\nu_2)
\end{align*}
\end{lemma}
\begin{proof}
	Let $\nu$ and $\nu'$ be two probability distributions and $0\leq \lambda\leq 1$.
	We need to show that \[\mathcal{F}(\lambda \nu + (1-\lambda)\nu') -\lambda \mathcal{F}(\nu) -(1-\lambda)\mathcal{F}(\nu')\leq 0\]
	This follows from a simple computation which shows that:
	\begin{align*}
		\mathcal{F}(\lambda \nu + (1-\lambda)\nu') -\lambda \mathcal{F}(\nu) -(1-\lambda)\mathcal{F}(\nu') = -\frac{1}{2}\lambda(1-\lambda)MMD(\nu,\nu')^2 \leq 0.
	\end{align*}
\end{proof}







\begin{lemma}\label{lem:derivative_mmd}\manote{Notations still needs to be adjusted in this lemma}
	Let $\phi$ be a vector field on $\X$ and $\nu$ in $\mathcal{P}_2(\X)$. Consider the path $\delta_t$ between $\nu$ and $(I+\phi)_{\#}\nu$ given by:
	\begin{align*}
		\delta_t=  (I+t\phi)_{\#}\nu \qquad \forall t\in [0,1]
	\end{align*}
The time derivative of $\mathcal{F}(\delta_t)$ is given by:
	\begin{align*}
		\dot{\F}(\delta_t)&=\int \nabla f_{\mu,\delta_t}(x+t\phi(x)) \phi(x)d\nu(x)\\
	\end{align*}
where $f_{\mu,\delta_t}$ is the witness function between $\mu$ and $\delta_t$ as defined in \cref{eq:witness_function}.	
	Moreover, under \cref{assump:bounded_trace,assump:bounded_hessian}, the second time derivative satisfies:
	
	\begin{align*}
		\ddot{\F}(\delta_t) \vert \leq 3L \int \Vert \phi(x) \Vert^2 d\nu(x)
	\end{align*}
	where $L$ is a positive constant defined in \cref{assump:bounded_trace,assump:bounded_hessian}.
	
\end{lemma}
\begin{proof}
For simplicity, we write $f_t$ instead of $f_{\mu,\delta_t}$.
We start by computing the first derivative. Recalling that $\mathcal{F}(\delta_t)$ is given by $\frac{1}{2}\Vert f_t\Vert^2_{\kH} $, it follows that:
\[
\dot{\F}(\delta_t)=\langle f_{t},\frac{df_{t}}{dt}\rangle_{\kH}
\]. Using the definition
of $\rho_{t}=(I+t\phi)_{\#}\nu_0$ it follows that:
\[
\frac{df_{t}}{dt}=\int \phi(X).\nabla k(\pi_{t}(X),.)d\nu(X)
\]
hence:
\[
\frac{d\mathcal{F}\rho_{t})}{dt}=2\int\phi(X).\nabla f_{t}(\pi_{t}(X))d\nu(X)
\]
Now the second derivative is obtained by direct derivation of the above expression:
	\begin{align*}
		\frac{d^2 \mathcal{F}(\delta_t)}{dt^2} =& \int \phi(X)^THf_t(\pi_t(X))\phi(X)d\nu(X)\\ 
		&+\int \phi(X)^T\nabla_x\nabla_y k(\pi_t(X),\pi_t(X')) ) \phi(X')d\nu(X)d\nu(X') 
	\end{align*}
where $Hf_t$ is the hessian of $f_t$ in space and  $\nabla_x\nabla_y k(x,y)$ is the cross diagonal term of the hessian of $k$. By \ref{assump:bounded_hessian}, the first term in the above equation can be easily upper-bounded by:
\begin{align*}
	4L \int \Vert \phi(X)\Vert^2d\nu(X)  
\end{align*}
The last term can also be upper-bounded by $2L$ by \ref{assump:bounded_trace}.

\end{proof}
 d
Let $  \nu$ and $\nu'$ be two distributions and $\Pi$ a coupling between $\nu$ and $\nu'$. We consider the path $\rho_t$ defined as $\rho_t=(\pi_t)_{\#}\Pi$ where $\pi_t(X,Y)=(1-t)X+tY$. It is possible to provide an expression for the time derivative of $\mathcal{F}{\rho_t}$. This is given by 

%\begin{lemma}\label{lem:time_derivative}
%The time derivative of $\mathcal{F}(\rho_t)$ is given by:
%	\begin{align*}
%		\frac{d \mathcal{F}(\rho_t)}{dt}&=\int \nabla f_t(\pi_t(X)).(Y-X)d\Pi(X,Y)\\
%	\end{align*}
%	where $f_t$ is the witness function at time $t$ and is given by:
%	\begin{align}
%	f_t(x)=\rho_t(k(X,x))-\mu(k(X,x)) \qquad \forall t\in [0,1]
%	\end{align}	
%\end{lemma}
%\begin{proof}
%	The proof is very similar to the one in \cref{lem:derivative_mmd}. Indeed we still have
%	\begin{align*}
%		\frac{d \mathcal{F}(\rho_t)}{dt} = \langle f_t , \frac{df_t}{dt} \rangle
%	\end{align*}
%	And the time derivative of $f_t$ at each point $x\in\mathbb{R}^d$ is obtained by direct computation:
%	\begin{align*}
%		 \frac{df_t}{dt}= \int \nabla k(\pi_t(X,Y),.).(Y-X)d\Pi(X,Y)
%	\end{align*}
%	The result follows using the reproducing property in $\kH$.
% \end{proof}










 


\end{document}